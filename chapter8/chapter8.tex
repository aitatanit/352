\chapter{Electromagnetic Energy and Momentum}\label{energy}
\section{Introduction}
In this chapter, we shall demonstrate that Maxwell's
equations conserve both energy and momentum.

\section{Energy Conservation}\label{s8.2}
We have seen that the energy density of an electric field is given by
[see Equation~(\ref{eeu})]
\begin{equation}
U_E = \frac{\epsilon_0\,E^2}{2},
\end{equation}
whereas the energy density of a magnetic field satisfies 
[see Equation~(\ref{eeb})]
\begin{equation}
U_B = \frac{B^2}{2\mu_0}.
\end{equation}
This suggests that the energy density of a general electromagnetic field is
\begin{equation}
U= \frac{\epsilon_0\,E^2}{2}+\frac{B^2}{2\mu_0}.
\end{equation}
We are now in a position to demonstrate that the classical theory of
electromagnetism conserves energy. We have already come across one conservation
law in electromagnetism: {\em i.e.},
\begin{equation}
\frac{\partial\rho}{\partial t} +\nabla\cdot {\bf j} = 0.
\end{equation}
This is the equation of charge conservation. Integrating over some volume
$V$, bounded by a surface $S$, and making use of Gauss' theorem, we obtain
\begin{equation}
-\frac{\partial}{\partial t}\!\int_V \rho\,d V = \oint_S {\bf j}\cdot d{\bf S}.
\end{equation}
In other words, the rate of decrease of the charge contained in volume $V$ equals
the net flux of charge across surface $S$. This suggests that an energy conservation
law for electromagnetism should have the form
\begin{equation}\label{e8.6}
-\frac{\partial}{\partial t}\!\int_V U\,d V = \oint_S {\bf u}\cdot d{\bf S}.
\end{equation}
Here, $U$ is the energy density of the electromagnetic field, and ${\bf u}$ is
the flux of electromagnetic energy ({\em i.e.}, 
energy  $|{\bf u}|$ per unit time, per unit cross-sectional area, passes a given
point in the direction of ${\bf u}$). According to the above equation, the
rate of decrease of the electromagnetic energy in volume $V$ equals the net flux
of electromagnetic energy across surface $S$.

However, Equation~(\ref{e8.6}) is incomplete, because electromagnetic fields can gain or lose energy
by interacting with matter. We need to factor this into our analysis. 
We saw earlier (see Section~\ref{s4.2}) 
that the rate of heat dissipation per unit volume in a
conductor (the so-called ohmic heating rate) is ${\bf E}\cdot{\bf j}$. 
This energy is extracted from electromagnetic fields, so the rate of energy
loss of the fields in volume $V$ due to interaction with matter is 
$\int_V {\bf E}\cdot{\bf j}\,dV$. Thus, Equation~(\ref{e8.6})  generalizes to 
\begin{equation}
-\frac{\partial}{\partial t} \!\int_V U\,dV= \oint_S {\bf u}\cdot d{\bf S}
+ \int_V {\bf E}\cdot{\bf j}\,dV.
\end{equation}
From Gauss' theorem, the above equation is equivalent to
\begin{equation}
\frac{\partial U}{\partial t} +\nabla\cdot {\bf u} = - {\bf E} \cdot {\bf j}.
\end{equation}
Let us now see if we can derive an expression of this form from Maxwell's equations. 

We start from the differential form of Amp\`{e}re's law (including the displacement current):
\begin{equation}
\nabla\times{\bf B} = \mu_0\, {\bf j} +\epsilon_0\mu_0 \frac{\partial {\bf E}}
{\partial t}.
\end{equation}
Dotting this equation with the electric field yields
\begin{equation}
- {\bf E} \cdot {\bf j} = - \frac{ {\bf E}\cdot \nabla\times{\bf B}}{\mu_0}
+\epsilon_0 \,{\bf E}\cdot \frac{\partial {\bf E}}{\partial t}.
\end{equation}
This can be rewritten
\begin{equation}
- {\bf E} \cdot {\bf j} = - \frac{ {\bf E}\cdot \nabla\times{\bf B}}{\mu_0}
+\frac{\partial}{\partial t}\!\left(\frac{\epsilon_0 \,E^2}{2}\right).
\end{equation}
Now, from vector field theory,
\begin{equation}
\nabla\cdot({\bf E}\times{\bf B}) \equiv {\bf B}\cdot\nabla\times{\bf E} - {\bf E}
\cdot \nabla\times{\bf B},
\end{equation}
so
\begin{equation}
- {\bf E} \cdot {\bf j} = \nabla \!\cdot\!
\left(\frac{{\bf E}\times{\bf B}}{\mu_0}\right)
- \frac{{\bf B}\cdot\nabla\times {\bf E}}{\mu_0} + 
\frac{\partial}{\partial t}\!\left(\frac{\epsilon_0\, E^2}{2}\right).
\end{equation}
The differential form of Faraday's law yields
\begin{equation}
\nabla\times{\bf E} = - \frac{\partial {\bf B}}{\partial t},
\end{equation}
so
\begin{equation}
- {\bf E} \cdot {\bf j} = \nabla \!\cdot\!
\left(\frac{{\bf E}\times{\bf B}}{\mu_0}\right)
+\mu_0^{-1}\,  {\bf B}\cdot\frac{\partial {\bf B}}{\partial t}  + 
\frac{\partial}{\partial t}\!\left(\frac{\epsilon_0\,E^2}{2}\right).
\end{equation}
This can be rewritten
\begin{equation}
- {\bf E} \cdot {\bf j} = \nabla\! \cdot
\!\left(\frac{{\bf E}\times{\bf B}}{\mu_0}\right)
+ \frac{\partial}{\partial t}\!\left( \frac{\epsilon_0\,E^2}{2} +\frac{B^2}{2\mu_0}
\right).
\end{equation}
Thus, we obtain the desired conservation law,
\begin{equation}\label{econs}
\frac{\partial U}{\partial t} + \nabla\cdot{\bf u} = -{\bf E}\cdot {\bf j},
\end{equation}
where 
\begin{equation}\label{e8.18}
U  = \frac{\epsilon_0\,E^2}{2} + \frac{B^2}{2\mu_0}
\end{equation}
is the electromagnetic energy density,
and
\begin{equation}\label{e8.19}
{\bf u} = \frac{{\bf E}\times{\bf B}}{\mu_0}
\end{equation}
is the electromagnetic energy flux. The latter quantity is usually called the 
{\em Poynting flux}, after its discoverer.

Let us see whether our expression for the electromagnetic energy flux makes sense.
We all know that if we stand in the sun we get hot. This
occurs because we absorb electromagnetic radiation emitted by the Sun. So,
radiation must transport energy. The electric and magnetic fields in electromagnetic
radiation are mutually perpendicular, and are also perpendicular to the direction
of propagation $\hat{\bf k}$ (this is a unit vector). Furthermore, $B=E/c$. 
Equation (\ref{e4.71}) can easily be transformed into the following relation between
the electric and magnetic fields of an electromagnetic wave:
\begin{equation}
{\bf E} \times{\bf B} = \frac{E^2}{c}\, \hat{\bf k}.
\end{equation}
Thus, the Poynting flux for electromagnetic radiation is
\begin{equation}\label{e8.21}
{\bf u} = \frac{E^2}{\mu_0\, c} \,\hat{\bf k} = \epsilon_0 \,c \,E^2\, \hat{\bf k}.
\end{equation}
This expression tells us that electromagnetic waves transport energy along their
direction of propagation, which seems to make sense. 

The energy density of electromagnetic radiation is
\begin{equation}
U = \frac{\epsilon_0\,E^2}{2} + \frac{B^2}{2\mu_0} = \frac{\epsilon_0\,E^2}{2}
+ \frac{E^2}{2\mu_0 \,c^2} = \epsilon_0 \,E^2,
\end{equation}
using $B=E/c$.   Note that 
the electric and magnetic fields in an electromagnetic wave have {\em equal}\/ energy densities. 
Since electromagnetic waves travel at the speed of light, we would
expect the energy flux through one square meter in one second to equal the energy
contained in a volume of length $c$ and unit cross-sectional area: {\em i.e.},
$c$ times the energy density. Thus,
\begin{equation}
|{\bf u}| = c\,U = \epsilon_0\, c\,  E^2,
\end{equation}
which is in accordance with Equation~(\ref{e8.21}). 

As another example, consider a straight cylindrical wire of radius $a$, and
uniform resistivity $\eta$. Suppose that the wire is co-axial with the $z$-axis.
Let us adopt standard cylindrical polar coordinates ($r$, $\theta$, $z$). 
If a current of uniform density ${\bf j} = j\,{\bf e}_z$ flows down the
wire then Ohm's law tells us that there is a uniform longitudinal
electric field ${\bf E} = E\,{\bf e}_z$ within the wire, where $E=\eta\,j$. 
According to Amp\`{e}re's circuital law, the current also generates a circulating magnetic field, inside the wire, of the form
${\bf B} = (\mu_0\,r\,j/2)\,{\bf e}_\theta$. Hence, the Poynting flux, ${\bf u} = {\bf E}\times {\bf B}/\mu_0$, within the wire points {\em radially inwards}, and is of magnitude $u= \eta\,j^2\,r/2$. The net energy flux
into a cylindrical surface, co-axial with the wire, of radius $r$ and length $l$ is $U = u\,2\pi\,r\,l=\eta\,j^2\,V(r)$, where $V(r)=\pi\,r^2\,l$ is the volume enclosed by the surface.
However, $\eta\,j^2$ is the rate of electromagnetic energy loss per unit
volume, due to ohmic heating. Hence, $\eta\,j^2\,V(r)$ represents the net rate
of electromagnetic energy loss due to ohmic heating in the region lying within the cylindrical
surface. Thus, we can see that this energy loss is balanced by the inwards flux
of electromagnetic energy across the surface. This flux represents energy which is ultimately derived from the battery which drives the current through
the wire.

In the presence of diamagnetic and magnetic media, starting from 
Equation~(\ref{ejmag}), we can derive an energy conservation law
of the form 
\begin{equation}
\frac{\partial U}{\partial t} + \nabla\cdot{\bf u} = -{\bf E}\cdot {\bf j}_t,
\end{equation}
via analogous steps to those used to derive Equation~(\ref{econs}). Here,
the electromagnetic energy density is written
\begin{equation}
U = \frac{1}{2}\,{\bf E}\cdot{\bf D} + \frac{1}{2}\,{\bf B}\cdot{\bf H},
\end{equation}
which is consistent with Equation~(\ref{e8.18}). The Poynting
flux takes the form
\begin{equation}
{\bf u} = {\bf E}\times{\bf H},
\end{equation}
which is consistent with Equation~(\ref{e8.19}). Of course, the above expressions
are only valid for {\em linear} dielectric and magnetic media.


\section{Electromagnetic Momentum}
We have seen that electromagnetic waves carry energy.
 It turns out that they also carry momentum. Consider the following argument, due
to Einstein. Suppose that we have a railroad car of mass $M$ and length
$L$ which is free
to move in one dimension---see Figure~\ref{f47}. Suppose that electromagnetic radiation of total
energy $E$ is emitted from one end of the car, propagates along the length of
the car, and is then absorbed at the other end. The effective mass of this radiation
is $m= E/c^2$ (from Einstein's famous relation $E=m\, c^2$). At first sight,
the process described above appears to cause the centre of mass of the system 
to spontaneously shift. This violates the law of momentum conservation (assuming the
railway car is subject to no horizontal external forces). The only way in which the
centre of mass of the system can remain stationary is if the railway car
{\em moves} in the opposite direction to the direction of propagation of
the radiation. In fact, if the car moves by a distance $x$ then the centre of
mass of the system is the same before and after the radiation pulse provided that
\begin{equation}
M\, x  = m \,L = \frac{E}{c^2}\, L.
\end{equation}
Incidentally, it is assumed that $m\ll M$ in this derivation.
\begin{figure}
\epsfysize=3.in
\centerline{\epsffile{chapter8/fig8.1.eps}}
\caption{\em Einstein's thought experiment regarding electromagnetic momentum.}\label{f47}
\end{figure}


But, what actually causes the car to move? If the radiation possesses momentum
$p$ then the car will recoil with the same momentum when the radiation is emitted.
When the radiation hits the other end of the car then the car acquires momentum
$p$ in the opposite direction, which stops the motion. The time of flight of
the radiation is $L/c$. So, the distance traveled by a mass $M$ with momentum
$p$ in this time is
\begin{equation}
x = v \,t = \frac{p}{M} \frac{L}{c},
\end{equation}
giving
\begin{equation}\label{e8.26}
p = M\,x\, \frac{c}{L} = \frac{E}{c}.
\end{equation}
Thus, the momentum carried by electromagnetic radiation equals its energy divided by
the speed of light. The same result can be obtained from the well-known
relativistic formula
\begin{equation}
E^2 = p^2 c^2 + m^2 c^4
\end{equation}
relating the energy $E$, momentum $p$, and mass $m$ of a particle. According to
quantum theory, electromagnetic radiation is made up of {\em massless}\/particles
called {\em photons}. Thus,
\begin{equation}
	p = \frac{E}{c}
\end{equation}
for individual photons, so the same must be true of electromagnetic radiation
as a whole. It follows from Equation~(\ref{e8.26})
 that the momentum density $g$ of electromagnetic
radiation equals its energy density over $c$, so
\begin{equation}
g = \frac{U}{c}= \frac{|{\bf u}|}{c^2} = \frac{\epsilon_0 \,E^2}{c}.
\end{equation}
It is reasonable to suppose that the momentum points along the direction
of the energy flow (this is obviously the case for photons),
 so the vector momentum density (which gives the direction,
as well as the magnitude, of the momentum per unit volume) of electromagnetic
radiation is
\begin{equation}\label{e8.30}
{\bf g} = \frac{{\bf u}}{c^2}.
\end{equation}
Thus, the momentum density equals the energy flux over $c^2$. 

Of course, the electric field associated with an electromagnetic wave oscillates
rapidly, which implies  that the previous expressions  for the energy density,
energy flux, and momentum density of electromagnetic radiation are also
rapidly oscillating. It is convenient to average over many periods of
the oscillation (this average is denoted $\langle \,\rangle$). Thus,
\begin{eqnarray}
\langle U\rangle &=& \frac{\epsilon_0\, E_0^{\,2}}{2},\\[0.5ex]
\langle {\bf u}\rangle&=& \frac{c\,\epsilon_0 \,E_0^{\,2}}{2} \,\hat{\bf k}= c \,\langle U\rangle\, \hat{\bf k}, \label{e8.35} \\[0.5ex]
\langle {\bf g} \rangle &=& \frac{\epsilon_0 \,E_0^{\,2}}{2\,c}\,\hat{\bf k} = \frac{\langle U\rangle}{c}
\,\hat{\bf k},
\end{eqnarray}
where the factor $1/2$ comes from averaging $\cos^2 (\omega\, t)$. Here,
$E_0$ is the peak amplitude of the electric field associated with the wave. 

If electromagnetic radiation possesses momentum then it must exert a force on
bodies which absorb (or emit) radiation. Suppose that a body is placed in
a beam of perfectly collimated radiation, which it absorbs completely. The amount
of momentum absorbed per unit time, per unit cross-sectional area, is simply the
amount of momentum contained in a volume of length $c$ and unit cross-sectional
area: {\em i.e.}, $c$ times the momentum density, $g$. An absorbed momentum per
unit time, per unit area, is equivalent to a pressure. In other words, the radiation
exerts a pressure $c\,g$ on the body. Thus, the  {\em radiation pressure} is given by
\begin{equation}\label{e8.32}
p = \frac{\epsilon_0\,E^2}{2} =\, \langle U\rangle.
\end{equation}
So, the pressure exerted by collimated electromagnetic radiation is equal to
its average energy density. 

Consider a cavity filled with electromagnetic radiation. What is the radiation
pressure exerted on the walls? In this situation, the radiation propagates in
all directions with equal probability. Consider  radiation propagating at an
angle $\theta$ to the local normal to the wall. The amount of such radiation
hitting the wall per unit time, per unit area, is proportional to $\cos\theta$.
Moreover, the component of  momentum normal to the wall which the radiation
carries is also proportional to $\cos\theta$. Thus, the pressure exerted on the
wall is the same as in Equation~(\ref{e8.32}), except that it is weighted by the
average of $\cos^2 \theta$ over all solid angles, in order to take into account
the fact 
that obliquely propagating radiation exerts a pressure which is $\cos^2\theta$
times that of normal radiation. The average of $\cos^2\theta$ over all solid angles
is $1/3$, so for isotropic radiation
\begin{equation}
p = \frac{  \langle U\rangle}{3}.
\end{equation}
Clearly, the pressure exerted by isotropic radiation is one third of
its average energy density. 

The power incident on the surface of the Earth due to radiation emitted by
the Sun is about $1300$ W\,m$^{-2}$. So, what is the radiation pressure?
Since,
\begin{equation}
\langle |{\bf u}|\rangle = c\, \langle U\rangle = 1300 \,{\rm W\, m^{-2}},
\end{equation}
then
\begin{equation}
p =\, \langle U\rangle \simeq 4\times 10^{-6}\,{\rm N\, m^{-2}}.
\end{equation}
Here, the radiation is assumed to be perfectly collimated. 
Thus, the radiation pressure exerted on the Earth is minuscule (for comparison, the pressure of the atmosphere 
is  about $10^5$ N\,m$^{-2}$). Nevertheless, this small pressure due to
radiation is important in outer space, since it 
is responsible for continuously sweeping
dust particles out of the Solar System. It is quite common for comets to exhibit
two separate tails. One (called the {\em gas tail}) consists of ionized gas, and is
swept along by the Solar Wind (a stream of charged particles and magnetic field-lines
emitted by the Sun). The other (called the {\em dust tail}) consists of uncharged
dust particles, and is swept radially outward (since light travels in straight-lines) from the Sun by radiation pressure.
Two separate 
tails are observed if  the local direction of the Solar Wind is not radially
outward from the Sun (which is quite often the case). 

The radiation pressure from sunlight is very weak. However, that produced by
laser beams can be enormous (far higher than any conventional pressure which
has ever been produced in a laboratory). For instance, the lasers used in Inertial
Confinement Fusion ({\em e.g.}, the NOVA experiment in
Lawrence Livermore National Laboratory) 
typically have energy fluxes of $10^{18}$ W\,m$^{-2}$. 
This translates to a radiation pressure of about $10^4$ atmospheres!

\section{Momentum Conservation}\label{s8.4}
It follows from Equations~(\ref{e8.19}) and (\ref{e8.30}) that the momentum
density of electromagnetic fields can be written
\begin{equation}\label{e8.41x}
{\bf g} = \epsilon_0\,{\bf E}\times{\bf B}.
\end{equation}
Now, a momentum conservation equation for electromagnetic fields 
should take the integral form
\begin{equation}
-\frac{\partial}{\partial t}\int_V g_i\,dV = \int_S G_{ij}\,dS_j
+ \int_V \left[\rho\,{\bf E} + {\bf j}\times{\bf B}\right]_i\,dV.
\end{equation}
Here, $i$ and $j$ run from 1 to 3 (1 corresponds to the $x$-direction,
2 to the $y$-direction, and 3 to the $z$-direction). Moreover, the Einstein
summation convention is employed for repeated indices ({\em e.g.},
$a_j\,a_j\equiv {\bf a}\cdot{\bf a}$). Furthermore, the tensor
$G_{ij}$ represents the flux of the $i$th component of electromagnetic momentum  in the $j$-direction. This tensor (a tensor is a direct generalization of a vector with two indices instead of one) is called
the {\em momentum flux density tensor}. Hence, the above equation states
that the rate of loss of electromagnetic momentum in some volume $V$
is equal to the flux of electromagnetic momentum  across the bounding surface $S$ plus the rate at which momentum is transferred to matter inside $V$.
The latter rate is, of course, just the net electromagnetic force acting
on matter inside $V$: {\em i.e.}, the volume integral of the electromagnetic
force density, $\rho\,{\bf E} + {\bf j}\times{\bf B}$.
Now, a direct generalization of the divergence theorem states that
\begin{equation}
\int_S G_{ij}\,dS_j \equiv \int_V \frac {\partial G_{ij}}{\partial x_j}\,dV,
\end{equation}
where $x_1 \equiv x$, $x_2\equiv y$, {\em etc.} Hence, in differential form, our
momentum conservation equation for electromagnetic fields is written
\begin{equation}\label{e8.41}
-\frac{\partial}{\partial t}\left[\epsilon_0\,{\bf E}\times{\bf B}\right]_i =
\frac {\partial G_{ij}}{\partial x_j} + [\rho\,{\bf E} + {\bf j}\times{\bf B}]_i.
\end{equation}
Let us now attempt to derive an equation of this form from Maxwell's equations.

Maxwell's equations are as follows:
\begin{eqnarray}
\nabla\cdot{\bf E} &=& \frac{\rho}{\epsilon_0},\label{e8.42}\\[0.5ex]
\nabla\cdot{\bf B} &=& 0,\label{e8.43}\\[0.5ex]
\nabla\times{\bf E} &=& - \frac{\partial {\bf B}}{\partial t},\label{e8.44}\\[0.5ex]
\nabla\times{\bf B}&=& \mu_0\,{\bf j} + \epsilon_0\mu_0\, \frac{\partial {\bf E}}{\partial t}.\label{e8.45}
\end{eqnarray}
We can cross Equation~(\ref{e8.45}) divided by $\mu_0$ with ${\bf B}$, and rearrange, to give
\begin{equation}
-\epsilon_0\,\frac{\partial {\bf E}}{\partial t}\times {\bf B} = \frac{{\bf B}\times(\nabla\times {\bf B})}{\mu_0} + {\bf j}\times {\bf B}.
\end{equation}
Next, let us cross ${\bf E}$ with Equation~(\ref{e8.44}) times $\epsilon_0$,
rearrange, and add the result to the above equation. We obtain
\begin{equation}
-\epsilon_0\,\frac{\partial {\bf E}}{\partial t}\times {\bf B}-\epsilon_0\,
{\bf E}\times\frac{\partial {\bf B}}{\partial t}  = 
\epsilon_0\,{\bf E}\times (\nabla\times{\bf E})+\frac{{\bf B}\times(\nabla\times {\bf B})}{\mu_0} + {\bf j}\times {\bf B}.
\end{equation}
Next, making use of Equations~(\ref{e8.42}) and (\ref{e8.43}), we get
\begin{eqnarray}\label{e8.48}
-\frac{\partial}{\partial t} \left[\epsilon_0\,{\bf E}\times{\bf B}\right]&=&
\epsilon_0\,{\bf E}\times (\nabla\times{\bf E})+\frac{{\bf B}\times(\nabla\times {\bf B})}{\mu_0}\nonumber\\[0.5ex]&& - \epsilon_0\,(\nabla\cdot {\bf E}) \,{\bf E} - \frac{1}{\mu_0}(\nabla\cdot {\bf B}) \,{\bf B}+ 
\rho\,{\bf E}+{\bf j}\times {\bf B}.
\end{eqnarray}
Now, from vector field theory,
\begin{equation}
\nabla (E^2/2) \equiv {\bf E}\times(\nabla\times {\bf E}) + ({\bf E}\cdot{\nabla}){\bf E},
\end{equation}
with a similar equation for ${\bf B}$. Hence, Equation~(\ref{e8.48})
takes the form
\begin{eqnarray}
-\frac{\partial}{\partial t} \left[\epsilon_0\,{\bf E}\times{\bf B}\right]&=&
\epsilon_0\left[\nabla(E^2/2) - (\nabla\cdot {\bf E})\,{\bf E} - ({\bf E}\cdot\nabla){\bf E}\right]\nonumber\\[0.5ex]&& 
+\frac{1}{\mu_0}\left[\nabla(B^2/2) - (\nabla\cdot {\bf B})\,{\bf B} - ({\bf B}\cdot\nabla){\bf B}\right]\nonumber\\[0.5ex]&&+
\rho\,{\bf E}+{\bf j}\times {\bf B}.
\end{eqnarray}
Finally, when written in terms of components, the above equation becomes
\begin{eqnarray}
-\frac{\partial}{\partial t} \left[\epsilon_0\,{\bf E}\times{\bf B}\right]_i&=&
\frac{\partial}{\partial x_j}\!\left[\epsilon_0\,E^2\,\delta_{ij}/2  - \epsilon_0\,E_i\,E_j+ B^2\,\delta_{ij}/2\,\mu_0 - B_i\,B_j/\mu_0\right]\nonumber\\[0.5ex]&&+
\left[\rho\,{\bf E}+{\bf j}\times {\bf B}\right]_i,
\end{eqnarray}
since $[(\nabla\cdot{\bf E})\,{\bf E}]_i \equiv (\partial E_j/\partial x_j)\,E_i$,
and  $[({\bf E}\cdot\nabla){\bf E}]_i \equiv E_j\,(\partial E_i/\partial x_j)$.
Here, $\delta_{ij}$ is a Kronecker delta symbol ({\em i.e.}, $\delta_{ij}=1$
if $i=j$, and $\delta_{ij}=0$ otherwise).
Comparing the above equation with Equation~(\ref{e8.41}), we conclude that
the momentum flux density tensor of electromagnetic fields takes the
form
\begin{equation}\label{e8.55x}
G_{ij} = \epsilon_0\,(E^2\,\delta_{ij}/2-E_i\,E_j) + (B^2\,\delta_{ij}/2-B_i\,B_j)/ \mu_0.
\end{equation}

The momentum conservation equation (\ref{e8.41}) is sometimes written
\begin{equation}
[\rho\,{\bf E} + {\bf j}\times{\bf B}]_i = \frac {\partial T_{ij}}{\partial x_j}
-\frac{\partial}{\partial t}\left[\epsilon_0\,{\bf E}\times{\bf B}\right]_i,
\end{equation}
where 
\begin{equation}
T_{ij} = - G_{ij} = \epsilon_0\,(E_i\,E_j-E^2\,\delta_{ij}/2) + (B_i\,B_j-B^2\,\delta_{ij}/2)/ \mu_0
\end{equation}
is called the {\em Maxwell stress tensor}.

Consider a uniform electric field, ${\bf E} = E\,{\bf e}_z$. 
According to Equation~(\ref{e8.55x}), the momentum flux density
tensor of such a field is
\begin{equation}\label{e8.58x}
{\bf G} = \left(\begin{array}{ccc}\epsilon_0\,E^2/2&0&0\\[0.5ex]
0&\epsilon_0\,E^2/2&0\\[0.5ex]
0&0&-\epsilon_0\,E^2/2\end{array}
\right).
\end{equation}
As is well-known, the momentum flux density tensor of a conventional gas of pressure $p$ is
written
\begin{equation}
{\bf G} = \left(\begin{array}{ccc} p&0&0\\[0.5ex]
0&p&0\\[0.5ex]
0&0&p\end{array}
\right).
\end{equation}
In other words,  from any small volume element there is an equal {\em outward}\/ momentum flux density $p$ in all three Cartesian directions, which simply corresponds to an {\em isotropic}\/ gas pressure, $p$.  This suggests that
a positive diagonal element in a momentum stress tensor corresponds to
a {\em pressure}\/ exerted in the direction of the corresponding Cartesian axis. Furthermore, a negative diagonal element corresponds to negative
pressure, or {\em tension}, exerted in the direction of the corresponding
Cartesian axis. Thus, we conclude, from  Equation~(\ref{e8.58x}), that
electric field-lines act rather like {\em mutually repulsive elastic bands}:
{\em i.e.}, there is a pressure force acting perpendicular to the field-lines which tries to
push them apart, whilst a tension force acting along the field-lines simultaneously tries to shorten them. As an example, we have seen that the normal
electric field $E_\perp$ above the surface of a charged conductor exerts an outward
pressure $\epsilon_0\,E_\perp^{\,2}/2$ on the surface. One way of interpreting this pressure is to say that it is due to the tension in the
electric field-lines anchored in the surface. Likewise, the force of
attraction between the two plates of a charged parallel plate capacitor
can be attributed to the tension in the electric field-lines running
between them.

It is easily demonstrated that the momentum flux density tensor of
a uniform magnetic field, ${\bf B} = B\,{\bf e}_z$, is
\begin{equation}
{\bf G} = \left(\begin{array}{ccc}\,B^2/2\mu_0&0&0\\[0.5ex]
0&B^2/2\mu_0&0\\[0.5ex]
0&0&-B^2/2\mu_0\end{array}
\right).
\end{equation}
Hence, magnetic field-lines also act like mutually repulsive elastic bands.
For instance, the uniform field ${\bf B}$ inside a conventional solenoid exerts an
outward pressure $B^2/2\mu_0$ on the windings which generate and confine it.

\section{Angular Momentum Conservation}
An electromagnetic field which possesses a momentum density ${\bf g}$ must
also possess an {\em angular momentum density}\/
\begin{equation}\label{e8.61x}
{\bf h} = {\bf r}\times {\bf g}.
\end{equation}
It follows that electromagnetic fields can exchange angular momentum, as
well as linear momentum, with ordinary matter. As an illustration of this, consider
the following famous example. 

Suppose that we have two thin co-axial cylindrical conducting shells
of radii $a$ and $c$, where $a< c$. Let the length of both cylinders be
$l$. Suppose, further, that the cylinders are free to rotate independently about their common axis. Finally, let the inner cylinder carry charge $-Q$, and the
outer cylinder charge $+Q$ (where $Q>0$). Now, suppose that a uniform co-axial cylindrical
solenoid winding of radius $b$ (where $a<b<c$), 
and number of turns  per unit length $N$, is placed between the
two cylinders, and energized with a current $I$. Note that the total angular
momentum of this system is a conserved quantity, since the system is
isolated.

Consider an initial state in which
both cylinders are stationary. Ramping  the solenoid current down to zero is observed to cause the two cylinders to start to {\em rotate}---the inner cylinder in
the opposite sense to the sense of current circulation, and the outer cylinder in the same sense.
In other words, the two cylinders acquire angular momentum when the current in the solenoid coil is ramped down. Where does this angular
momentum come from? Clearly, it can only have come from the electric
and magnetic fields in the region between the cylinders.  Let us investigate
further.

It is convenient to define cylindrical polar coordinates ($r$, $\theta$, $z$)
which are co-axial with the common axis of the two cylinders and the solenoid coil.
As is easily demonstrated from Gauss' law, the electric field takes the
form ${\bf E} = E_r\,{\bf e}_r$, where
\begin{equation}
E_r = \left\{
\begin{array}{lcl}
-Q/(2\pi\epsilon_0\,r\,l)&\mbox{\hspace{0.5cm}}&\mbox{for $a\leq r\leq c$}\\[0.5ex]
0&&\mbox{otherwise}
\end{array}
 \right..
\end{equation}
Likewise, it is easily shown from Amp\`{e}re's circuital law
that the initial magnetic field is written ${\bf B} = B_z\,{\bf e}_z$, where
\begin{equation}
B_z = \left\{
\begin{array}{lcl}
\mu_0\,N\,I&\mbox{\hspace{0.5cm}}&\mbox{for $r\leq b$}\\[0.5ex]
0&&\mbox{otherwise}
\end{array}
 \right..
\end{equation}
It follows from Equation~(\ref{e8.41x}) that the initial momentum density
of the electromagnetic field is ${\bf g} = g_\theta\,{\bf e}_\theta$, where
\begin{equation}
g_\theta = \left\{
\begin{array}{lcl}
\mu_0\,N\,I\,Q/(2\pi\,r\,l)&\mbox{\hspace{0.5cm}}&\mbox{for $a \leq r\leq b$}\\[0.5ex]
0&&\mbox{otherwise}
\end{array}
 \right..
\end{equation}
Hence, from Equation~(\ref{e8.61x}), the initial angular momentum density
of the electromagnetic field is ${\bf h} = h_z\,{\bf e}_z$, where
\begin{equation}
h_z = \left\{
\begin{array}{lcl}
\mu_0\,N\,I\,Q/(2\pi\,l)&\mbox{\hspace{0.5cm}}&\mbox{for $a \leq r\leq b$}\\[0.5ex]
0&&\mbox{otherwise}
\end{array}
 \right..
\end{equation}
In other words, the electromagnetic field possesses a {\em uniform}\/ $z$-directed angular momentum density ${\bf h}$ in the region between the
inner cylinder and the solenoid winding. It follows that the initial 
angular momentum content of the electromagnetic field, ${\bf L}$, is equal to ${\bf h}$
multiplied by the volume of this region. Hence, we obtain ${\bf L} = L_z\,{\bf e}_z$, where
\begin{equation}
L_z = \frac{(b^2-a^2)\,\mu_0\,N\,I\,Q}{2}.
\end{equation}
Of course, as the current in the solenoid winding is ramped down this
electromagnetic angular momentum is lost, and, presumably, transferred
to the two cylinders. Let us examine how this transfer is effected.

Any change in the current flowing in the solenoid winding generates
an inductive electric field. From Faraday's law, this field
takes the form ${\bf E} = E_\theta\,{\bf e}_\theta$, where
\begin{equation}
E_\theta = \left\{
\begin{array}{lcl}
-\mu_0\,N\,\dot{I}\,r/2 &\mbox{\hspace{0.5cm}}&\mbox{for $r\leq b$}
\\[0.5ex]
-\mu_0\,N\,\dot{I}\,b^2/(2\,r)&&\mbox{otherwise}
\end{array}
 \right..
\end{equation}
This electric field exerts a  torque ${\bf T}_a=T_a\,{\bf e}_z$, where
\begin{equation}
T_a = -Q\,E_\theta(a)\,a = \frac{\mu_0\,N\,\dot{I}\,Q\,a^2}{2},
\end{equation}
on the inner cylinder, and a  torque ${\bf T}_b=T_b\,{\bf e}_z$, where
\begin{equation}
T_b= Q\,E_\theta(b)\,b =- \frac{\mu_0\,N\,\dot{I}\,Q\,b^2}{2},
\end{equation}
on the outer cylinder. Thus, the net angular momentum
acquired by the inner cylinder, as the current in the solenoid coil
is ramped down, is ${\bf L}_a=L_a\,{\bf e}_z$, where
\begin{equation}
L_a = \int T_a\,dt =  -\frac{\mu_0\,N\,I\,Q\,a^2}{2}.
\end{equation}
Likewise, the net angular momentum acquired by the outer cylinder is
${\bf L}_b = L_b\,{\bf e}_z$, where
\begin{equation}
L_b = \int T_b\,dt =  \frac{\mu_0\,N\,I\,Q\,b^2}{2}.
\end{equation}
Hence, the net angular momentum acquired by the two cylinders,
as a whole, is ${\bf L} = L_z\,{\bf e}_z$, where
\begin{equation}
L_z = \frac{(b^2-a^2)\,\mu_0\,N\,I\,Q}{2}.
\end{equation}
This, of course, is equal to the $z$-directed angular momentum lost
by the electromagnetic field.

{\small
\section{Exercises}
\renewcommand{\theenumi}{8.\arabic{enumi}}
\begin{enumerate}
\item A solenoid consists of a wire wound uniformly around a solid cylindrical
core of radius $a$, length $l$, and permeability $\mu$. Suppose that the
wire has $N$ turns per unit length, and carries a current $I$. What is the energy stored within the core?
Suppose that the current
in the wire is gradually ramped down. Calculate the
integral of the Poynting flux (due to the induced electric field) over the surface of the core. Hence, demonstrate that the instantaneous rate of
decrease of the energy within the core is always equal to the integral of the  Poynting flux over its surface.
\item A co-axial cable consists of two thin co-axial cylindrical conducting
shells of radii $a$ and $b$ (where $a<b$). Suppose that the inner
conductor carries the longitudinal current $I$ and the charge per unit
length $\lambda$. Let the outer conductor carry equal and opposite
current and charge per unit length. What is the flux of electromagnetic energy
and momentum down the cable (in the direction of the inner current)?
What is the electromagnetic pressure acting on the inner and
outer conductors?
\item Consider a co-axial cable consisting of two thin co-axial cylindrical conducting shells of radii $a$ and
$b$ (where $a<b$). Suppose that the inner conductor carries  current per unit
length $I$, circulating in the plane perpendicular to its axis, and
charge per unit length $\lambda$. Let the outer conductor carry equal and
opposite current per unit length and charge per unit length. What is the
flux of electromagnetic angular momentum down the cable? What is the electromagnetic pressure acting on the inner and outer conductors?
\item Calculate the electrostatic force acting between two identical point electrical charges
by finding the net electromagnetic momentum flux across a
plane surface located half-way between the charges. Verify that the result
is consistent with Coulomb's law.
\item Calculate the magnetic force per unit length acting between two long
parallel straight wires carrying identical currents by finding the net electromagnetic momentum flux across a
plane surface located half-way between the wires. Verify that the result is
consistent with Amp\`{e}re's law.
\item Calculate the mean force acting  on a perfect mirror of area $A$ which reflects normally incident electromagnetic radiation of peak electric field $E_0$.
Suppose that the mirror only reflects a fraction $f$ of the incident electromagnetic energy, and absorbs the remainder. What now is the force
acting on the mirror?
\item A thin spherical conducting shell of radius $a$ carries a charge
$Q$. Use the concept of electric field-line tension to find the force of repulsion between any two halves of the shell.
\item A solid sphere of radius $a$ carries a net charge $Q$ which
is uniformly distributed over its volume. What is the force of
repulsion between any two halves of the sphere?
\item A $U$-shaped electromagnet of permeability $\mu\gg 1$, length $l$, pole separation $d$, and uniform  cross-sectional
area $A$ is energized by a current $I$ flowing in a winding with $N$ turns.
Find the force with which the magnet attracts a bar of the same material
which is placed over both poles (and completely covers them).
\item An iron sphere of radius $a$ and uniform magnetization
${\bf M} = M\,{\bf e}_z$ carries an electric charge $Q$. Find the net angular
momentum of the electromagnetic field surrounding the sphere. Suppose that
the magnetization of the sphere decays to zero. Demonstrate that the induced
electric field exerts a torque on the sphere which is such as to impart to it a mechanical angular momentum equal to that lost by the electromagnetic field.
\end{enumerate}
\renewcommand{\theenumi}{arabic{enumi}}
}
