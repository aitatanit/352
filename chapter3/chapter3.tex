\chapter{Time-Independent Maxwell Equations}\label{maxwell1}

\section{Introduction}
In this chapter, we shall recast the familiar force laws of electrostatics and magnetostatics
 as vector field equations.

\section{Coulomb's Law}
Between 1785 and 1787, the French physicist Charles Augustine de Coulomb performed
a series of experiments involving electric charges, and 
eventually established what is
nowadays known  as {\em Coulomb's law}. According to this law, the force
acting between two static electric charges is central, inverse-square,
and proportional to the product of the charges. Two like
charges repel one another,
whereas two unlike charges attract. Suppose that two charges, $q_1$ 
and $q_2$, are 
located at position vectors  ${\bf r}_1$ and ${\bf r}_2$, respectively. The 
electrical force acting
on the second charge is written
\begin{equation}\label{e31}
{\bf f}_2 = \frac{q_1\, q_2}{4\pi\epsilon_0} \frac{{\bf r}_2 - {\bf r}_1}
{|{\bf r}_2-{\bf r}_1|^{3}}
\end{equation}
in vector notation---see Figure~\ref{f24}.  An equal and
opposite force acts on the first charge,
in accordance with Newton's third law of motion.
The SI unit of electric charge is the coulomb (C). The magnitude of the charge on
an electron is $1.6022\times 10^{-19}$ C. Finally, the universal constant $\epsilon_0$
is called the {\em permittivity of free space}, and takes the value
\begin{equation}
\epsilon_0 = 8.8542\times 10^{-12} \,\,{\rm C^{\,2} \,N^{-1}m^{-2}}.
\end{equation}
\begin{figure}
\epsfysize=1.5in
\centerline{\epsffile{chapter3/fig3.1.eps}}
\caption{\em Coulomb's law.}\label{f24}
\end{figure}

Suppose that two masses, $m_1$ and $m_2$, are located at position vectors
${\bf r}_1$ and ${\bf r}_2$, respectively. According to Newton's law
of gravity, the gravitational force acting on the second mass
is written
\begin{equation}\label{e33}
{\bf f}_2 = - G \,m_1 \,m_2\, \frac{{\bf r}_2 - {\bf r}_1}
{|{\bf r}_2 - {\bf r}_1|^{3}}
\end{equation}
in vector notation. The gravitational constant $G$ takes the value
\begin{equation}
G = 6.6726\times 10^{-11}\,\,{\rm N \,m^2 \,kg^{-2}}.
\end{equation}
Note that Coulomb's law has the same mathematical form as Newton's law of gravity.
In particular, they  are both {\em inverse-square} force laws: {\em i.e.},
\begin{equation}
|{\bf f}_2| \propto \frac{1}{|{\bf r}_2- {\bf r}_1|^2}.
\end{equation}
However, these laws differ in two crucial respects. Firstly, the force due to gravity
is always {\em attractive} (there is no such thing as a negative mass). 
Secondly, the magnitudes of the two forces are vastly different. 
Consider the ratio of the electrical and gravitational forces acting on two
particles. This ratio is a constant, independent of the relative positions
of the particles, and is given by
\begin{equation}
\frac{|{\bf f}_{\rm electrical}|}{|{\bf f}_{\rm gravitational}|}
= \frac{|q_1|}{m_1}\frac{|q_2|}{m_2} \frac{1}{4\pi\epsilon_0 \,G}.
\end{equation}
For electrons, the charge to mass ratio is $|q|/m = 1.759\times 10^{11}~{\rm C\, kg^{-1}}$,
so
\begin{equation}
\frac{|{\bf f}_{\rm electrical}|}{|{\bf f}_{\rm gravitational}|}
= 4.17\times 10^{42}.
\end{equation}
This is a colossal number! Suppose we were studying a physics problem involving the motion
of particles  under the action of two forces with the
same range, but differing in magnitude by a factor
$10^{42}$. It would seem a plausible approximation (to say the least) to
start the investgation by neglecting the weaker force altogether. Applying this reasoning to the motion of
particles in the Universe, we would expect the Universe to be governed entirely by 
electrical forces. However, this is not the case. The force which
holds us to the surface of the Earth, and  prevents us  from
floating off into space, is gravity. The force which causes the Earth
to orbit the Sun is also gravity. In fact, on astronomical length-scales gravity is 
the dominant force, and electrical forces are largely irrelevant.
The key to understanding this paradox is that there are both
positive and negative electric charges, whereas there are only positive
gravitational ``charges.'' This means that gravitational forces are always
cumulative,
whereas electrical forces can cancel one another out. Suppose, for the
sake of argument,  that the Universe
starts out with randomly distributed 
 electric charges. Initially, we expect
electrical forces to completely dominate gravity. These forces try to make
 every positive
charge get as far away as possible from the  other positive charges in the Universe, and as close
as possible to the other negative charges. After a while, we  expect 
the positive and
negative charges to form close pairs. Just how close is determined by Quantum
Mechanics, but, in general, it is fairly close: {\em i.e.}, about $10^{-10}$ m.
The electrical forces due to the charges in each pair
effectively  cancel
one another out  on length-scales much larger than the mutual spacing
of the pair. 
However, it is only possible for
gravity to be the dominant long-range force in the Universe if the number
of positive charges is almost equal to the number of
negative charges. In this
situation, every positive charge can find a negative charge to team up with, and
there are virtually no charges left over. In order for the cancellation
of long-range electrical forces to be effective, the relative difference in the
number of positive and negative charges in the Universe must be  incredibly
small. In fact, positive and negative charges have to cancel one another
 to such accuracy that most physicists believe that the net electrical
charge of the Universe is
{\em exactly} zero. But, it is not sufficient for the Universe to start out with zero
charge. Suppose there were some elementary particle process which did not
conserve electric charge. Even if this were to go on at a very low
rate, it would not take long before the fine balance between
positive and negative charges in the Universe was wrecked. So, 
it is important that electric
charge is a {\em conserved}\/ quantity ({\em i.e.}, the net charge of the Universe can neither
increase or decrease). As far as we know, this is the case. To date, no 
elementary particle reactions have been discovered which create or destroy net
electric charge.

In summary,
 there are two long-range forces in the Universe, electricity and gravity.
The former is enormously stronger than the latter, but is usually ``hidden'' away
inside neutral atoms. The fine balance of forces due
to negative and positive electric charges starts to break down on atomic scales.
In fact, interatomic and intermolecular forces are all
  electrical in nature. So, electrical forces
 are basically what prevent us  from
falling though the floor. But, this is electromagnetism on the 
microscopic or atomic scale---what is usually termed  {\em Quantum Electromagnetism}. This book is about
{\em Classical Electromagnetism}. That is, electromagnetism on length-scales much
larger than the atomic scale. Classical Electromagnetism 
generally describes phenomena
in which some sort of ``violence''  is done to matter, so that the
close pairing of negative and positive
charges is  disrupted. This allows electrical forces to manifest
themselves
 on macroscopic length-scales. Of course, very little disruption is necessary
before gigantic forces are generated. Hence, it is no coincidence that the vast majority
of useful machines which humankind has devised during the last century or so
are electrical in nature. 

Coulomb's law and Newton's law are both examples of what are usually referred to as
{\em action at a distance}\/ laws.
  According to Equations~(\ref{e31}) and (\ref{e33}), if the first charge
or mass is moved then the force acting on the second charge or mass 
responds {\em immediately}. In particular, equal and opposite forces act on the two charges or masses
at all times. However, this cannot be correct according to Einstein's Special Theory of
Relativity, which implies that the maximum speed with which information can propagate through
 the Universe
is the speed of light in vacuum. So, if the first charge or mass is moved then there must
always be a time delay ({\em i.e.}, at least the time needed for a light
signal to propagate between the two charges or masses) before the second charge or
mass responds. Consider a rather extreme example. Suppose the first charge or
mass is suddenly annihilated. The second charge or mass only finds out about
this some time later. During this time interval, the second charge or mass
experiences an electrical or gravitational force  which is as if 
the first charge or mass were still there.
 So, during this period, there is an action but no
reaction, which violates Newton's third law of motion.
 It is clear that action at a distance
is not compatible with Relativity, and, consequently,
that Newton's third law of motion is 
 not
strictly true. Of course, Newton's third law is intimately tied up with the
conservation of linear momentum in the Universe. This is a concept which most physicists are loath
to abandon.  It turns out that we can ``rescue'' momentum conservation by abandoning
action at a distance theories, and instead adopting so-called  {\em field theories}\/ in which
there is a medium, called a field, which transmits the force from one particle
to another. Of course, in electromagnetism there are two fields---the electric field,
and the magnetic field. Electromagnetic forces are transmitted via these 
fields at the speed of light, which implies that the laws
of Relativity are  never violated.
Moreover, the fields can soak up energy and  momentum. This means that even when
the actions and reactions acting on charged particles are not quite equal and opposite,
momentum is still conserved. 
We can bypass some of the problematic aspects of  action at a distance by
only considering {\em steady-state}\/ situations. For the moment, this is how we shall
proceed.

Consider $N$ charges, $q_1$ though $q_N$, which are located at position vectors
${\bf r}_1$ through ${\bf r}_N$, respectively. Electrical forces obey what is known as
the {\em principle of superposition}:{\em i.e.}, the electrical force acting on a test charge
$q$ at position vector ${\bf r}$ is simply the vector sum of all of the
Coulomb law forces exerted on it by each of the $N$ charges taken in isolation. In other
words, the electrical force exerted by the $i$th charge (say) on the test charge is
the same as if all of the other charges were not there. Thus, the force acting
on the test charge is given by
\begin{equation}
{\bf f}({\bf r}) = q \sum_{i=1}^N \frac{q_i}{4\pi\epsilon_0}
\frac{{\bf r}-{\bf r}_i}{|{\bf r}-{\bf r}_i|^3}.
\end{equation}
It is helpful to define a vector field ${\bf E}({\bf r})$, called the
{\em electric field}, which is the force exerted on a unit test charge located at
position vector ${\bf r}$. So, the force on a test charge is written
\begin{equation}\label{e39}
{\bf f} = q\,{\bf E},
\end{equation}
and the electric field is given by
\begin{equation}\label{e310}
{\bf E}({\bf r}) = \sum_{i=1}^N \frac{q_i}{4\pi\epsilon_0} 
\frac{{\bf r}- {\bf r}_i}{|{\bf r}-{\bf r}_i|^3}.
\end{equation}
At this point, we have no reason to believe that the electric field has any
real physical existence. It is just a useful device for calculating the force which
acts on test charges placed at  various locations.

The electric field from a single charge $q$ located at the origin is purely radial,
points outwards if the charge is positive, inwards if it is negative, and has
magnitude
\begin{equation}\label{e311}
E_r (r) = \frac{q}{4\pi\epsilon_0\,r^2},
\end{equation}
where $r= |{\bf r}|$.
We can represent an electric field by {\em field-lines}.
 The direction of the lines
indicates 
the direction of the
local electric field, and the density of the lines perpendicular to this direction
is proportional to the magnitude of the local electric field.
It follows from Equation~(\ref{e311}) that the number of field-lines crossing the
surface of a sphere centered on a point charge (which is equal to
$E_r$ times the area, $4\pi\,r^2$, of the surface) is {\em independent}\/
of the radius of the sphere.
Thus, the field of a point positive charge is represented by a group of equally
spaced, unbroken, straight lines radiating from the charge---see Figure~\ref{f23}.
Likewise, field of a point negative charge is represented by a group of
equally spaced, unbroken, straight-lines converging on the charge.
\begin{figure}
\epsfysize=2.in
\centerline{\epsffile{chapter3/fig3.2.eps}}
\caption{\em Electric field-lines generated by a positive charge.}\label{f23}
\end{figure}

The electric 
field from a collection of charges is 
simply the vector sum of the fields
from each of the charges taken in isolation. In other words, electric fields are 
completely {\em superposable}. Suppose that, instead of having discrete charges, we 
have a continuous distribution of charge represented by a {\em charge density}\/
$\rho({\bf r})$. Thus, the charge at position vector ${\bf r}'$ is
$\rho({\bf r}')\,d^3{\bf r}'$, where $d^3{\bf r}'$ is the volume element
at ${\bf r}'$. It follows from a simple extension of Equation~(\ref{e310}) that the electric
field generated by this charge distribution is
\begin{equation}\label{e312}
{\bf E}({\bf r}) =\frac{1}{4\pi\epsilon_0}
 \int \rho({\bf r}')\, \frac{{\bf r}- {\bf r}' }
{|{\bf r} - {\bf r}'|^3} \,d^3{\bf r}',
\end{equation}
where the volume integral is over all space, or, at least, over all space for which
$\rho({\bf r}')$ is non-zero. We shall sometimes refer to the above result
as {\em Coulomb's law}, since it is essentially equivalent to Equation~(\ref{e31}).

\section{Electric  Scalar Potential}
Suppose that ${\bf r} = (x,\,y,\,z)$ and ${\bf r'}= (x',\,y',\,z')$ in Cartesian coordinates.
The $x$ component of $({\bf r} - {\bf r'})/{|{\bf r} - {\bf r'}|^3}$ is
written
\begin{equation}
\frac{x - x'}{[(x-x')^2+(y-y')^2 + (z-z')^2]^{\,3/2}}.
\end{equation}
However, it is easily demonstrated that
\begin{eqnarray}
\frac{x - x'}{[(x-x')^2+(y-y')^2 + (z-z')^2]^{\,3/2}} = \mbox{\hspace{4cm}}
&&\\[0.5ex]
\mbox{\hspace{4cm}}-\frac{\partial}{\partial x}\!\left(
\frac{1}{[(x-x')^2+(y-y')^2 + (z-z')^2]^{\,1/2}}\right).&&\nonumber
\end{eqnarray}
Since there is nothing special about the $x$-axis, we can write
\begin{equation}\label{e315}
\frac{{\bf r}- {\bf r}' }
{|{\bf r} - {\bf r}'|^3} = -\nabla\!\left(\frac{1}{|{\bf r} - {\bf r'}|}\right),
\end{equation}
where $\nabla\equiv (\partial/\partial x,\,\partial/\partial y, \,
\partial/\partial z)$
is a differential operator which involves the components of ${\bf r}$ but not
those of ${\bf r}'$. 
It follows from Equation~(\ref{e312}) that
\begin{equation}\label{e316}
{\bf E} = -\nabla \phi,
\end{equation}
where
\begin{equation}\label{e317}
\phi({\bf r}) = \frac{1}{4\pi\epsilon_0}
\int \frac{ \rho({\bf r}')}{|{\bf r} - {\bf r}'|} \,d^3{\bf r}'.
\end{equation}
Thus, the electric field generated by a collection of fixed charges can be written
as the gradient of a scalar field---known as the {\em electric scalar potential}---and this field can be expressed as a
simple volume integral involving the charge distribution.

The scalar potential generated by a charge $q$ located at the origin is
\begin{equation}
\phi( r) = \frac{q}{4\pi\epsilon_0\,r}.
\end{equation}
According to Equation~(\ref{e310}), the scalar potential generated by a set of $N$
discrete charges $q_i$, located at ${\bf r}_i$, is
\begin{equation}
\phi({\bf r}) = \sum_{i=1}^N \phi_i ({\bf r}),
\end{equation}
where
\begin{equation}\label{e320}
\phi_i({\bf r}) = \frac{q_i}{4\pi\epsilon_0\,|{\bf r} - {\bf r}_i|}.
\end{equation}
Thus,  the scalar potential is just the sum of the potentials generated by each
of the charges taken in isolation.

Suppose that a particle of
charge $q$ is taken along some path from point $P$ to point $Q$.
The net work done on the particle by electrical forces is 
\begin{equation}
{\cal W} = \int_P^Q {\bf f}\cdot d{\bf l},
\end{equation}
where ${\bf f}$ is the electrical force, and $d{\bf l}$ is a line element along the
path. Making use of Equations~(\ref{e39}) and (\ref{e316}), we obtain
\begin{equation}\label{e322}
{\cal W} = q \int_P^Q {\bf E}\cdot d{\bf l} = - q\int_P^Q \nabla\phi\cdot d{\bf l}
= -q \left[\,\phi(Q)- \phi(P)\,\right].
\end{equation}
Thus, the work done on the particle is simply minus its charge times the difference
in electric potential between the end point and the beginning point. This quantity
is clearly {\em independent}\/ of the path taken between $P$ and $Q$. So, an electric field
generated by stationary charges is an example of a {\em conservative}\/ field. In fact, this
result follows immediately from vector field theory once we are told,
in Equation~(\ref{e316}),  that the electric field
is the gradient of a scalar potential. The work done on the particle
when it is taken around a closed loop is zero, so
\begin{equation}
\oint_C {\bf E}\cdot d{\bf l} = 0
\end{equation}
for any closed loop $C$. This implies from Stokes' theorem that
\begin{equation}\label{e324}
\nabla\times {\bf E} = {\bf 0}
\end{equation}
for any electric field generated by stationary charges. Equation~(\ref{e324})
also follows directly 
from Equation~(\ref{e316}), since $\nabla\times\nabla \phi \equiv {\bf 0}$ for any scalar potential
$\phi$. 

The SI unit of electric potential is the volt, which is equivalent to a joule
per coulomb. Thus, according to Equation~(\ref{e322}),
 the electrical work done on a particle when it is
taken between two points is the product of minus its charge and the voltage difference 
between the points. 

We  are familiar with the idea that a particle moving in 
a gravitational field possesses  potential energy as well as  kinetic
energy. If the particle moves from  point $P$ to a lower point $Q$ then the
gravitational field does work on the particle causing its kinetic energy to
increase. The increase in kinetic energy of the particle is balanced by an
equal decrease in its potential energy, so that the overall energy of the
particle is a conserved quantity. Therefore, the work done on the particle
as it moves from $P$ to $Q$ is {\em minus}\/ the difference in its gravitational
potential energy between points $Q$ and $P$. Of course, it only makes sense to
talk about gravitational potential energy because the gravitational field
is  {\em conservative}. Thus, the work done in taking a particle between two
points is {\em path-independent}, and, therefore, well-defined. This means that the
difference in potential energy of the particle between the beginning and end 
 points is also
well-defined.
We have already seen that
an electric field generated by stationary charges is a conservative field.
In follows that 
we can define an electrical potential energy of a particle moving in such a field.
By analogy with gravitational fields, the work done in taking a particle 
from point $P$ to point $Q$ is
equal to minus  the difference in  potential energy of the particle between
points $Q$
and $P$.  It follows from Equation~(\ref{e322}) that 
 the potential  energy of the particle at a general
point $Q$, relative to some reference point $P$ (where the potential energy is set to zero), is given by
\begin{equation}
W (Q)= q \,\phi(Q).
\end{equation}
Free particles try to move down gradients of  potential energy, in order to
attain a
minimum potential energy state. Thus, free particles in the Earth's gravitational
field tend to fall downwards.
Likewise, positive charges moving in an electric field
tend to migrate  towards regions with the most negative
voltage, and {\em vice versa}\/ for negative charges. 

The scalar electric potential is undefined to an additive
constant. So, the transformation
\begin{equation}
\phi({\bf r}) \rightarrow \phi({\bf r}) + c
\end{equation}
leaves the electric field unchanged according to Equation~(\ref{e316}).
The potential can be fixed unambiguously
 by specifying its value at a single point. The
usual convention is to say that the potential  is zero at infinity. This convention
is implicit in Equation~(\ref{e317}), where it can be seen that $\phi\rightarrow 0$ as $|{\bf r}|
\rightarrow\infty$, provided that the total charge $\int \rho({\bf r}')\,d^3{\bf r}'$
is finite. 

\section{Gauss' Law}\label{s33}
Consider a single charge $q$ located at the origin. The electric field generated by
such a charge is given by Equation~(\ref{e311}). Suppose that we surround the charge by 
a concentric spherical surface $S$ of radius $r$---see Figure~\ref{f25}.  The flux of the
electric field through this surface is 
given by
\begin{equation}\label{e327}
\oint_S{\bf E}\cdot d{\bf S}= \oint_S  E_r\, dS_r = E_r(r) \, 4\pi \,r^2 =
\frac{q}{4\pi\epsilon_0\,r^2}\,\,4\pi \,r^2 =
 \frac{q}{\epsilon_0},
\end{equation}
since the normal to the surface is always
parallel to the local        electric field.
\begin{figure}
\epsfysize=2.in
\centerline{\epsffile{chapter3/fig3.3.eps}}
\caption{\em Gauss' law.}\label{f25}
\end{figure}
However, we also know from Gauss' theorem that
\begin{equation}\label{e328}
\oint_S {\bf E}\cdot d{\bf S} = \int_V \nabla\cdot {\bf E} \,\,d^3{\bf r},
\end{equation}
where $V$ is the volume enclosed by surface $S$. Let us evaluate 
$\nabla\cdot {\bf E}$ directly. In Cartesian coordinates, the field is written
\begin{equation}
{\bf E} = \frac{q}{4\pi\epsilon_0} \left(\frac{x}{r^3},\,\, \frac{y}{r^3},
\,\, \frac{z}{r^3}
\right),
\end{equation}
where $r^2=x^2+y^2+z^2$. So,
\begin{equation}\label{e330}
\frac{\partial E_x}{\partial x} = \frac{q}{4\pi\epsilon_0}\left(
\frac{1}{r^3} - \frac{3\,x}{r^4}\frac{x}{r} \right) =
\frac{q}{4\pi\epsilon_0}\frac{r^2-3\,x^2}{r^5}.
\end{equation}
Here, use has been made of
\begin{equation}
\frac{\partial r}{\partial x} = \frac{x}{r}.
\end{equation}
Formulae analogous to Equation~(\ref{e330}) can be obtained for $\partial E_y/\partial y$ and
 $\partial E_z/\partial z$.
The divergence of the field is thus given by
\begin{equation}\label{e332}
\nabla\cdot {\bf E} = \frac{\partial E_x}{\partial x}+
\frac{\partial E_y}{\partial y}+ \frac{\partial E_z}{\partial z}
= \frac{q}{4\pi\epsilon_0} \frac{3\,r^2 - 3\,x^2-3\,y^2-3\,z^2}{r^5} = 0.
\end{equation}
This is a puzzling result! We have from Equations~(\ref{e327}) and (\ref{e328}) that 
\begin{equation}\label{e333}
\int_V \nabla \cdot{\bf E}\,\,d^3{\bf r}= \frac{q}{\epsilon_0},
\end{equation}
and yet we have just proved that $\nabla\cdot {\bf E}=0$. This paradox can be
resolved after a close examination of Equation~(\ref{e332}). At the origin
($r=0$) we find that $\nabla\cdot
{\bf E} = 0/0$, which means that $\nabla\cdot {\bf E}$
can take any value at this point. 
Thus, Equations~(\ref{e332}) and (\ref{e333}) can be reconciled
if $\nabla \cdot {\bf E}$ is some sort
of ``spike'' function: {\em i.e.}, it is zero everywhere except arbitrarily
close to the origin,
where it becomes very large. This
must occur in such a manner that the volume  integral over the spike
is finite. 

Let us examine how we might construct a one-dimensional spike function.
Consider the ``box-car'' function
\begin{equation}
g(x,\epsilon) = \left\{
\begin{array}{lll}
1/\epsilon &\mbox{\hspace{1cm}}& {\rm for~} |x| < \epsilon/2\\
0 &&{\rm otherwise}
\end{array}
\right.
\end{equation}
---see Figure~\ref{f26}.
It is clear that 
\begin{equation}\label{e335}
\int_{-\infty}^{\infty} g(x,\epsilon)\,dx = 1.
\end{equation}
Now consider  the function 
\begin{equation}
\delta(x) = \lim_{\epsilon\rightarrow 0} g(x,\epsilon).
\end{equation}
This is zero everywhere except arbitrarily close to $x=0$. 
According to Equation~(\ref{e335}), 
it also possess a finite integral;
\begin{equation}\label{e337}
\int_{-\infty}^{\infty} \delta (x)\,dx = 1.
\end{equation}
Thus, $\delta(x)$ has all of the required properties of a spike function.
The one-dimensional  spike function $\delta(x)$ is called the 
{\em Dirac delta-function}, after the Cambridge physicist Paul Dirac who 
invented it in 1927 whilst investigating Quantum Mechanics. The delta-function is an
example of what mathematicians call a {\em generalized function}: it is not
well-defined at $x=0$, but its integral is nevertheless well-defined.
Consider the integral
\begin{equation}
\int_{-\infty}^{\infty} f(x)\,\delta(x)\,dx,
\end{equation}
where $f(x)$ is a function which is well-behaved in the vicinity of $x=0$. 
Since the delta-function is zero everywhere apart from very close
to $x=0$, it is clear that
\begin{equation}\label{e339}
\int_{-\infty}^{\infty}f(x)\, \delta(x)\,dx = f(0)\int_{-\infty}^{\infty} \delta(x)
\,dx = f(0),
\end{equation}
where use has been made of Equation~(\ref{e337}). The above equation, which is valid for any
well-behaved
function, $f(x)$, is effectively the definition of a delta-function. 
A simple change of variables allows us to define $\delta(x-x_0)$, which is
a  spike function centred on $x=x_0$. Equation~(\ref{e339}) gives
\begin{equation}
\int_{-\infty}^{\infty} f(x)\,\delta(x-x_0)\,dx = f(x_0).
\end{equation}
\begin{figure}
\epsfysize=2.in
\centerline{\epsffile{chapter3/fig3.4.eps}}
\caption{\em A box-car function.}\label{f26}
\end{figure}

We actually want a three-dimensional spike function:
{\em i.e.}, a function
 which is zero everywhere
apart from arbitrarily close to the origin, and whose volume integral is unity.
If we denote this function by $\delta({\bf r})$ then it is easily seen that
the three-dimensional delta-function is the product of three one-dimensional
delta-functions:
\begin{equation}
\delta({\bf r}) = \delta(x)\,\delta(y)\,\delta(z).
\end{equation}
This function is clearly zero everywhere except the origin. But is its volume
integral unity? Let us integrate over a cube of dimension $2\,a$ which is 
centred on the origin, and aligned along the Cartesian axes. This
volume integral is obviously separable, so that
\begin{equation}
\int \delta({\bf r})\,d^3{\bf r} = \int_{-a}^{a} \delta(x)\,dx
 \int_{-a}^{a} \delta(y)\,dy \int_{-a}^{a} \delta(z)\,dz.
\end{equation}
The integral can be turned into an integral over all space by taking
the limit $a\rightarrow\infty$. However, we know that for one-dimensional
delta-functions $\int_{-\infty}^{\infty} \delta(x)\,dx = 1$, so it follows
from the above equation that
\begin{equation}\label{e343}
\int \delta({\bf r})\,d^3{\bf r} =1,
\end{equation}
which is the desired result. A simple generalization of previous arguments yields
\begin{equation}
\int f({\bf r})\, \delta({\bf r})\,d^3{\bf r} =f({\bf 0}),
\end{equation}
where $f({\bf r})$ is any well-behaved scalar field. Finally, we can change variables
and write
\begin{equation}
 \delta({\bf r} - {\bf r}') = \delta(x-x')\,\delta(y-y')\,\delta(z-z'),
\end{equation}
which is a three-dimensional  spike function centred on 
${\bf r} = {\bf r}'$. It is easily demonstrated that
\begin{equation}
\int f({\bf r})\,\delta({\bf r}- {\bf r}')\,d^3{\bf r}= f({\bf r}').
\end{equation}
Up to now, we have only considered volume integrals taken
 over all space. However, it
should be obvious that the above result  also holds for integrals
over any finite volume $V$ which contains the point ${\bf r} = {\bf r}'$. Likewise,
the integral is zero if $V$ does not contain ${\bf r} = {\bf r}'$.

Let us now return to the problem in hand. The electric field generated by a charge
$q$ located at the origin has $\nabla\cdot{\bf E}=0$ everywhere apart from the
origin, and also satisfies
\begin{equation}
\int_V {\nabla}\cdot{\bf E}\,\,d^3{\bf r} = \frac{q}{\epsilon_0}
\end{equation}
for a spherical volume $V$ centered on the origin. These two facts imply that
\begin{equation}\label{e348}
\nabla\cdot {\bf E} = \frac{q}{\epsilon_0} \,\delta({\bf r}),
\end{equation}
where use has been made of Equation~(\ref{e343}).

At this stage,  vector field theory has yet to show its worth.
After all, we have just spent an inordinately long time proving something using
vector field theory which we 
previously proved in one line [see Equation~(\ref{e327})]  using conventional
analysis. It is time to demonstrate the power of vector field theory. 
Consider, again, a charge $q$ at the origin surrounded by a spherical surface $S$
which is centered on the origin. 
Suppose that we now displace the surface $S$, so that it is no longer centered
on the origin. What is the flux of the electric field out of S? This is no longer
a simple problem for conventional analysis, because the normal to the surface is
not  parallel to the local electric field. However, using  vector field theory
this problem is no more difficult than the previous one. We have
\begin{equation}\label{e349}
\oint_S {\bf E} \cdot d{\bf S} = \int_V \nabla\cdot {\bf E}\,\,d^3{\bf r}
\end{equation}
from Gauss' theorem,
plus Equation~(\ref{e348}).
From these equations, 
it is clear that the flux of ${\bf E}$ out of $S$ is $q/\epsilon_0$ for a
spherical surface displaced from the origin. However, the flux becomes zero when the
displacement is sufficiently large that the origin is no longer enclosed by
the sphere. 
It is possible to prove this via conventional analysis, but it is certainly not easy.
Suppose  that the surface $S$ is not spherical  but is
instead  highly distorted. What now is the flux of ${\bf E}$ out of $S$? This
is a virtually impossible problem in conventional analysis, but it is still easy
using vector field theory. Gauss' theorem and Equation~(\ref{e348}) tell us that
the flux is $q/\epsilon_0$ provided that the surface contains the origin,
and that the flux is zero otherwise. This result is completely independent of the shape of $S$.

Consider $N$ charges $q_i$ located at ${\bf r}_i$. A simple generalization of
Equation~(\ref{e348}) gives
\begin{equation}
\nabla\cdot {\bf E} = \sum_{i=1}^N
 \frac{q_i}{\epsilon_0}\, \delta({\bf r} - {\bf r}_i).
\end{equation}
Thus, Gauss' theorem (\ref{e349}) implies that
\begin{equation}\label{e351}
\oint_S {\bf E}\cdot d{\bf S} = \int_V \nabla\cdot{\bf E}\,\,
d^3{\bf r}= \frac{Q}{\epsilon_0},
\end{equation}
where $Q$ is the total charge enclosed by the surface $S$. This result is called
{\em Gauss' law}, and does
not depend on the shape of the surface. 

Suppose, finally, that instead of having a set of discrete charges, we have a
continuous charge distribution described by a charge density $\rho({\bf r})$.
The charge contained in a small rectangular volume of dimensions $dx$, $dy$, and $dz$
located at position
 ${\bf r}$ is $Q=\rho({\bf r})\,dx\,dy\,dz$. However, if we integrate
$\nabla\cdot {\bf E}$ over this volume element we obtain
\begin{equation}
\nabla \cdot {\bf E} \,\,dx\,dy\,dz = \frac{Q}{\epsilon_0}= 
\frac{\rho\,dx\,dy\,dz}{\epsilon_0},
\end{equation}
where use has been made of Equation~(\ref{e351}). Here, the volume element is assumed to be
sufficiently small that $\nabla\cdot{\bf E}$ does not vary significantly across
it. Thus, we get
\begin{equation}\label{e353}
\nabla \cdot {\bf E} = \frac{\rho}{\epsilon_0}.
\end{equation}
This is the first of four field equations, called Maxwell's equations, which 
 together form 
a complete description of electromagnetism. Of course, our derivation of
Equation~(\ref{e353}) is only valid for electric fields generated by stationary
charge distributions. In principle, additional terms might be required to describe
fields generated by moving charge distributions. However, 
it turns out that this is not the case,
and that Equation~(\ref{e353}) is universally valid. 

Equation~(\ref{e353}) is a differential equation describing the electric field generated
by a set of  charges. We already know the solution to this equation when the
charges are stationary:
it is given by Equation~(\ref{e312}),
\begin{equation}\label{e354}
{\bf E}({\bf r}) = \frac{1}{4\pi \epsilon_0} \int \rho({\bf r}')\,
\frac{{\bf r} - {\bf r}'}{|{\bf r} - {\bf r}'|^3}\,d^3{\bf r}'.
\end{equation}
Equations~(\ref{e353}) and (\ref{e354}) can be reconciled provided 
\begin{equation}\label{e355}
\nabla\cdot\left(\frac{{\bf r} - {\bf r}'}{|{\bf r} - {\bf r}'|^3}\right)
=- \nabla^2\left(\frac{1}{|{\bf r} - {\bf r}'|}\right)= 4\pi\,\delta({\bf r}
-{\bf r}'),
\end{equation}
where use has been made of Equation~(\ref{e315}).
It follows that
\begin{eqnarray}
\nabla\cdot{\bf E}({\bf r})& = &\frac{1}{4\pi\epsilon_0}\int
\rho({\bf r}')\,\nabla\cdot\left(\frac{{\bf r} - {\bf r}'}
{|{\bf r} - {\bf r}'|^3}\right)\,d^3{\bf r}'\nonumber\\[0.5ex]
&=&
\int \frac{\rho({\bf r}')}{\epsilon_0} \,\delta({\bf r} - {\bf r}')\,
d^3{\bf r}' = \frac{\rho({\bf r})}{\epsilon_0},
\end{eqnarray}
which is the desired result. The most general form  of Gauss' law, Equation~(\ref{e351}),
is obtained by integrating Equation~(\ref{e353}) over a volume $V$ surrounded by 
a surface $S$, and making use of
Gauss' theorem:
\begin{equation}
\oint_S {\bf E}\cdot d{\bf S} = \frac{1}{\epsilon_0} \int_V \rho({\bf r})\,d^3{\bf r}.
\end{equation}

One particularly interesting application of Gauss' law is {\em Earnshaw's
theorem}, which states that it is impossible for a collection of charged particles to
remain in static equilibrium solely under the influence of (classical) electrostatic forces.
For instance, consider the motion of the $i$th particle in the
electric field, ${\bf E}({\bf r})$, generated by all of the other static particles.
The equilibrium position of the $i$th particle corresponds to some
point ${\bf r}_i$, where ${\bf E}({\bf r}_i)={\bf 0}$. By  implication,
${\bf r}_i$ does not correspond to the equilibrium position of
any other particle.
However, in order
for ${\bf r}_i$  to be a {\em stable} equilibrium point, the particle
must experience a {\em restoring force} when it moves a small
distance away from ${\bf r}_i$ in {\em any} direction. Assuming that the
$i$th particle is positively charged, this means that the electric
field  must  point radially towards
${\bf r}_i$ at all neighbouring points. Hence, if we apply Gauss' law to a small
sphere centred on ${\bf r}_i$ then there must be a negative flux of
${\bf E}$ through the surface of the sphere, implying the presence of a negative
charge at ${\bf r}_i$. However, there is no such charge at ${\bf r}_i$.
Hence, we conclude that ${\bf E}$ cannot point radially towards ${\bf r}_i$
at all neighbouring points. In other
words, there must be some neighbouring points at which ${\bf E}$ is directed {\em away}
from ${\bf r}_i$. Hence, a positively charged particle
placed at ${\bf r}_i$ can always escape by moving to such points.
One corollary of Earnshaw's theorem is that classical electrostatics cannot
account for the stability of atoms and molecules.

As an example of the use of Gauss' law, let us calculate the electric field
generated by a spherically symmetric charge annulus of inner radius $a$,
and outer radius $b$, centered on the origin, and carrying a uniformly
distributed charge $Q$. Now, from symmetry, we expect a spherically
symmetric charge distribution to generated a spherically symmetric
potential, $\phi(r)$. It therefore follows that the electric field is both
spherically symmetric and radial: {\em i.e.}, ${\bf E} = E_r(r)\,{\bf e}_r$.
Let us apply Gauss' law to an imaginary spherical surface, of radius
$r$, centered on the origin---see Figure~\ref{fgauss}. Such a surface is generally known as a {\em Gaussian surface}. Now, according to
Gauss' law, the flux of the electric field out of the surface is equal to
the enclosed charge, divided by $\epsilon_0$. The flux is easy to calculate since the electric field is everywhere perpendicular to the
surface. We obtain
$$
4\pi\,r^2\,E_r(r) = \frac{Q(r)}{\epsilon_0},
$$
where $Q(r)$ is the charge enclosed by a Gaussian surface of radius $r$.
Now, it is evident that
\begin{equation}
Q(r) = \left\{ 
\begin{array}{lcc}
0&\mbox{\hspace{1cm}}&r<a\\[0.5ex]
[(r^3-a^3)/(b^3-a^3)]\, Q&&a\leq r\leq b\\[0.5ex]
Q&&b<r
\end{array}
\right..
\end{equation}
Hence,
\begin{equation}
E_r(r) = \left\{ 
\begin{array}{lcc}
0&\mbox{\hspace{1cm}}&r<a\\[0.5ex]
[Q/(4\pi\epsilon_0\,r^2)]\,[(r^3-a^3)/(b^3-a^3)]&&a\leq r\leq b\\[0.5ex]
Q/(4\pi\epsilon_0\,r^2)&&b<r
\end{array}
\right..
\end{equation}
The above electric field distribution illustrates two important points. Firstly,
 the electric field generated outside a spherically symmetric
charge distribution is the same that which would be generated if all of the charge in the distribution
were concentrated at its center. Secondly, zero electric field is generated inside
an empty cavity surrounded by a spherically symmetric charge distribution.
\begin{figure}
\epsfysize=2.in
\centerline{\epsffile{chapter3/fig3.5.eps}}
\caption{\em An example use of Gauss' law.}\label{fgauss}
\end{figure}

We can  easily determine the electric potential associated with the above 
electric field using
\begin{equation}
\frac{\partial\phi(r)}{\partial r} = -E_r(r).
\end{equation}
The boundary conditions are that $\phi(\infty)=0$, and that $\phi$ is
{\em continuous}\/ at $r=a$ and $r=b$. (Of course, a discontinuous potential would lead to
an infinite electric field, which is unphysical.) It follows that
\begin{equation}
\phi(r) = \left\{ 
\begin{array}{lcc}
[Q/(4\pi\epsilon_0)]\,(3/2)\,[(b^2-a^2)/(b^3-a^3)]&\mbox{\hspace{0.5cm}}&r<a\\[0.5ex]
[Q/(4\pi\epsilon_0\,r)]\,[(3b^3\,r-r^3-2\,a^3)/2\,(b^3-a^3)]&&a\leq r\leq b\\[0.5ex]
Q/(4\pi\epsilon_0\,r)&&b<r
\end{array}
\right..
\end{equation}
Hence, the work done in slowly moving a charge from infinity to the
center of the distribution (which is minus the work done by the electric field)
is
\begin{equation}
W = q\,\left[\phi(0)-\phi(\infty)\right] = \frac{q\,Q}{4\pi\epsilon_0}\,\frac{3}{2}\left(\frac{b^2-a^2}{b^3-a^3}\right).
\end{equation}

\section{Poisson's Equation}
We have seen that the electric field generated by a set of stationary charges can be written as
the gradient of a scalar potential, so  that
\begin{equation}
{\bf E} = - \nabla \phi.
\end{equation}
This equation can be combined with the field equation (\ref{e353}) to give a partial
differential equation for the scalar potential:
\begin{equation}\label{e359}
\nabla^2 \phi = -\frac{\rho}{\epsilon_0}.
\end{equation}
This is an example of a very famous type of partial differential equation known
as {\em Poisson's equation}.

In its most general form, Poisson's equation is written
\begin{equation}
\nabla^2 u = v,
\end{equation}
 where $u({\bf r})$ is some scalar potential which is
to be determined, and
$ v({\bf r})$ is a known ``source function.'' The most common boundary condition
applied to this equation is that the potential $u$  is zero at infinity.
The solutions to Poisson's equation are completely superposable. Thus, if
$u_1$ is the potential generated by the source function $v_1$, and
$u_2$ is the potential generated by the source function $v_2$, so that
\begin{equation}
\nabla^2 u_1 = v_1, \mbox{\hspace{1cm}} \nabla^2 u_2=v_2,
\end{equation}
then the potential generated by $v_1 + v_2$ is $u_1+u_2$, since
\begin{equation}
\nabla^2(u_1+u_2) = \nabla^2 u_1 + \nabla^2 u_2 = v_1 + v_2.
\end{equation}
Poisson's equation has this property because it is {\em linear}\/ in both the
potential and the source term.

The fact that the solutions to Poisson's equation are superposable suggests a
general method for solving this equation. Suppose that we could construct all
of the  solutions
generated by point sources. Of course, these solutions
must satisfy the appropriate boundary conditions.
Any general source function can be built up out of a set of suitably weighted
point sources, so the general solution of Poisson's equation must be
expressible as a similarly weighted sum over the point source solutions. Thus, once we
know all of the point source solutions we can construct any other solution.
In mathematical terminology, we require the solution to
\begin{equation}\label{e363}
\nabla^2 G({\bf r}, \,{\bf r}') = \delta({\bf r} - {\bf r}')
\end{equation}
which goes to zero as $|{\bf r}|
 \rightarrow\infty$. The function $G({\bf r}, \,{\bf r}')$
is the solution generated by a unit point source located at position ${\bf r}'$.
This function is known to mathematicians as a {\em Green's function}. The solution
generated by a general source function $v({\bf r})$ is simply the 
appropriately weighted sum of
all of the Green's function solutions:
\begin{equation}\label{e364}
u({\bf r}) = \int G({\bf r},\, {\bf r}')\, v({\bf r}')\,d^3 {\bf r}'.
\end{equation}
We can easily demonstrate that this is the correct solution:
\begin{equation}
\nabla^2 u({\bf r}) = \int \left[\nabla^2 
G({\bf r},\, {\bf r}')\right]\, v({\bf r}')\,d^3 {\bf r}'
= \int \delta({\bf r} - {\bf r}')\,v({\bf r}')\,d^3 {\bf r}' = v({\bf r}).
\end{equation}

Let us return to Equation~(\ref{e359}):
\begin{equation}\label{e366}
\nabla^2 \phi = -\frac{\rho}{\epsilon_0}.
\end{equation}
The Green's function for this equation satisfies Equation~(\ref{e363}) with $|G|\rightarrow \infty$
as $|{\bf r}|\rightarrow 0$. It follows from Equation~(\ref{e355}) that
\begin{equation}\label{e367}
G({\bf r}, {\bf r}') = -\frac{1}{4\pi} \frac{1}{|{\bf r} - {\bf r}'|}.
\end{equation}
Note, from Equation~(\ref{e320}), that the Green's function has the same form as the potential
generated by a point charge. This is hardly surprising, given the definition of
a Green's function. It follows from Equation~(\ref{e364}) and (\ref{e367}) that the general solution
to Poisson's equation, (\ref{e366}), is written
\begin{equation}
\phi({\bf r}) = \frac{1}{4\pi\epsilon_0}\int \frac{\rho({\bf r}')}
{|{\bf r} - {\bf r}'|} \,d^3{\bf r}'.
\end{equation}
In fact, we have already obtained this solution by another method [see Equation~(\ref{e317})].

\section{Amp\`{e}re's Experiments}
As legend has it, in 1820 the Danish physicist Hans Christian \O rsted was giving a lecture 
demonstration of various electrical and
magnetic effects. Suddenly, much to his surprise, he noticed that
the needle of a compass he was holding
was deflected  when he moved it close to  a current carrying
wire. Up until then, magnetism has been thought of as solely a property of some rather
unusual rocks called loadstones.
Word of this discovery spread quickly along the scientific grapevine,
and the French physicist  Andre Marie Amp\`{e}re 
immediately decided to investigate further. 
Amp\`{e}re's apparatus consisted (essentially) of a long straight wire carrying an
electric 
current $I$. Amp\`{e}re quickly discovered that the needle of a small compass maps
out a series of concentric circular loops in the plane
perpendicular to  a current carrying wire---see Figure~\ref{f27}.
The direction of circulation around these magnetic loops is conventionally taken to be
the direction in which the {\em North}\/ pole of the compass needle
points.
Using  this convention, the circulation of the loops is given by a
right-hand rule: if the thumb of the right-hand points along the direction of the
current then the fingers of the right-hand circulate in the same sense as the 
magnetic loops.
\begin{figure}
\epsfysize=2.5in
\centerline{\epsffile{chapter3/fig3.6.eps}}
\caption{\em Magnetic loops around a current carrying wire.}\label{f27}
\end{figure}

Amp\`{e}re's next series of experiments involved bringing a short test wire, carrying
a current $I'$,
close to the original wire, and investigating the force exerted on the test wire---see Figure~\ref{f28}.
This experiment is not quite as clear cut as Coulomb's experiment because, unlike
electric charges, 
electric currents cannot exist as point entities---they
have to flow in complete circuits. We must
imagine that the circuit which connects with  the central wire is sufficiently
far away that it has no appreciable influence on the outcome of the experiment.
The circuit which connects with
 the test wire is more problematic. Fortunately, if the
feed wires are twisted around each other, as indicated in Figure~\ref{f28}, then 
they effectively cancel one another out, and also do not influence the outcome of
the experiment.
\begin{figure}
\epsfysize=2.25in
\centerline{\epsffile{chapter3/fig3.7.eps}}
\caption{\em Amp\`{e}re's experiment.}\label{f28}
\end{figure}

Amp\`{e}re discovered that the force exerted on the test wire is directly proportional
to its length. He also made the following observations.
If the current in the test wire 
({\em i.e.}, the test current) flows parallel to the current in the central wire 
then the two wires  attract one another. If the current in the test
wire is reversed then the two wires  repel one another.
If the test current points radially towards the central wire 
(and the current in the central wire flows upwards) then the test wire
is subject to a downwards force. If the test current is reversed then the force is
upwards. If the test current is rotated in a single plane, so that it starts
parallel to the central current and ends up pointing radially
towards it, then the force on
the test wire is of constant magnitude, and is always at right-angles to the
test current. If the test current is parallel to a magnetic loop then there is
no force exerted on the test wire. If the test current is rotated in
a single plane, so that it starts parallel to the central current and ends up
pointing along a magnetic loop, then the magnitude of the force on the
test wire attenuates like $\cos\theta$ (where $\theta$ is the angle the current
is turned through---$\theta=0$ corresponds to the
case where the test current is parallel to the central current),
 and its direction is again always at right-angles to
the test current. Finally, Amp\`{e}re was able to establish that the attractive
force between two parallel current carrying wires is proportional to the product of
the two currents, and
 falls off like the inverse of the perpendicular
distance between the wires.

This rather complicated force law can be summed up succinctly in vector notation
provided that we define a vector field ${\bf B}({\bf r})$, called the {\em magnetic field}, 
whose direction is always parallel to the loops mapped out by a small
  compass. The dependence of the force per unit length, ${\bf F}$, acting on a
test wire with the different 
possible orientations of the test current is described  by 
\begin{equation}
{\bf F} = {\bf I}' \times{\bf B},\label{e369}
\end{equation}
where ${\bf I}'$ is a vector whose direction and magnitude are the same as those
of the test current. Incidentally, the SI unit of electric current is the
ampere (A), which is the same as a coulomb per second.
The SI unit of magnetic field-strength is the tesla (T), which is the
same as a newton per ampere per meter.
The variation of the force per unit length acting on
a test wire with the strength of the 
central current and the perpendicular distance $r$ to the central wire is 
summed up by saying that the magnetic field-strength is proportional to $I$ and
inversely proportional to $r$. Thus, defining cylindrical
polar coordinates aligned along the
axis of the central current, we have
\begin{equation}
B_\theta = \frac{\mu_0\,I}
{2 \pi\, r},
\end{equation}
with $B_r=B_z = 0$. The constant of proportionality $\mu_0$ is called the
{\em permeability of free space}, and takes the value
\begin{equation}
\mu_0 = 4\pi \times 10^{-7}\,{\rm N\,A^{-2}}.
\end{equation}

The concept of a magnetic field allows the calculation of the force on a test
wire to be conveniently split into two parts. In the first part, we calculate the
magnetic field generated by the current flowing in the central wire. This field
circulates  in the plane normal to the wire: its magnitude is
proportional to the central current, and inversely proportional to the  perpendicular
distance from the wire. In the second part, we use
Equation~(\ref{e369}) to calculate the force per unit
length acting on a
short current carrying wire located in the magnetic field 
generated by the central current.
This force is perpendicular to both the magnetic field and the direction of the
test current. Note that, at this stage, we have no reason to suppose that the magnetic
field has any real physical existence. It is introduced merely to facilitate the calculation
of the  force exerted  on  the test wire by the central wire.

\section{Lorentz Force}\label{slorentz}
The flow of an electric current down
a conducting  wire is ultimately due to  the motion of
electrically charged particles
(in most cases, electrons) through the conducting medium.
 It seems reasonable, therefore, that
the force exerted on the wire when it is placed in a magnetic field is really
the resultant of the  forces exerted on these moving charges. Let us
suppose that this is the case. 

Let $A$ be the 
(uniform) cross-sectional area of the (cylindrical) wire, and let $n$ be the number density
of mobile charges in the conductor. Suppose that the
mobile charges each have charge $q$ and velocity ${\bf v}$. We must assume that
the conductor also contains stationary charges, of charge $-q$ and number density
$n$ (say), so that the net charge density in the wire is zero. In most conductors, the
mobile charges are electrons, and the stationary charges are ions.
The magnitude of the electric current flowing through the wire is simply the
number of coulombs per second which flow past a given point. In one second,
a mobile charge moves a distance $v$, so all of the charges contained in a
cylinder of cross-sectional area $A$ and length $v$ flow past a given point.
Thus, the magnitude of the current is $q\,n\, A\,v$. The direction of the 
current is the same as the direction of motion of the charges, so the
vector current is ${\bf I}' = q\,n\,A\,{\bf v}$.
According to Equation~(\ref{e369}), the force per unit length acting on the wire is
\begin{equation}
{\bf F} = q\,n\, A \,{\bf v}\times{\bf B}.
\end{equation}
However, a unit length of the wire contains $n\,A$ moving charges. So, assuming
that each charge is subject to an equal force from the magnetic field (we have
no reason to suppose otherwise), the force acting on an individual charge is
\begin{equation}
{\bf f} = q\,{\bf v} \times{\bf B}.
\end{equation}
We can combine this with Equation~(\ref{e39}) to give the force acting on a charge $q$ moving
with velocity ${\bf v}$ in an electric field ${\bf E}$ and a magnetic field
${\bf B}$:
\begin{equation}\label{e374}
{\bf f} = q\,{\bf E} + q\,{\bf v} \times{\bf B}.
\end{equation}
This is called the {\em Lorentz force law}, after the Dutch physicist
Hendrik Antoon Lorentz who first formulated it. The electric
force on a charged particle is parallel to the local electric field.
The magnetic force, however, is perpendicular to both the local magnetic
field and the particle's direction of motion. No magnetic force is exerted on a
stationary charged particle.

The
equation of motion of a free particle of charge $q$ and
mass $m$ moving in electric and
magnetic fields is
\begin{equation} \label{e3.75}
m\frac{d{\bf v}}{dt} = q\,{\bf E} + q\,{\bf v} \times{\bf B},
\end{equation}
according to the Lorentz force law.
This equation of motion was first verified in a famous experiment carried out
by the Cambridge physicist J.J.~Thomson in 1897. Thomson was investigating
{\em cathode rays}, a then mysterious form of radiation emitted by a heated
metal element held at a large negative voltage ({\em i.e.}, a cathode) with respect
to another metal element ({\em i.e.}, an anode)  in an evacuated tube. 
German physicists held that cathode rays were
a form of electromagnetic radiation, whilst British and French physicists suspected
that they were, in reality, a stream of charged particles. Thomson was able to
demonstrate that the latter view was correct. In Thomson's experiment, the
cathode rays passed though a region of ``crossed'' electric and magnetic
fields (still in vacuum). The fields were perpendicular to the original
trajectory of the rays, and were also mutually perpendicular.

Let us analyze  Thomson's experiment.  Suppose that
the rays are originally traveling in the $x$-direction, and are subject to
a uniform electric field $E$ in the   $z$-direction and a uniform magnetic
field $B$ in the $-y$-direction---see Figure~\ref{fthom}. Let us assume, as Thomson did, that cathode
rays are a stream of particles of mass $m$ and charge $q$. The
equation of motion of the particles in the $z$-direction is
\begin{equation}
m \,\frac{d^2 z}{dt^2} = q\left(E - v \,B\right),\label{e376}
\end{equation}
where $v$ is the velocity of the particles in the $x$-direction.
Thomson started off his experiment by
only turning on the  electric field in his apparatus, and
measuring the
deflection $d$ of the ray in the $z$-direction after it had traveled a
distance  $l$ through the electric field. It is clear from the equation
of motion that 
\begin{equation}\label{e377}
d = \frac{q}{m} \frac{E\,t^2}{2} = \frac{q}{m} \frac{E\,l^2}{2\,v^2},
\end{equation}
where the ``time of flight'' $t$ is replaced by $l/v$. This formula is only
valid if $d\ll l$, which is assumed to be the case. 
Next, Thomson  turned on
the magnetic field in his apparatus, and adjusted it so that the cathode ray was
no longer deflected. The lack of deflection implies that the net force on the
particles in the $z$-direction was zero. In other words, the electric and
magnetic forces balanced exactly. It follows from Equation~(\ref{e376})
that with a properly adjusted magnetic field-strength
\begin{equation}\label{e378}
v = \frac{E}{B}.
\end{equation}
Equations~(\ref{e377}) and (\ref{e378})
  can be combined and rearranged to give the charge to mass ratio of
the particles in terms of measured quantities:
\begin{equation}
\frac{q}{m} = \frac{2\,d\, E}{l^2 \,B^2}.
\end{equation}
Using this method, Thomson inferred that cathode rays were made up of
negatively charged particles (the sign of the charge is obvious from the
direction of the deflection in the electric field) with a charge to mass
ratio of $-1.7\times 10^{11}$~C/kg. A decade later, in 1908, the American Robert
Millikan performed his famous ``oil drop'' experiment, and discovered that
mobile electric charges are quantized in units of $-1.6\times 10^{-19}$~C. 
Assuming that mobile electric charges and the particles which
make up cathode rays are one and the same thing,
 Thomson's and Millikan's experiments imply that the mass
of  these particles is  $9.4\times 10^{-31}$~kg. Of course, this is the mass of
an electron (the modern value is $9.1\times 10^{-31}$~kg), and  
$-1.6\times 10^{-19}$~C is the charge of an electron. Thus, cathode rays are, in fact,
streams of electrons which are  emitted from a heated cathode, and then
accelerated because of  the large voltage difference between the cathode and anode.
\begin{figure}
\epsfysize=2.5in
\centerline{\epsffile{chapter3/fig3.8.eps}}
\caption{\em Thomson's experiment.}\label{fthom}
\end{figure}

Consider, now, a particle of mass $m$ and charge $q$ moving in a uniform
magnetic field, ${\bf B} = B\,{\bf e}_z$. According, to
Equation~(\ref{e3.75}), the particle's equation of motion can be written:
\begin{equation}
m\,\frac{d{\bf v}}{dt} = q\,{\bf v}\times{\bf B}.
\end{equation}
This reduces to
\begin{eqnarray}
\frac{dv_x}{dt} &=& {\mit\Omega}\,v_y,\\[0.5ex]
\frac{dv_y}{dt} &=&-{\mit\Omega}\,v_x,\\[0.5ex]
\frac{d v_z}{dt} &=& 0.
\end{eqnarray}
Here, ${\mit\Omega} = q\,B/m$ is called the {\em cyclotron frequency}.
The above equations can easily be solved to give
\begin{eqnarray}
v_x&=& v_\perp\,\cos({\mit\Omega}\,t),\\[0.5ex]
v_y&=& - v_\perp\,\sin({\mit\Omega}\,t),\\[0.5ex]
v_z &=& v_\parallel,
\end{eqnarray}
and
\begin{eqnarray}
x&=& \frac{v_\perp}{\mit\Omega}\,\sin({\mit\Omega}\,t),\\[0.5ex]
y&=& \frac{v_\perp}{\mit\Omega}\,\cos({\mit\Omega}\,t),\\[0.5ex]
z &=& v_\parallel\,t.
\end{eqnarray}
According to these equations, the particle trajectory is a {\em spiral}\/
whose axis is parallel to the magnetic field---see Figure~\ref{flamor}. The radius of the
spiral is $\rho=v_\perp/{\mit\Omega}$, where $v_\perp$ is the particle's
constant speed in the plane perpendicular to the magnetic field. Here, $\rho$
is termed the {\em Larmor radius}.
The particle
drifts parallel to the magnetic field at a constant velocity, $v_\parallel$. Finally,
the particle gyrates in the plane perpendicular to the magnetic field at the cyclotron
frequency. Oppositely charged particles gyrate in opposite directions.
\begin{figure}
\epsfysize=0.75in
\centerline{\epsffile{chapter3/fig3.9.eps}}
\caption{\em Trajectory of a charged particle in a uniform magnetic field.}\label{flamor}
\end{figure}

Finally, if a particle is subject to a force ${\bf f}$, and moves a distance 
$\delta
{\bf r}$ in a time interval $\delta t$, then the work done on the particle by the
force is
\begin{equation}\label{e380}
\delta W = {\bf f}\cdot \delta {\bf r}.
\end{equation}
Thus, the power input  to the particle from the force field is
\begin{equation}
P = \lim_{\delta t\rightarrow 0} \frac{\delta W}{\delta t} = {\bf f}\cdot
 {\bf v},
\end{equation}
where ${\bf v}$ is the particle's velocity. It follows from the Lorentz force
law, Equation~(\ref{e374}), that the power input to a particle moving in electric and magnetic
fields is
\begin{equation}
P = q\,{\bf v}\cdot {\bf E}.
\end{equation}
Note that a charged particle can gain (or lose) energy from an electric
field, but not from a magnetic field. This is because the magnetic force is
always perpendicular to the particle's direction of motion, and, therefore, does
no work on the particle [see Equation~(\ref{e380})]. Thus, in particle accelerators, magnetic
fields are often used to guide particle motion ({\em e.g.}, in a circle), but the
actual acceleration is always performed by electric fields.

\section{Amp\`{e}re's Law}
Magnetic fields, like electric fields, are completely superposable. So, if
a field ${\bf B}_1$ is generated by a current $I_1$ flowing through some circuit,
and a field ${\bf B}_2$ is generated by a current $I_2$ flowing through another
 circuit, then when the currents $I_1$ and $I_2$ flow through both circuits
simultaneously the generated magnetic field is ${\bf B}_1+{\bf B}_2$.

Consider two parallel wires separated by a perpendicular distance $r$
and carrying electric currents $I_1$ and $I_2$, respectively---see Figure~\ref{f29}. The magnetic field
 strength at the second wire due to the current flowing in the first wire 
is $B = \mu_0\, I_1/2\pi \,r$. This field is orientated at right-angles to the second
wire, so the force per unit length exerted on the second wire is
\begin{equation}
F= \frac{\mu_0\, I_1 \,I_2}{2\pi \,r}.\label{e383}
\end{equation}
This follows from Equation~(\ref{e369}), which is valid for continuous wires as well as short
test wires. The force acting on the second wire is directed radially inwards towards
the first wire (assuming that $I_1\,I_2>0$). The magnetic field strength at the first wire due to the
current flowing in the second wire is $B= \mu_0\, I_2/2\pi\ r$. This field
is orientated at right-angles to the first wire, so the force per unit length acting
on the first wire is equal and opposite to that acting on the second wire, 
according to Equation~(\ref{e369}).  Equation~(\ref{e383}) is sometimes called {\em Amp\`{e}re's law},
and is clearly another example of an action at a distance law: {\em i.e.}, if the
current in the first wire is suddenly changed then the force on the second wire
immediately adjusts. In reality, there should be a short time delay, at
least as long as the propagation time for a light signal between the two wires. 
Clearly, Amp\`{e}re's law is not strictly correct. However, as long as we restrict
our investigations to {\em steady}\/ currents it is perfectly adequate.  
\begin{figure}
\epsfysize=1.75in
\centerline{\epsffile{chapter3/fig3.10.eps}}
\caption{\em Two parallel current carrying wires.}\label{f29}
\end{figure}

\section{Magnetic Monopoles?}
Suppose that we have an infinite straight wire carrying an electric current
$I$. Let the wire be aligned along the $z$-axis. The magnetic field
generated by such a  wire is written
\begin{equation}\label{e384}
{\bf B} = \frac{\mu_0 \,I}{2\pi}\left(\frac{-y}{r^2},\, \frac{x}{r^2},\, 0 \right)
\end{equation}
in Cartesian coordinates, where $r=\sqrt{x^2+y^2}$. The divergence of this
field is
\begin{equation}
\nabla\cdot {\bf B} = \frac{\mu_0\, I}{2\pi}\left( \frac{2\,y\,x}{r^4}-\frac{2\,x\,y}{r^4}
\right) = 0,
\end{equation}
where use has been made of $\partial r/\partial x = x/r$, {\em etc.} We saw in Section~\ref{s33}
that the divergence of the electric field appeared, at first sight, to be zero, but,
was, in reality,  a delta-function, because the volume integral of
$\nabla\cdot {\bf E}$ was non-zero. Does the same
sort of thing happen for the divergence of the magnetic field? Well, if we could
find a closed surface $S$ for which $\oint_S {\bf B}\!\cdot \!d{\bf S} \neq 0$ then,
according to Gauss' theorem, $\int_V \nabla \cdot {\bf B} \,dV \neq 0 $, where 
$V$ is the volume enclosed by $S$. This would certainly imply
that $\nabla\cdot {\bf B} $ is some sort of delta-function. So, can we
find such a surface? Consider a cylindrical surface
aligned  with the wire. The magnetic field is everywhere tangential to the
outward surface element, so this surface certainly has zero magnetic flux coming
out of it. In fact, it is impossible to invent any closed surface for which
$\oint_S {\bf B}\cdot d{\bf S} \neq 0$ with ${\bf B}$ given by Equation~(\ref{e384}) (if
you do not believe this, just try and find one!). This suggests  that the divergence
of a magnetic field generated by steady electric currents
really is zero. Admittedly, we
have only proved this for infinite straight currents, but, as will be demonstrated
presently, it is true in general.

If  $\nabla\cdot {\bf B}=0$ then  ${\bf B}$ is a
{\em solenoidal} vector field. In other words,  field-lines of ${\bf B}$
never begin or end.  This is certainly the
case in Equation~(\ref{e384}) where the field-lines are a set of concentric circles centred
on the $z$-axis.  What about magnetic
fields generated by permanent magnets (the modern equivalent of loadstones)?
Do they also never begin or end? Well, we know that a conventional bar
magnet has both a North and South magnetic pole (like the Earth). If we track the
magnetic field-lines with a small compass they  all emanate from the South
pole, spread out, and eventually reconverge on the North pole---see Figure~\ref{f30}. It appears likely
(but we cannot prove it with a compass) that the field-lines inside the magnet
connect from the North to the South pole so as to form closed loops which never begin or end.
\begin{figure}
\epsfysize=2.5in
\centerline{\epsffile{chapter3/fig3.11.eps}}
\caption{\em Magnetic field-lines generated by a bar magnet.}\label{f30}
\end{figure}

Can we produce an isolated North or South magnetic pole: for instance, by snapping
a bar magnet in two? A compass needle would always point towards an isolated
North pole, so this would act  like a negative ``magnetic charge.'' 
Likewise, a compass needle would always point away from an isolated South
pole, so this would act like a positive ``magnetic charge.'' 
It is clear, from Figure~\ref{f31},
that if we take  a closed surface $S$ containing an isolated
magnetic pole, which is usually termed a {\em magnetic monopole}, then 
$\oint_S {\bf B}\cdot d{\bf S} \neq 0$: the flux will be positive for an isolated
South pole, and negative for an isolated North pole. 
It follows from Gauss'
theorem that if $\oint_S {\bf B}\cdot d{\bf S} \neq 0$ then $\nabla\cdot{\bf B}
\neq 0$. Thus, the  statement that magnetic fields are solenoidal, or that
$\nabla\cdot{\bf B} = 0$, is equivalent to  the statement that {\em
there are no magnetic monopoles}. It is not clear, {\em a priori}, that this is
a true statement. In fact, it is quite possible to formulate electromagnetism so as to
allow for magnetic monopoles. However, as far as we are aware, there are no  magnetic
monopoles in the Universe. 
At least, if there are any then they are all 
hiding from us!
We know that if we try to make a magnetic monopole  by snapping a bar magnet in two
then we just end up with two smaller bar magnets. If we snap one of these smaller
magnets in two then we end up with two even smaller bar magnets. We can continue
this process down to the atomic level without ever producing a magnetic monopole.
In fact, permanent magnetism 
is generated by electric currents circulating on the atomic
scale, and so this type of
magnetism is not fundamentally different to the magnetism generated
by macroscopic currents. 
\begin{figure}
\epsfysize=2.25in
\centerline{\epsffile{chapter3/fig3.12.eps}}
\caption{\em Magnetic field-lines generated by magnetic monopoles.}\label{f31}
\end{figure}

In conclusion, {\em all}\/ steady magnetic fields in the Universe are generated by 
circulating electric currents of some description. Such fields are solenoidal:
that is, they never begin or end, and satisfy the field equation
\begin{equation}
\nabla \cdot {\bf B} = 0.
\end{equation}
This, incidentally,
 is the second of Maxwell's equations. Essentially, it says that there is no
such thing as a magnetic monopole. We have only proved that $\nabla\cdot{\bf B} =0$
 for
steady magnetic fields, but, in fact, this is also the case  for time-dependent fields
(see later).

\section{Amp\`{e}re's Circuital Law}
Consider, again, an infinite straight wire aligned along the $z$-axis and  carrying
a current $I$. The field generated by such a wire is written
\begin{equation}
B_\theta = \frac{\mu_0\, I}{2\pi \,r}
\end{equation}
in cylindrical polar coordinates. Consider a circular loop $C$ in the $x$-$y$ plane
which is centred on the wire. Suppose that the radius of this loop is $r$.
Let us evaluate the line integral $\oint_C {\bf B} \cdot d{\bf l}$.
This integral is easy to perform because the magnetic field is
always  parallel to the
line element. 
We have
\begin{equation}\label{e388}
\oint_C {\bf B} \cdot d{\bf l} = \oint B_\theta \,r\,d\theta= \mu_0 \,I.
\end{equation}
However, we know from Stokes' theorem that
\begin{equation}\label{e389}
\oint_C {\bf B} \cdot d{\bf l} = \int_S \nabla\times {\bf B} \cdot d{\bf S},
\end{equation}
where $S$ is any surface attached to the loop $C$. 

Let us evaluate $\nabla\times {\bf B} $ directly. According to Equation~(\ref{e384}),
\begin{eqnarray}\label{e390a}
(\nabla\times{\bf B})_x &=& \frac{\partial B_z}{\partial y} - \frac{\partial B_y}
{\partial z} = 0,\\[0.5ex]
(\nabla\times{\bf B})_y &=&  \frac{\partial B_x}{\partial z} - \frac{\partial B_z}
{\partial x} = 0,\\[0.5ex]
(\nabla\times {\bf B})_z &=&  \frac{\partial B_y}{\partial x} - \frac{\partial B_x}
{\partial y} \nonumber\\[0.5ex]&&= \frac{\mu_0\, I}{2\pi} \left(
\frac{1}{r^2} - \frac{2\,x^2}{r^4} +\frac{1}{r^2} -\frac{2 \,y^2}{r^4}\right) =0,\label{e390c}
\end{eqnarray}
where use has been made of $\partial r/\partial x = x/r$, {\em etc.} We now have a problem.
Equations (\ref{e388}) and (\ref{e389}) imply that
\begin{equation}
\int_S \nabla\times {\bf B} \cdot d{\bf S} = \mu_0 \,I.\label{e391}
\end{equation}
But,  we have just demonstrated that $\nabla\times{\bf B} = {\bf 0}$. This problem
is very reminiscent of the difficulty we had earlier with ${\nabla}\cdot {\bf E}$. 
Recall that $\int_V {\nabla}\cdot {\bf E} \,dV= q/\epsilon_0$ for a volume $V$ containing
a discrete charge $q$, but that $\nabla \cdot{\bf E} = 0$ at a general point.
We got around this problem
 by saying that $\nabla\cdot {\bf E}$ is a three-dimensional
delta-function whose spike is  coincident with the location of the charge. 
Likewise, we can get around our present difficulty by saying that $\nabla\times
{\bf B}$ is a two-dimensional delta-function. A three-dimensional delta-function
is a singular (but integrable) {\em point} in space, whereas a two-dimensional 
delta-function is a singular {\em line} in space. 
It is clear from an examination of Equations~(\ref{e390a})--(\ref{e390c}) that the only component of
$\nabla\times{\bf B}$ which can be singular is the $z$-component, and that this
can only be singular on the $z$-axis ({\em i.e.}, $r=0$).
Thus,  the singularity
coincides with the location of the current, and we can write 
\begin{equation}
\nabla\times{\bf B} = \mu_0 \,I\,\delta(x)\,\delta(y)\,{\bf e}_z.\label{e392}
\end{equation}
The above equation certainly gives $(\nabla\times{\bf B})_x
=(\nabla\times{\bf B})_y = 0$,  and $(\nabla\times{\bf B})_z=0$ everywhere apart
from the $z$-axis, in accordance with Equations~(\ref{e390a})--(\ref{e390c}). Suppose that we integrate over
a plane surface $S$ connected to the loop $C$. The surface element is
$d{\bf S} = dx\,dy\,{\bf e}_z$, so
\begin{equation}
\int_S \nabla\times{\bf B} \cdot d{\bf S} = \mu_0 \,I \int\int
 \delta(x)\,\delta(y)\, dx\,dy
\end{equation}
where the integration is performed over the region $\sqrt{x^2+ y^2} \leq r$. 
However, since the  only part  of $S$ which actually
contributes to the surface integral is the bit which 
lies infinitesimally close to the $z$-axis, we can integrate over all $x$ and $y$
without changing the result. Thus, we obtain
\begin{equation}
\int_S \nabla\times{\bf B} \cdot d{\bf S} = \mu_0\, I \int_{-\infty}^\infty
\delta(x)\, dx\int_{-\infty}^{\infty}
\delta(y)\,dy = \mu_0\, I,\label{e394}
\end{equation}
which is in agreement with Equation~(\ref{e391}). 

But, why  have we gone to so much trouble to prove 
something using vector field theory which can be demonstrated
 in one line via conventional
analysis [see Equation~(\ref{e388})]? The answer, of course, is that the vector field
result is easily generalized, whereas the
conventional result is just  a special case.
For instance, it is clear that Equation~(\ref{e394}) is true for {\em any}\/ surface attached to the loop
C, not just a plane surface.
 Moreover, suppose that we distort our simple circular loop $C$
so that it is no longer circular or even lies in one plane.
 What now is the line integral
of ${\bf B}$ around the loop? This is no longer a simple problem for conventional
analysis, because the magnetic field is not parallel to a line element of the
loop. 
However, according to Stokes' theorem,
\begin{equation}
\oint_C {\bf B} \cdot d{\bf l} = \int_S \nabla\times{\bf B} \cdot d{\bf S},
\end{equation}
with $\nabla\times{\bf B}$ given by Equation~(\ref{e392}). Note that the only part of 
 $S$ which contributes to the surface integral is an infinitesimal region
centered on the $z$-axis. So, as long as 
$S$ actually intersects the $z$-axis,
it does not matter what shape the rest the surface is, and we  always get the same answer
for the surface integral: namely,
\begin{equation}
\oint_C {\bf B}\cdot d{\bf l} = \int_S \nabla\times{\bf B} \cdot d{\bf S}
=\mu_0\, I.
\end{equation}

Thus, provided  the curve $C$ circulates the $z$-axis, and, therefore, any surface
$S$ attached to $C$  intersects the $z$-axis (an odd number of times), the line integral
$\oint_C {\bf B}\cdot d{\bf l}$ is equal to $\mu_0\, I$.
  Of course, if $C$ does not circulate the $z$-axis then
an attached surface $S$ does not intersect the $z$-axis (an odd number of times) and 
$\oint_C {\bf B}\cdot d{\bf l}$ is zero. There is one more proviso. The line
integral $\oint_C {\bf B}\cdot d{\bf l}$ is $\mu_0\, I$ for a loop which circulates the
$z$-axis in a clockwise direction (looking up the
$z$-axis). However, if the loop circulates in an anti-clockwise
direction then the integral is $-\mu_0\, I$. This follows because in the latter case
the $z$-component of the surface element $d{\bf S}$ is oppositely directed to
the current flow at the point where the surface intersects the wire. 

Let us now consider $N$ wires directed parallel to the $z$-axis, with coordinates
($x_i$, $y_i$) in the $x$-$y$ plane, each carrying a current $I_i$ in the
positive $z$-direction. It is fairly obvious that Equation~(\ref{e392}) generalizes
to
\begin{equation}\label{e397}
\nabla\times{\bf B} = \mu_0\sum_{i=1}^N I_i\, \delta(x-x_i)\,\delta(y-y_i)
\,{\bf e}_z.
\end{equation}
If we integrate the magnetic field around some closed curve $C$, which can have any
shape and does not  necessarily lie in one plane, then Stokes' theorem and
the above equation imply that 
\begin{equation}\label{e398}
\oint_C {\bf B}\cdot d{\bf l} = \int_S \nabla\times{\bf B} \cdot d{\bf S}
= \mu_0 \,{\cal I},
\end{equation}
where ${\cal I}$ is the total current enclosed by the curve. Again, if the
curve circulates the $i$th wire  in a clockwise direction (looking down the 
direction of current flow) then the wire contributes $I_i$ 
to the aggregate current ${\cal I}$.
On the other hand, 
if the curve circulates in an anti-clockwise direction then the wire contributes
$-I_i$. Finally, if the curve does not circulate the wire at all then
the wire contributes nothing to ${\cal I}$. 

Equation (\ref{e397}) is a field equation describing how a set of $z$-directed
current carrying wires generate a magnetic field. These wires have
zero-thickness, which implies that we are trying to squeeze a finite amount of
current into an infinitesimal region. This
accounts for the delta-functions on the right-hand side of
the equation. Likewise, we obtained  delta-functions in Section~\ref{s33}
because we were dealing with point charges. Let us now generalize to the more
realistic case of diffuse currents. Suppose that the $z$-current flowing through
 a small
 rectangle in the $x$-$y$ plane, centred on  coordinates ($x$, $y$) and of dimensions
$dx$ and $dy$, is $j_z(x,y)\,dx\,dy$. Here, $j_z$ is termed the {\em current density}\/ in
the $z$-direction. Let us integrate $(\nabla\times{\bf B})_z$ over this rectangle.
The rectangle is assumed to be sufficiently small that $(\nabla\times{\bf B})_z$
does not vary appreciably across it. According to Equation~(\ref{e398}), this integral is
equal to $\mu_0$ times the total $z$-current flowing through the rectangle. Thus,
\begin{equation}
(\nabla\times{\bf B})_z \,dx\,dy = \mu_0\, j_z\,dx\,dy,
\end{equation}
which implies that
\begin{equation}
(\nabla\times{\bf B})_z = \mu_0 \,j_z.\label{e3100}
\end{equation}
Of course, there is nothing special about the $z$-axis.
Hence, we can obtain analogous equations for diffuse currents flowing
in the $y$- and $z$-directions. We can combine all of these equations
to form a single vector field equation
which 
describes how  electric currents generate magnetic fields: {\em i.e.},
\begin{equation}
\nabla\times{\bf B} = \mu_0 \,{\bf j},\label{e3102}
\end{equation}
where ${\bf j} = (j_x, j_y, j_z)$ is the vector current density.
This is the third Maxwell equation. The electric current flowing through a
small area $d{\bf S}$ located at position ${\bf r}$ is ${\bf j}({\bf r})
 \cdot d{\bf S}$.
Suppose that space is filled with  particles of charge $q$,  number density
$n({\bf r})$, and velocity ${\bf v}({\bf r})$. The charge density is
given by $\rho({\bf r})= q\,n$. The current density is given
by ${\bf j}({\bf r})
 = q \,n\, {\bf v}$, and  is obviously a proper vector field (velocities
are proper vectors since they are ultimately derived from displacements). 

If we form the line integral of ${\bf B}$ around some general closed curve
$C$, making use of Stokes' theorem and the field equation (\ref{e3102}), then we
obtain
\begin{equation}\label{e3103}
\oint_C {\bf B} \cdot d{\bf l} = \mu_0 \int_S {\bf j} \cdot d{\bf S}.
\end{equation}
In other words, the line integral of the magnetic field around any closed loop
$C$ 
is equal to $\mu_0$ times the flux of the current density through $C$. This
result is called {\em Amp\`{e}re's circuital law}. If the currents flow in 
zero-thickness wires then Amp\`{e}re's circuital law reduces to Equation~(\ref{e398}).

The flux of the current density
through $C$ is evaluated by integrating ${\bf j}\cdot d{\bf S}$ over any surface
$S$ attached to $C$. Suppose that we take  two different surfaces $S_1$ and
$S_2$. It is clear that if Amp\`{e}re's circuital
law is to make any sense then the surface integral $\int_{S_1} {\bf j}
\cdot d{\bf S} $ had better equal the integral $\int_{S_2} {\bf j}
\cdot d{\bf S} $. That is, when we work out the flux of the current density
though $C$ using two different attached surfaces then we had better get
the same answer, otherwise Equation~(\ref{e3103}) is wrong (since the left-hand side is clearly independent
of the surface spanning C). 
We saw in Chapter~\ref{vector} that if the  integral of a vector field ${\bf A}$
over some surface attached to a loop depends only on the loop, and is 
independent of the surface which spans it, then this implies that
$\nabla\cdot {\bf A} = 0$. Hence,
we require that $\nabla\cdot{\bf j} = 0$ in order to make the
flux of the current density through $C$ a well-defined quantity. We can also see this directly from the field equation (\ref{e3102}). We know that
the divergence of a curl is automatically zero, so taking the divergence 
of Equation~(\ref{e3102}), we obtain
\begin{equation}
\nabla\cdot{\bf j} = 0.
\end{equation}

We have shown that if Amp\`{e}re's circuital law is to make any sense then we need
$\nabla\cdot{\bf j} = 0$. Physically, this implies that the net current flowing
through any closed surface $S$ is zero. 
Up to now, we have only considered
stationary charges and steady currents. It is clear that if all charges are
stationary and all currents are steady then there can be no net current flowing
through a closed surface $S$, since this would imply a build up of charge in the
volume $V$ enclosed by $S$. In other words, as long as we restrict our investigation
to stationary charges, and steady currents, then we expect $\nabla\cdot {\bf j}=0$,
and Amp\`{e}re's circuital law makes sense. However, suppose that we now relax this
restriction. Suppose that some of the charges in a volume $V$ decide to move
outside $V$. Clearly, there will be a non-zero net flux of electric current through
the bounding surface $S$ whilst this is happening. This implies from
Gauss' theorem that $\nabla\cdot{\bf j} \neq 0$. Under these circumstances
Amp\`{e}re's circuital law collapses in  a heap. We shall see later that we can rescue
Amp\`{e}re's  circuital law by adding an extra term involving a time derivative to the
right-hand side of the field equation (\ref{e3102}).
 For steady-state situations ({\em i.e.}, $\partial/\partial t=0$), this
extra term can be neglected. Thus, the field equation 
$\nabla\times{\bf B} = \mu_0 \,{\bf j}$
is, in fact, only two-thirds of Maxwell's third equation: there is a term missing
from the right-hand side. 

We have now derived two field equations involving magnetic fields (strictly speaking,
we have only derived one
and two-thirds equations):
\begin{eqnarray}
\nabla\cdot{\bf B} &=& 0,\label{e3105a}\\[0.5ex]
\nabla\times{\bf B} &=& \mu_0 \,{\bf j}.\label{e3105b}
\end{eqnarray}
We obtained these equations by looking at the fields generated by infinitely
long, straight, steady  currents. This, of course, is a rather special class of
currents. We should now go back and repeat the process for general currents.
In fact, 
if we did this we would find that the above field equations still hold (provided
that the currents are steady). Unfortunately, this
demonstration is rather messy and extremely tedious. There is a better approach.
Let us {\em assume}\/ that the above field equations are valid for any 
set of steady currents. 
We can then, with relatively little effort, 
use these equations to generate the correct formula for the magnetic field
induced by a general set of steady currents, thus proving that our assumption
is correct. More of this later.

As an example of the use of Amp\`{e}re's circuital law, let us calculate the magnetic field
generated by a cylindrical current annulus of inner radius $a$,
and outer radius $b$, co-axial with the $z$-axis, and carrying a uniformly
distributed $z$-directed current $I$. Now, from symmetry, and by analogy
with the magnetic field generated by a straight wire, we expect the current distribution to generate a 
magnetic field of the form ${\bf B}= B_\theta(r)\,{\bf e}_\theta$. 
Let us apply Amp\`{e}re's circuital law to an imaginary circular loop in the $x$-$y$ plane, of radius
$r$, centered on the $z$-axis---see Figure~\ref{famp}. Such a loop is generally known as an {\em Amp\`{e}rian loop}. Now, according to
Amp\`{e}re's circuital law, the line integral of the magnetic field around the loop is equal to
the  current enclosed by the loop, multiplied by $\mu_0$. The line integral is easy to calculate since the magnetic field is everywhere tangential to the
loop. We obtain
$$
2\pi\,r\,B_\theta(r) = \mu_0\,I(r),
$$
where $I(r)$ is the current enclosed by  an Amp\`{e}rian loop of radius $r$.
Now, it is evident that
\begin{equation}
I(r) = \left\{ 
\begin{array}{lcc}
0&\mbox{\hspace{1cm}}&r<a\\[0.5ex]
[(r^2-a^2)/(b^2-a^2)]\, I&&a\leq r\leq b\\[0.5ex]
I&&b<r
\end{array}
\right..
\end{equation}
Hence,
\begin{equation}\label{eamp1}
B_\theta(r) = \left\{ 
\begin{array}{lcc}
0&\mbox{\hspace{1cm}}&r<a\\[0.5ex]
[\mu_0\,I/(2\pi\,r)]\,[(r^2-a^2)/(b^2-a^2)]&&a\leq r\leq b\\[0.5ex]
\mu_0\,I/(2\pi\,r)&&b<r
\end{array}
\right..
\end{equation}
\begin{figure}
\epsfysize=2.in
\centerline{\epsffile{chapter3/fig3.13.eps}}
\caption{\em An example use of Amp\`{e}re's circuital law.}\label{famp}
\end{figure}

\section{Helmholtz's Theorem}\label{s310}
 Up to now, we have 
only studied the 
electric and magnetic fields generated by stationary charges and steady currents.
We have found that these fields are describable in terms of four field  equations: {\em i.e.},
\begin{eqnarray}
\nabla\cdot{\bf E} &=& \frac{\rho}{\epsilon_0},\label{e3106a}\\[0.5ex]
\nabla\times{\bf E} &=& {\bf 0}\label{e3106b}
\end{eqnarray}
for electric fields,
and
\begin{eqnarray}
\nabla\cdot{\bf B} &=&0,\label{e3107a}\\[0.5ex]
\nabla\times{\bf B} &=& \mu_0\, {\bf j}\label{e3107b}
\end{eqnarray}
for magnetic fields. There are no other field equations. 
This strongly suggests that if we know the divergence and the curl of a vector field
then we know everything there is to know about the field. In fact, this is the
case. There is a mathematical theorem which sums this up. It is called
{\em Helmholtz's theorem}, after the German polymath Hermann Ludwig Ferdinand von
Helmholtz.

Let us start with scalar fields. Field equations are a type of differential equation:
{\em i.e.}, they deal with the infinitesimal differences in quantities between neighbouring
points.  So what kind of
differential equation completely specifies a
scalar field? This is easy. Suppose that we have a scalar field $\phi({\bf r})$
and a field equation which tells us the gradient of this field at all points:
something like
\begin{equation}
\nabla\phi = {\bf A},
\end{equation}
where ${\bf A}({\bf r})$ is a vector field. Note that
we need $\nabla\times{\bf A}= {\bf 0}$
for self-consistency, since the curl of a gradient is automatically zero.
The above equation completely
specifies $\phi({\bf r})$ once we are given the value of the field at a single
point, $P$ (say). Thus,
\begin{equation}
\phi(Q) = \phi(P)+\int_P^Q \nabla\phi\cdot d{\bf l} = 
\phi(P)+\int_P^Q {\bf A}\cdot d{\bf l},
\end{equation}
where $Q$ is a general point. The fact that $\nabla\times{\bf A} = {\bf 0}$ means
that ${\bf A}$ is a conservative field, which guarantees that the above
equation  gives a {\em unique}\/ value for $\phi$ at a general point in space.

Suppose that we have a vector field ${\bf F}({\bf r})$. How many differential equations
do we need to completely specify this field? Hopefully, we only need two: one
giving the divergence of the field, and one giving its curl. Let us
test this hypothesis. Suppose that we have two field equations:
\begin{eqnarray}
\nabla\cdot {\bf F} = D,\label{e3110a}\\[0.5ex]
\nabla\times{\bf F} = {\bf C},\label{e3110b}
\end{eqnarray}
where $D({\bf r})$ is a scalar field and ${\bf C}({\bf r})$ is a vector field.
For self-consistency, we need
\begin{equation}
\nabla\cdot {\bf C} = 0,
\end{equation}
since the divergence of a curl is automatically zero. So, do these
two field equations, plus some suitable boundary conditions, completely specify
${\bf F}$? Suppose that we write
\begin{equation}\label{e3112}
{\bf F} = -\nabla U +\nabla\times{\bf W}.
\end{equation}
In other words, we are saying that a general vector field ${\bf F}$ is the sum of
a conservative field, $\nabla U$, and a solenoidal field, $\nabla\times{\bf W}$.
This sounds plausible, but it remains to be proved.
Let us start by taking the divergence of the above equation, and making use of
 Equation~(\ref{e3110a}). We get
\begin{equation}\label{e3113}
\nabla^2 U = -D.
\end{equation}
Note that the vector field ${\bf W}$ does not figure in this equation, because
the divergence of a curl is automatically zero. Let us now take the curl of
Equation~(\ref{e3112}):
\begin{equation}
\nabla\times{\bf F} = \nabla\times\nabla\times{\bf W}= \nabla(\nabla\cdot{\bf W})
- \nabla^2{\bf W} = -\nabla^2 {\bf W}.
\end{equation}
Here, we assume that the divergence of ${\bf W}$ is zero. This is another thing
which remains to be proved. Note that the scalar field $U$ does not figure in
this equation, because the curl of a divergence is automatically zero.
Using Equation~(\ref{e3110b}), we
get
\begin{eqnarray}
\nabla^2 W_x &=& - C_x,\label{e3115a}\\[0.5ex]
\nabla^2 W_y &=& - C_y,\\[0.5ex]
\nabla^2 W_z &=& - C_z,\label{e3115c}
\end{eqnarray}
So, we have transformed our problem into four differential equations, Equation~(\ref{e3113})
and Equations~(\ref{e3115a})--(\ref{e3115c}),  which
we need to solve. Let us look at these equations. We immediately notice that
they all have exactly the same form. In fact, they are all versions of Poisson's
equation. We can now make use of a principle made famous by Richard P.\ Feynman:
``the same equations have the same solutions.'' Recall that earlier on we 
came across the following equation:
\begin{equation}\label{e3116}
\nabla^2\phi = -\frac{\rho}{\epsilon_0},
\end{equation}
where $\phi$ is the electrostatic potential and $\rho$ is the charge density.
We proved that the solution to this equation, with the boundary condition that
$\phi$ goes to zero at infinity, is
\begin{equation}\label{e3117}
\phi({\bf r} ) = \frac{1}{4\pi \epsilon_0} \int\frac{\rho({\bf r'})}
{|{\bf r} - {\bf r'}|}\,d^3{\bf r'}.
\end{equation}
Well, if the same equations have the same solutions, and Equation~(\ref{e3117}) is the solution
to Equation~(\ref{e3116}), then we can immediately write down the solutions to
Equation~(\ref{e3113}) and Equations~(\ref{e3115a})--(\ref{e3115c}). We get
\begin{equation}\label{e3118}
U({\bf r} ) = \frac{1}{4\pi } \int\frac{D({\bf r'})}
{|{\bf r} - {\bf r'}|}\,d^3{\bf r'},
\end{equation}
and
\begin{eqnarray}
W_x({\bf r} ) &=& \frac{1}{4\pi } \int\frac{C_x({\bf r'})}
{|{\bf r} - {\bf r'}|}\,d^3{\bf r'},\\[0.5ex]
W_y({\bf r} ) &=& \frac{1}{4\pi } \int\frac{C_y({\bf r'})}
{|{\bf r} - {\bf r'}|}\,d^3{\bf r'},\\[0.5ex]
W_z({\bf r} ) &=& \frac{1}{4\pi } \int\frac{C_z({\bf r'})}
{|{\bf r} - {\bf r'}|}\,d^3{\bf r'}.
\end{eqnarray}
The last three equations can be combined to form a single vector equation:
\begin{equation}\label{e3120}
{\bf W} ({\bf r})= \frac{1}{4\pi } \int\frac{{\bf C}({\bf r'})}
{|{\bf r} - {\bf r'}|}\,d^3{\bf r'}.
\end{equation}

We assumed earlier that $\nabla\cdot {\bf W} = 0$. Let us check to
see if this is true. Note that
\begin{equation}\label{e3121}
\frac{\partial}{\partial x}\! \left(\frac{1}{|{\bf r} - {\bf r}'|}\right) = 
-\frac{x-x'}{|{\bf r} - {\bf r}'|^3} = \frac{x'-x}{|{\bf r} - {\bf r}'|^3}
= -\frac{\partial}{\partial x'} \!\left(\frac{1}{|{\bf r} - {\bf r}'|}\right),
\end{equation}
which implies that
\begin{equation}\label{e2.145h}
\nabla\left( \frac{1}{|{\bf r} - {\bf r}'|}\right) = - \nabla'
 \left( \frac{1}{|{\bf r} - {\bf r}'|}\right),
\end{equation}
where $\nabla'$ is the operator $(\partial/\partial x', \partial/\partial y',
\partial/\partial z')$. Taking the divergence of Equation~(\ref{e3120}), and making use of
the above relation, we obtain
\begin{equation}
\nabla\cdot{\bf W} = \frac{1}{4\pi} \int {\bf C}({\bf r'})\cdot
\nabla\left( \frac{1}{|{\bf r} - {\bf r}'|}\right)\,d^3{\bf r}'
= -  \frac{1}{4\pi} \int {\bf C}({\bf r'})\cdot
\nabla'\left( \frac{1}{|{\bf r} - {\bf r}'|}\right)\,d^3{\bf r}'.\label{e3123}
\end{equation}
Now
\begin{equation}
\int_{-\infty}^{\infty} g \,\frac{\partial f}{\partial x}\,dx = \left[
g\,f\right]_{-\infty}^{\infty} - \int_{-\infty}^{\infty}
 f\,\frac{\partial g}{\partial x}\,dx.
\end{equation}
However, if $g\,f\rightarrow 0$ as $x\rightarrow\pm\infty$ then we can neglect the
first term on the right-hand side of the above equation and write
\begin{equation}
\int_{-\infty}^{\infty} g\, \frac{\partial f}{\partial x}\,dx = -
\int_{-\infty}^{\infty} f\, \frac{\partial g}{\partial x}\,dx.
\end{equation}
A simple generalization of this result yields
\begin{equation}\label{e2.152x}
\int {\bf g}\cdot\nabla f\,d^3{\bf r} = - \int f\,\nabla\cdot {\bf g}\,d^3{\bf r},
\end{equation}
provided that $g_x\,f \rightarrow 0$ as $|{\bf r}|\rightarrow\infty$, {\em etc.}
Thus, we can
deduce that
\begin{equation}\label{e3127}
\nabla\cdot{\bf W} = \frac{1}{4\pi}\int \frac{\nabla'\!\cdot\!{\bf C}({\bf r}') }
{|{\bf r} - {\bf r}'|} \,d^3{\bf r'}
\end{equation}
from Equation~(\ref{e3123}), provided  $|{\bf C}({\bf r})|$ is bounded as
 $|{\bf r}|\rightarrow\infty$. However, we have already shown that
$\nabla\cdot {\bf C} = 0$ from self-consistency arguments, so the above equation
 implies that $\nabla\cdot{\bf W} = 0$, which is the desired result.

We have  constructed a vector field ${\bf F}({\bf r})$ which satisfies Equations~(\ref{e3110a}) and (\ref{e3110b})
and  behaves sensibly at infinity: {\em i.e.}, $|{\bf F}|\rightarrow 0$ as
$|{\bf r}|\rightarrow \infty$. But, is our solution the only possible solution
of Equations~(\ref{e3110a}) and (\ref{e3110b}) with sensible boundary conditions at infinity? Another way of posing
this question is to ask whether there are any solutions of
\begin{equation}
\nabla^2 U = 0,~~~\nabla^2 W_i =0,\label{e3128}
\end{equation}
where $i$ denotes $x$, $y$, or $z$, which are bounded at infinity. If there are
then we are in trouble, because we can take our solution and add to it an
arbitrary amount of a vector field with zero divergence and zero curl, and thereby
obtain another solution which also satisfies physical boundary conditions. 
This would imply that our solution is not unique. In other words, it is not
possible to unambiguously reconstruct a vector field given its divergence,
its curl, and physical boundary conditions. Fortunately, the equation
\begin{equation}
\nabla^2 \phi = 0,
\end{equation}
which is called {\em Laplace's equation}, has a very interesting property: its solutions are
{\em unique}. That is, if we can find a solution to Laplace's equation
which satisfies the boundary conditions then we are guaranteed that this is
the {\em only}\/ solution. We shall prove this later on in Section~\ref{suniq}. 
Well, let us invent some solutions to Equations~(\ref{e3128}) which are bounded at infinity.
How about
\begin{equation}
U=W_i =0 \,?\label{e3130}
\end{equation}
These solutions certainly satisfy Laplace's equation, and are well-behaved
at infinity. Because the solutions to Laplace's equations are unique, we know
that Equations~(\ref{e3130}) are the only solutions to Equations~(\ref{e3128}). 
This means that there is no vector field which satisfies physical boundary
equations at infinity and has zero divergence and zero curl. In other words, our
solution to Equations~(\ref{e3110a}) and (\ref{e3110b}) is the {\em only} solution. Thus, we have
unambiguously reconstructed the vector field ${\bf F}$ given its divergence, its
curl, and sensible boundary conditions at infinity. This is Helmholtz's theorem.

We have just demonstrated a number of very useful, and also very important, points. 
First, according to Equation~(\ref{e3112}), a general vector field can be written as the
sum of a conservative field and a solenoidal field. Thus, we ought to be able
to write electric and magnetic fields in this form. Second, a general vector
field which is zero at infinity is completely specified once its divergence
and its curl are given. Thus, we can guess that the laws of electromagnetism
can be written as four field equations,
\begin{eqnarray}
\nabla\cdot{\bf E} &=& {\it something},\\[0.5ex]
\nabla\times{\bf E} &=& {\bf something},\\[0.5ex]
\nabla\cdot{\bf B} &=& {\it something},\\[0.5ex]
\nabla\times{\bf B} &=& {\bf something},
\end{eqnarray}
without knowing the first thing about electromagnetism (other than the fact that
it deals with two vector fields). Of course, 
Equations~(\ref{e3106a})--(\ref{e3107b}) are of
exactly this form. We also know that there are only four  field equations, since
the above equations are sufficient to completely reconstruct both ${\bf E}$ and
${\bf B}$. Furthermore, we know that we can solve the field equations without
even knowing what the right-hand sides look like. After all, we solved Equations~(\ref{e3110a})--(\ref{e3110b})
for completely general right-hand sides. [Actually, the right-hand sides have
to go to zero at infinity, otherwise integrals like Equation~(\ref{e3118}) blow up.]
We also know that any solutions we find are unique. In other words, there is only
one possible steady
electric and magnetic field which can be generated by a given set of
stationary charges and steady currents. The third thing which we proved was that
if the right-hand sides of the above field equations are all zero then
the only physical solution is ${\bf E}({\bf r}) = {\bf B}({\bf r}) = {\bf 0}$. This implies
that steady electric and magnetic fields cannot generate themselves.
Instead, they have to be generated by stationary charges and
steady currents. So, if we come across a steady electric field then we know that if
we trace the field-lines back we shall eventually find  a charge. 
Likewise, a steady magnetic field implies that there is a steady
current flowing somewhere. All of these results follow from vector field theory
({\em i.e.}, from the mathematical properties of vector fields in three-dimensional space), prior to
any investigation of electromagnetism.  

\section{Magnetic Vector Potential}\label{svecs}
Electric fields generated by stationary charges obey
\begin{equation}
\nabla\times{\bf E} = {\bf 0}.
\end{equation}
This immediately allows us to write
\begin{equation}
{\bf E} = -\nabla \phi,\label{e3133}
\end{equation}
since the curl of a gradient is automatically zero. In fact, whenever we come
across an irrotational vector field in Physics we can always write it as the
gradient of some scalar field. This is clearly a useful thing to do, since it
enables us to
replace a vector field by a much simpler scalar field.
 The quantity $\phi$ in the above equation
is known as the {\em electric scalar potential}. 

Magnetic fields generated by steady currents (and unsteady currents, for that matter)
satisfy
\begin{equation}
\nabla\cdot {\bf B} = 0.
\end{equation}
This immediately allows us to write 
\begin{equation}
{\bf B} = \nabla\times{\bf A},\label{e3135}
\end{equation}
since the divergence of a curl is automatically zero. In fact, whenever we come
across a solenoidal vector field in Physics we can always write it as the curl
of some other 
vector field. This is not an obviously useful thing to do, however, since
it only allows us to replace one vector field by another. Nevertheless, Equation~(\ref{e3135})
is one of the
most useful equations we shall come across in this book. The quantity
${\bf A}$ is known as the {\em magnetic vector potential}.

We know from Helmholtz's theorem that a vector field is fully specified by
its divergence and its curl. The curl of the vector potential gives us the magnetic
field via Equation~(\ref{e3135}). However, the divergence of ${\bf A}$ has no physical
significance. In fact, we are completely free to choose $\nabla\cdot{\bf A}$ to
be whatever we like. Note that, according to Equation~(\ref{e3135}), the magnetic field
is invariant under the transformation 
\begin{equation}
{\bf A} \rightarrow {\bf A} - \nabla\psi.\label{e3136}
\end{equation}
In other words, the vector potential is undetermined to the gradient of a scalar 
field. This is just another way of saying that we are free to choose $\nabla\cdot
{\bf A}$. Recall that the electric scalar potential is undetermined to an
arbitrary additive constant, since the transformation
\begin{equation}
\phi \rightarrow \phi + c\label{e3137}
\end{equation}
leaves the electric field invariant in Equation~(\ref{e3133}). The transformations (\ref{e3136})
and (\ref{e3137}) are examples of what mathematicians call  {\em gauge transformations}.
The choice of a particular function $\psi$ or a particular constant $c$ is
referred to as a choice of the 
gauge. We are free to fix the gauge to be whatever we
like. The most sensible choice is the one which makes our equations as simple
as possible. The usual gauge for the scalar potential $\phi$ is such
that $\phi\rightarrow 0$ at infinity. The usual gauge for ${\bf A}$
(in steady-state situations) is such that
\begin{equation}\label{e3138}
\nabla\cdot{\bf A} = 0.
\end{equation}
This particular choice is known as the {\em Coulomb gauge}.

It is obvious that we can always add a constant to $\phi$ so as to make
it zero at infinity. But it is not at all obvious that we can always
perform a gauge transformation such as to make $\nabla\cdot{\bf A}$ zero.
Suppose that we have found some vector field ${\bf A}({\bf r})$ whose curl gives the
magnetic field but whose divergence in non-zero. Let
\begin{equation}
\nabla \cdot{\bf A} = v({\bf r}).
\end{equation}
So, can we find a scalar field $\psi$ such that after we perform the
gauge transformation (\ref{e3136}) we are left with $\nabla\cdot{\bf A} = 0$? Taking
the divergence of Equation~(\ref{e3136}) it is clear that we need to find a
function  $\psi$ which
satisfies
\begin{equation}
\nabla^2\psi = v.
\end{equation}
But this is just Poisson's equation. We know that we can always find a
unique solution of this equation (see Section~\ref{s310}). This proves that, in practice,
we can always set the divergence of ${\bf A}$ equal to zero. 

Let us again consider  an infinite straight wire directed along the $z$-axis and
carrying a current $I$. The magnetic field generated by such a wire is
written
\begin{equation}
{\bf B} = \frac{\mu_0 \,I}{2\pi} \left(\frac{-y}{r^2}, \frac{x}{r^2}, 0\right).
\end{equation}
We wish to find a vector potential ${\bf A}$ 
 whose curl is equal to  the above magnetic field,  and whose divergence is
zero. It is not difficult to see that 
\begin{equation}
{\bf A} = -\frac{\mu_0 \,I}{4\pi} \left(\,0,\,0,\,\ln[x^2+y^2]\,\right)
\end{equation}
fits the bill.
Note that the vector potential is parallel to the direction of the current. This
would seem to suggest that there is a more direct relationship between 
the vector potential and the current than there is between the magnetic field
and the current. The potential is not very well-behaved on the $z$-axis, but this
is just because we are dealing with an infinitely thin current. 

Let us take the curl of Equation~(\ref{e3135}). We find that
\begin{equation}
\nabla\times{\bf B} = \nabla\times\nabla\times {\bf A} = \nabla(\nabla\cdot{\bf A})
-\nabla^2{\bf A} = - \nabla^2{\bf A},
\end{equation}
where use has been made of the Coulomb gauge condition 
(\ref{e3138}). We can combine the above
relation with the field equation (\ref{e3102}) to give
\begin{equation}
\nabla^2{\bf A} = - \mu_0 \,{\bf j}.
\end{equation}
Writing this in component form, we obtain
\begin{eqnarray}
\nabla^2 A_x &=& - \mu_0\, j_x,\\[0.5ex]
\nabla^2 A_y &=& - \mu_0\, j_y,\\[0.5ex]
\nabla^2 A_z &=& - \mu_0\, j_z.
\end{eqnarray}
But, this is just Poisson's equation three times over. We can immediately
write the unique solutions to the above equations:
\begin{eqnarray}
A_x({\bf r}) = \frac{\mu_0}{4\pi} \int \frac{j_x({\bf r}')}{|{\bf r} - {\bf r}'|}
\,d^3{\bf r}',\\[0.5ex]
A_y({\bf r}) = \frac{\mu_0}{4\pi} \int \frac{j_y({\bf r}')}{|{\bf r} - {\bf r}'|}
\,d^3{\bf r}',\\[0.5ex]
A_z({\bf r}) = \frac{\mu_0}{4\pi} \int \frac{j_z({\bf r}')}{|{\bf r} - {\bf r}'|}
\,d^3{\bf r}'.
\end{eqnarray}
These solutions can be recombined to form a single vector solution
\begin{equation}
{\bf A}({\bf r}) = \frac{\mu_0}{4\pi} 
\int \frac{{\bf j}({\bf r}')}{|{\bf r} - {\bf r}'|}
\,d^3{\bf r}'.\label{e3147}
\end{equation}
Of course, we have seen a equation like this before:
\begin{equation}\label{e3148}
\phi({\bf r}) = \frac{1}{4\pi\epsilon_0} \int
\frac{\rho({\bf r}')}{|{\bf r} - {\bf r}'|}\,d^3{\bf r}'.
\end{equation}
Equations (\ref{e3147}) and (\ref{e3148}) are the unique solutions (given the arbitrary choice
of gauge) to the field equations (\ref{e3106a})--(\ref{e3107b}): they specify the magnetic
vector and electric scalar potentials generated by a set of stationary
charges, of charge density $\rho({\bf r})$, and a set of steady currents,
of current density ${\bf j}({\bf r})$. Incidentally, we can prove that
Equation~(\ref{e3147}) satisfies the gauge condition $\nabla\cdot{\bf A}=0$ by repeating
the analysis of Equations~(\ref{e3120})--(\ref{e3127}) (with ${\bf W}\rightarrow{\bf A}$
and ${\bf C} \rightarrow \mu_0\, {\bf j}$), and using the fact that
$\nabla\cdot{\bf j} = 0$ for steady currents. 

As an example, let us find the vector potential associated with the magnetic field
distribution (\ref{eamp1}). By symmetry, and by analogy with the
vector potential generated by a straight wire, we expect that
${\bf A} = A_z(r)\,{\bf e}_z$. Note that this form for ${\bf A}$
satisfies the Coulomb gauge.
Hence, using $\nabla\times {\bf A} = {\bf B}$,  we get
\begin{equation}
\frac{\partial A_z}{\partial r} =- B_\theta(r).
\end{equation}
The boundary conditions are that $A_z$ be continuous at $r=a$ and $r=b$,
since a discontinuous $A_z$ would generate an infinite magnetic field,
which is unphysical. Thus, we obtain
\begin{equation}
A_z(r) = \left\{ 
\begin{array}{lcc}
(\mu_0\,I/2\pi)\,[1/2- a^2\,\ln(b/a)/(b^2-a^2)]&\mbox{\hspace{0.25cm}}&r<a\\[0.5ex]
(\mu_0\,I/2\pi)\,([b^2/2-r^2/2-a^2\,\ln(b/r)]/[b^2-a^2])&&a\leq r\leq b\\[0.5ex]
-(\mu_0\,I/2\pi)\,\ln(r/b)&&b<r
\end{array}
\right..
\end{equation}

\section{Biot-Savart Law}
According to Equation~(\ref{e3133}), we can obtain an expression for the electric field 
generated by stationary charges by
taking minus the gradient of Equation~(\ref{e3148}). This yields
\begin{equation}
{\bf E} ({\bf r}) = \frac{1}{4\pi\epsilon_0} \int\rho({\bf r}')\,
\frac{{\bf r} - {\bf r}'}{|{\bf r} - {\bf r}'|^3}\,d^3{\bf r}',
\end{equation}
which we recognize as Coulomb's law written for a continuous charge distribution. 
According to Equation~(\ref{e3135}), we can obtain an  equivalent expression for the magnetic 
field generated by steady currents by taking the curl of Equation~(\ref{e3147}). This gives
\begin{equation}
{\bf B} ({\bf r}) = \frac{\mu_0}{4\pi} \int
\frac{{\bf j}({\bf r}')\times
({\bf r} - {\bf r}')}{|{\bf r} - {\bf r}'|^3}\,d^3{\bf r}',\label{e3150}
\end{equation}
where use has been made of the vector identity $\nabla\times(\phi\,{\bf A})= 
\phi\,\nabla\times{\bf A} + \nabla\phi\times{\bf A}$. Equation (\ref{e3150}) is
known as the {\em Biot-Savart law}\/ after the French physicists Jean Baptiste
Biot and Felix Savart: it completely specifies the magnetic field generated 
by a steady (but otherwise quite general) distributed  current.

Let us reduce  our  distributed  current  to 
an idealized zero thickness wire. We can do this by writing
\begin{equation}
{\bf j}({\bf r})\,d^3{\bf r} = {\bf I}({\bf r}) \,dl,\label{e3151}
\end{equation}
where ${\bf I}$ is the vector current ({\em i.e.}, its direction and magnitude specify
the direction and magnitude of the current) and
$dl$ is an element of length along the wire. Equations~(\ref{e3150}) and (\ref{e3151}) can
be combined to give 
\begin{equation}
{\bf B} ({\bf r}) = \frac{\mu_0}{4\pi} \int
\frac{{\bf I}({\bf r}')\times
({\bf r} - {\bf r}')}{|{\bf r} - {\bf r}'|^3}\,dl,\label{e3.152}
\end{equation}
which is the form in which the Biot-Savart law is most usually written. 
This law is to magnetostatics ({\em i.e.}, the study of magnetic
fields generated by steady currents) what Coulomb's law is to electrostatics
({\em i.e.}, the study of electric fields generated by stationary charges). Furthermore,
it can be experimentally verified given a set of currents, a
compass, a test wire, and a great deal of skill and patience. This
justifies  our
earlier assumption that the field equations (\ref{e3105a}) and (\ref{e3105b}) are valid for general
 current distributions (recall that we derived them by studying the  fields
generated by infinite straight wires). Note that both Coulomb's law and
the Biot-Savart law are {\em gauge independent}: {\em i.e.}, they do not depend on the 
particular choice of  gauge.

Consider  an infinite straight wire, directed along the
$z$-axis, and carrying a current $I$---see Figure~\ref{f32}.
Let us reconstruct the magnetic field generated by the wire at point
$P$ using the Biot-Savart
law. Suppose that the perpendicular distance to the wire is $\rho$. It is
easily seen that
\begin{eqnarray}
{\bf I} \times({\bf r} - {\bf r}') &= &I\,\rho\,\,{\bf e}_\theta,\\[0.5ex]
l &=& \rho\,\tan\phi,\\[0.5ex]
dl &=& \frac{\rho}{\cos^2\phi} \,d\phi,\\[0.5ex]
|{\bf r} - {\bf r}'| &=& \frac{\rho}{\cos\phi},
\end{eqnarray}
where $\theta$ is a cylindrical polar coordinate.
Thus, according to Equation~(\ref{e3.152}), we have
\begin{eqnarray}
B_\theta& = &\frac{\mu_0}{4\pi}  \int_{-\pi/2}^{\pi/2} \frac{I\,\rho}
{\rho^3 \,(\cos\phi)^{-3}}\frac{\rho}{\cos^2\phi} \,d\phi\nonumber\\[0.5ex]
&=&\frac{\mu_0 \,I}{4\pi \,\rho} \int_{-\pi/2}^{\pi/2} \cos\phi\,d\phi
= \frac{\mu_0 \,I}{4\pi \,\rho} \left[ \sin\phi\right]_{-\pi/2}^{\pi/2},
\end{eqnarray}
which gives the familiar result
\begin{equation}
B_\theta = \frac{\mu_0 \,I}{2\pi\, \rho}.\label{e3155}
\end{equation}
So, we have come full circle in our investigation of magnetic fields.
Note that  the simple result (\ref{e3155}) can only be obtained from the Biot-Savart law
after some non-trivial algebra. 
Examination  of
more complicated current distributions using this law invariably
leads to  lengthy, involved,  and extremely unpleasant
calculations. 
\begin{figure}
\epsfysize=2.5in
\centerline{\epsffile{chapter3/fig3.14.eps}}
\caption{\em A Biot-Savart law calculation.}\label{f32}
\end{figure}

\section{Electrostatics and Magnetostatics}
We have now completed our theoretical
investigation of electrostatics and magnetostatics. Our next task is to incorporate
time variation into our analysis.
However, before we start this, let us briefly review
our progress so far. We have found that the electric fields generated by stationary
charges, and the magnetic fields generated by steady currents, are describable
in terms of four field equations:
\begin{eqnarray}
\nabla\cdot {\bf E} &=& \frac{\rho}{\epsilon_0},\label{e3156a}\\[0.5ex]
\nabla\times{\bf E} &=& {\bf 0},\label{e3156b}\\[0.5ex]
\nabla\cdot{\bf B} &=& 0,\label{e3156c}\\[0.5ex]
\nabla\times{\bf B} &=& \mu_0 \,{\bf j}.\label{e3156d}
\end{eqnarray}
The boundary conditions are that the fields are zero at infinity, assuming that
the generating charges and currents are localized 
to some region in space. According to Helmholtz's theorem, the above field equations,
plus the boundary conditions, are sufficient to {\em uniquely}\/ specify the electric
and magnetic fields. The physical significance of this is that divergence
and curl are the only {\em rotationally invariant}\/ first-order differential properties
of a general vector field: {\em i.e.}, the only quantities which do not change their physical characteristics  when the
coordinate axes are rotated. Since Physics does not depend on the orientation of the coordinate axes
(which is, after all, quite arbitrary), it follows that divergence and curl are the {\em only}\/
quantities which can appear in  first-order differential field equations which claim to describe physical
phenomena.

The field equations can be integrated to give:
\begin{eqnarray}
\oint_S {\bf E} \cdot d{\bf S} &=& \frac{1}{\epsilon_0}\int_V \rho\,dV,\label{e3157a}\\[0.5ex]
\oint_C {\bf E} \cdot d{\bf l} &=& 0,\label{e3157b}\\[0.5ex]
\oint_S {\bf B} \cdot d{\bf S} &=& 0,\label{e3157c}\\[0.5ex]
\oint_C {\bf B} \cdot d{\bf l} &=& \mu_0 \int_{S'} {\bf j} \cdot d{\bf S}.\label{e3157d}
\end{eqnarray}
Here, $S$ is a closed surface enclosing a volume $V$. Also, $C$ is a closed loop,
and $S'$ is some surface attached to this loop. The field equations 
(\ref{e3156a})--(\ref{e3156d}) can be deduced
from Equations~(\ref{e3157a})--(\ref{e3157d}) using Gauss' theorem and Stokes' theorem. Equation
(\ref{e3157a}) is called Gauss' law, and says that the flux of the electric field
out of a closed surface is proportional to the enclosed electric charge. 
Equation~(\ref{e3157c}) has no particular name, and says that there is no such
things as a magnetic monopole. Equation~(\ref{e3157d}) is called Amp\`{e}re's circuital  law,
and says that the line integral of the magnetic
field around any closed loop is proportional
to the flux of the current density through the loop. 

The field equation (\ref{e3156b}) is automatically satisfied if we write
\begin{equation}
{\bf E} = -\nabla \phi.\label{e3158}
\end{equation}
Likewise, the field equation (\ref{e3156c}) is automatically satisfied if we write
\begin{equation}
{\bf B} = \nabla\times{\bf A}.\label{e3159}
\end{equation}
Here, $\phi$ is the electric scalar potential, and ${\bf A}$ is the magnetic vector
potential. The electric field is clearly unchanged if we add a constant to the
scalar potential:
\begin{equation}
{\bf E} \rightarrow {\bf E}~~~{\rm as}~~~\phi\rightarrow \phi + c.
\end{equation}
The magnetic field is similarly unchanged if we subtract  the gradient of a scalar field from
the vector potential:
\begin{equation}
{\bf B} \rightarrow {\bf B}~~~{\rm as}~~~{\bf A} \rightarrow {\bf A} - \nabla\psi.
\end{equation}
The above transformations, which leave the ${\bf E}$ and ${\bf B}$ fields
invariant, are called gauge transformations. We are free to choose $c$ and $\psi$
to be whatever we like: {\em i.e.}, we are free to choose the gauge.
The most sensible gauge is the one  which make our
equations as simple and symmetric as possible. This corresponds to the choice
\begin{equation}
\phi({\bf r}) \rightarrow 0~~~{\rm as}~~~|{\bf r}| \rightarrow \infty,
\end{equation}
and 
\begin{equation}
\nabla\cdot {\bf A} = 0.
\end{equation}
The latter convention is known as the Coulomb gauge.

 Taking the 
divergence of Equation~(\ref{e3158}) and the curl of Equation~(\ref{e3159}), and making use of the 
Coulomb gauge, we find that the four field equations (\ref{e3156a})--(\ref{e3156d}) can be reduced to
Poisson's equation written four times over:
\begin{eqnarray}
\nabla^2 \phi &=& - \frac{\rho}{\epsilon_0},\label{e3164a}\\[0.5ex]
\nabla^2{\bf A} &=& - \mu_0 \,{\bf j}.\label{e3164b}
\end{eqnarray}
Poisson's equation is just about the simplest {\em rotationally invariant} second-order
partial differential equation it is possible to write. Note that
$\nabla^2$ is clearly rotationally invariant, since it is the divergence of
a gradient, and both divergence and gradient are rotationally invariant.
We can always construct the solution to Poisson's equation, given the
boundary conditions. Furthermore, we have a {\em uniqueness theorem} which tells us
that our solution is the only possible solution. Physically, this means that
 there is only one electric and magnetic
field which is consistent with a given set of stationary charges and steady currents.
This sounds like an obvious, almost trivial, statement. But there are many
areas of Physics (for instance, Fluid Mechanics and Plasma Physics) where
we also believe, for physical reasons, that for a given set of boundary conditions
the solution should be unique. The difficulty is that  in most cases 
when we reduce a given problem to a partial differential equation we end up with
something far nastier than Poisson's equation. In general, we cannot solve
this equation. In fact, we usually cannot even prove that  it
possess a solution for general boundary conditions, let alone that the solution
is unique. So, we are very fortunate indeed that
in Electrostatics and Magnetostatics a general problem always boils down to solving a
tractable  partial differential equation. When physicists make statements to the effect that
``Electromagnetism is the best understood theory in Physics,'' which they often do, what they are really
saying is that the partial differential equations which crop up in this
theory are soluble and have unique solutions.

Poisson's equation
\begin{equation}
\nabla^2 u = v\label{e3165}
\end{equation}
is {\em linear}, which means that its solutions are superposable. We can exploit
this fact to construct a general solution to this equation. Suppose that we can
find the solution to
\begin{equation}
\nabla^2 G({\bf r}, {\bf r}') = \delta({\bf r} - {\bf r}')
\end{equation}
which satisfies the boundary conditions. 
This is the solution driven by  a unit amplitude point source located at position
vector ${\bf r}'$. Since any general source can be built up out of a weighted sum
of point sources, it follows that a general solution to Poisson's equation 
can be built up out of a similarly weighted superposition of point source solutions. 
Mathematically, we can write
\begin{equation}\label{e3167}
u({\bf r}) = \int G({\bf r}, {\bf r}')\, v({\bf r}')\,d^3{\bf r}'.
\end{equation}
The function $G({\bf r}, {\bf r}')$ is called the Green's function. The Green's function
for Poisson's equation is
\begin{equation}\label{e3168}
G({\bf r}, {\bf r}') = - \frac{1}{4\pi} \frac{1}{|{\bf r} - {\bf r}'|}.
\end{equation}
Note that this  Green's function is proportional to the scalar potential of
a point charge located at ${\bf r}'$: this is hardly surprising, given the
definition of a Green's function. 

According to Equations~(\ref{e3164a}), (\ref{e3164b}), (\ref{e3165}), (\ref{e3167}), and (\ref{e3168}), the 
scalar and vector potentials generated by
a set of stationary charges and steady currents take the form
\begin{eqnarray}
\phi({\bf r}) &=& \frac{1}{4\pi\epsilon_0} \int \frac{\rho({\bf r}')}
{|{\bf r} - {\bf r}'|} \,d^3{\bf r}',\label{e3169a}\\[0.5ex]
{\bf A}({\bf r}) &=& \frac{\mu_0}{4\pi} \int \frac{{\bf j}({\bf r}')}
{|{\bf r} - {\bf r}'|} \,d^3{\bf r}'.\label{e3169b}
\end{eqnarray}
Making use of Equations~(\ref{e3158}), (\ref{e3159}), (\ref{e3169a}), and (\ref{e3169b}), we obtain Coulomb's law,
\begin{equation}\label{e2.216x}
{\bf E}({\bf r}) = \frac{1}{4\pi\epsilon_0} \int \rho({\bf r}') \,\frac{{\bf r} 
-{\bf r}'}{|{\bf r} - {\bf r}'|^3}\,d^3{\bf r}',
\end{equation}
and the Biot-Savart law,
\begin{equation}
{\bf B} ({\bf r}) = \frac{\mu_0}{4\pi} \int \frac{{\bf j}({\bf r}')\times
({\bf r} - {\bf r}')}
{|{\bf r} - {\bf r}'|^3}\,d^3{\bf r}'.\label{e3171}
\end{equation}
Of course, both of these laws are examples of action at a distance laws (in that the electric and magnetic fields at a given point respond instantaneously to
changes in the charge and current densities at distant points), and,
therefore, violate the Special Theory of Relativity. However,  this is not a problem as long as we
restrict ourselves to fields generated by
{\em time-independent}\/ charge and current distributions. 

The next question  is by how much is this scheme which we have just worked out going to
be disrupted when we take time variation into account. The answer, somewhat
surprisingly, is by very little indeed. So, in Equations~(\ref{e3156a})--(\ref{e3171}) we can already
discern the basic outline of Classical Electromagnetism. Let us continue our
investigation.

{\small
\section{Exercises}
\renewcommand{\theenumi}{3.\arabic{enumi}}
\begin{enumerate}
\item A charge $Q$ is uniformly distributed in a sphere of radius $a$ centered
on the origin. Use symmetry and Gauss' law to find the electric field generated inside and
outside the sphere. What is the corresponding electric potential inside and
outside the sphere?
\item A charge per unit length $Q$ is uniformly distributed in an infinitely long cylinder
of radius $a$ whose axis corresponds to the $z$-axis. Use symmetry and  Gauss' law to
find the electric field generated inside and outside the cylinder. What is the
corresponding electric potential inside and outside the cylinder.
\item Find the electric charge distribution which generates the Yukawa
potential
$$
\phi(r) = \frac{q}{4\pi\epsilon_0}\,\frac{{\rm e}^{-r/a}}{r},
$$
where $r$ is a spherical polar coordinate, and $a$ a positive constant.
Why must the total charge in the distribution be zero?
\item An electric dipole consists of two equal and opposite charges, $q$ and $-q$, separated by a {\em small}\/ distance $d$. The strength and orientation of the dipole
is measured by its vector moment ${\bf p}$, which
is of magnitude $q\,d$, and points in the direction  of the displacement of the positive charge from the negative. Use the principle of superposition
to demonstrate that the electric potential generated by an electric dipole
of moment ${\bf p}$ situated at the origin is
$$
\phi({\bf r}) = \frac{{\bf p}\cdot{\bf r}}{4\pi\epsilon_0\,r^3}.
$$
Show that the corresponding electric field distribution is
$$
{\bf E}({\bf r}) = \frac{3\,({\bf p}\cdot{\bf r})\,{\bf r} - r^2\,{\bf p}}{4\pi\epsilon_0\,r^5}.
$$
\item A electric dipole of fixed moment ${\bf p}$ is situated at position
 ${\bf r}$ in a non-uniform external electric field ${\bf E}({\bf r})$. Demonstrate
that the net force on the dipole can be written ${\bf f} = - \nabla W$, where
$$
W =- {\bf p}\cdot{\bf E}.
$$
Hence, show that the potential energy of an electric dipole of moment ${\bf p}_1$
in the electric field generated by a second dipole of moment ${\bf p}_2$ is
$$
W = \frac{r^2\,({\bf p}_1\cdot{\bf p}_2) - 3\,({\bf p}_1\cdot{\bf r})\,({\bf p}_2\cdot{\bf r})}{4\pi\epsilon_0\,r^5},
$$
where ${\bf r}$ is the displacement of one dipole from another.
\item Show that the torque on an electric dipole of moment ${\bf p}$ in a uniform
external electric field ${\bf E}$ is
$$
\btau = {\bf p}\times {\bf E}.
$$
Hence, deduce that the potential energy of the dipole is
$$
W = -{\bf p}\cdot {\bf E}.
$$
\item A charge distribution $\rho({\bf r})$ is localized in the vicinity of the origin in a region of radius
$a$. Consider the electric potential generated at position ${\bf r}$, where
$|{\bf r}|\gg a$. Demonstrate via a suitable expansion in $a/r$ that
$$
\phi({\bf r}) = \frac{Q}{4\pi\epsilon_0\,r}+ \frac{{\bf p}\cdot{\bf r}}{4\pi\epsilon_0\,r^3} + \cdots,
$$
where $Q=\int \rho({\bf r})\,d^3{\bf r}$ is the total charge contained in the
distribution, and ${\bf p} = \int \rho({\bf r})\,{\bf r}\,d^3{\bf r}$ its electric
dipole moment.
\item Consider a scalar potential field $\phi({\bf r})$ generated by a set of stationary
charges. Demonstrate that the mean potential over any spherical surface which does
not contain a charge is equal to the potential at the center.
Hence, deduce that there can be no maxima or minima of the scalar potential
in a charge free region. Hint: The solution to this problem is more intuitive
than mathematical, and depends on the fact that the potential generated outside a uniform spherical charge shell is the same as that generated when
all of the charge is collected at its center.
\item Demonstrate that the Green's function for Poisson's equation in two
dimensions ({\em i.e.}, $\partial/\partial z\equiv 0$) is
$$
G({\bf r}, {\bf r}') = - \frac{\ln|{\bf r}-{\bf r}'|}{2\pi},
$$
where ${\bf r}= (x,\,y)$, {\em etc.}
Hence, deduce that the scalar potential field generated by the two-dimensional
charge distribution $\rho({\bf r})$ is
$$
\phi({\bf r}) = - \frac{1}{2\pi\epsilon_0}\int\rho({\bf r}')\,\ln|{\bf r}-{\bf r'}|\,d^3{\bf r}'.
$$
\item A particle of mass $m$ and charge  $q$ starts at rest from the origin at $t=0$ in a uniform electric
field $E$ directed along the $y$-axis, and a uniform magnetic field $B$ directed
along the $z$-axis. Find the particle's subsequent motion in the $x$-$y$ plane. Sketch the particle's trajectory.
\item In a parallel-plate magnetron the cathode and the anode are
flat parallel plates, and a uniform magnetic field $B$ is applied in a direction
parallel to the plates. Electrons are emitted from the cathode with essentially
zero velocity. If the separation between the anode and cathode is $d$,
and if the anode is held at a constant positive potential $V$ with respect to the cathode, show that no current will flow between the plates when
$$
V \leq \frac{e\,B^2\,d^2}{2\,m_e}.
$$
Here, $e$ is the magnitude of the electron charge, and $m_e$ the
electron mass.

\item An infinite, straight, circular cross-section wire of radius $a$ runs along the 
$z$-axis and carries a uniformly distributed $z$-directed current $I$. Use symmetry and Amp\`{e}re's circuital law to find the magnetic field distribution inside and outside the
wire. What is the corresponding magnetic vector potential inside and
outside the wire? Use the Coulomb gauge.
\item An infinite cylindrical current annulus of inner radius $a$, outer
radius $b$, and axis running along the $z$-axis carries a uniformly
distributed current per unit length $I$ in the $\theta$ direction: {\em i.e.},
${\bf j}\propto {\bf e}_\theta$. Use symmetry and Amp\`{e}re's circuital law to find the magnetic field distribution inside and outside the
annulus. What is the corresponding magnetic vector potential inside and
outside the annulus? Use the Coulomb gauge.
\item A thick slab extends from $z=-a$ to $z=a$, and is infinite in the $x$-$y$
plane. The slab carries a uniform current density ${\bf j} = J\,{\bf e}_x$.
Find the magnetic field and magnetic vector potential inside and outside the
slab. Use the Coulomb gauge.
\item Show that the magnetic vector potential due to two long, straight, $z$-directed wires, the first carrying a current $I$, and the second a current $-I$, 
is 
$$
{\bf A} = \frac{\mu_0\,I}{2\pi}\,\ln\left(\frac{r_1}{r_2}\right){\bf e}_z,
$$
where $r_1$ and $r_2$ are the perpendicular distances to the two wires.
\item Use the Biot-Savart law to:
\begin{enumerate}
\item Find the magnetic field at the center of a circular loop of radius $r$
carrying a current $I$.
\item Find the magnetic field at the center of a square loop carrying a current
$I$. Let $r$ be the perpendicular distance from the center to one of the
sides of the loop.
\item Find the magnetic field at the center of a regular $n$-sided polygon carrying a current $I$. Let $r$ be the perpendicular distance from the center
to one of the sides of the loop. Check that your answer reduces to the answer
from part (a) in the limit $n\rightarrow\infty$.
\end{enumerate}
\item Use the Biot-Savart law to show that the magnetic field
generated along the axis of a circular current loop of radius $a$ lying in
the $x$-$y$ plane and centered on the origin is
$$
B_z = \frac{\mu_0\,I}{2}\,\frac{a^2}{(a^2+z^2)^{3/2}}.
$$
Here, $I$ is the current circulating counter-clockwise (looking down the
$z$-axis) around the loop. Demonstrate that
$$
\int_{-\infty}^{\infty} B_z(z)\,dz = \mu_0\,I.
$$
Derive this result from Amp\`{e}re's circuital law.
\item A Helmholtz coil consists of two identical, single turn, circular coils, of radius $a$,  carrying
the same current, $I$, in the same sense, which are coaxial with one another,
and are separated by a distance $d$. Show that the variation of the magnetic field-strength in the vicinity
of the axial midpoint is minimized when $d=a$. Demonstrate that, in this optimal case, the
magnetic field-strength at the axial midpoint is
$$
B = \frac{8\,\mu_0\,I}{5\sqrt{5}\,a}.
$$

\item A force-free magnetic field is such that ${\bf j}\times {\bf B} = {\bf 0}$. Demonstrate that such a field satisfies
$$
\nabla^2 {\bf B} = -\alpha^2\,{\bf B},
$$
where $\alpha$ is some constant. Find the force-free field with the lowest value of $\alpha$ (excluding $\alpha=0$) in a cubic volume of dimension $a$ bounded by superconducting walls (in which ${\bf B}={\bf 0}$).
\item Consider a small circular current loop of radius $a$ lying in the
$x$-$y$ plane and centered on the origin. Such a loop constitutes a magnetic
dipole  of moment ${\bf m}$, where $m=\pi\,a^2\,I$, and $I$ is the circulating current. The direction of ${\bf m}$ is conventionally taken to be
normal to the plane of the loop, in the sense given by the right-hand
grip rule. Demonstrate that the magnetic vector potential generated
at position ${\bf r}$, where $|{\bf r}|\gg a$, is
$$
{\bf A}({\bf r}) = \frac{\mu_0}{4\pi}\,\frac{{\bf m}\times {\bf r}}{r^3}.
$$
Show that the corresponding magnetic field is
$$
{\bf B}({\bf r}) = \frac{\mu_0}{4\pi}\left(\frac{3\,({\bf r}\cdot{\bf m})\,{\bf r}- r^2\,{\bf m}}{r^5}\right).
$$

\item Demonstrate that the torque on a magnetic dipole  of moment ${\bf m}$ placed in a uniform
external magnetic field ${\bf B}$ is
$$
\btau = {\bf m}\times {\bf B}.
$$
Hence, deduce that the potential energy of the magnetic dipole is
$$
W = - {\bf m}\cdot{\bf B}.
$$
\item Consider two magnetic dipoles, ${\bf m}_1$ and ${\bf m}_2$. Suppose
that ${\bf m}_1$ is fixed, whereas ${\bf m}_2$ can rotate freely in any direction. Demonstrate that the equilibrium configuration of the second dipole is
such that
$$
\tan\theta_1 = - 2\,\tan\theta_2,
$$
where $\theta_1$ and $\theta_2$ are the angles subtended by ${\bf m}_1$
and ${\bf m}_2$, respectively, with the radius vector joining them.
\end{enumerate}
\renewcommand{\theenumi}{2.\arabic{enumi}}
}
