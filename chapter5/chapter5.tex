\chapter{Electrostatic Calculations}\label{estat}

\section{Introduction}
In this chapter, we shall  make a detailed investigation of the
electric fields generated by stationary charge distributions using Maxwell's equations. In particular, we shall examine the interaction of electrostatic
fields with ohmic conductors.

\section{Electrostatic Energy}
Consider a collection of $N$ static point charges $q_i$  located at position vectors
${\bf r}_i$, respectively (where $i$ runs from 1 to $N$). 
What is the electrostatic energy stored in such a collection? In other words, how much work would we have to do in order to assemble
the charges, starting from an initial state in which they are all 
at rest and very widely
separated?

 Well, we know that a static electric field is conservative, and can  consequently
be written in terms of
a scalar potential:
\begin{equation}
{\bf E} = - \nabla\phi.
\end{equation}
We also know that the electric force on a charge $q$  located at position ${\bf r}$ is
written
\begin{equation}
{\bf f} = q\, {\bf E}({\bf r}).
\end{equation}
The work {\em we} would have to do against electrical forces in order to {\em slowly}\/
move the charge from point $P$ to point $Q$ is simply
\begin{equation}\label{e5.3}
W =-  \int_P^Q {\bf f} \cdot d{\bf l} =- q \int_P^Q {\bf E} \cdot d{\bf l} 
=q\int_P^Q \nabla\phi \cdot d{\bf l} = q \left[ \phi(Q) - \phi(P)\right].
\end{equation}
The negative sign in the above expression comes about because we would have to 
exert a force $-{\bf f}$ on the charge, in order to counteract the force
exerted by the electric field. Recall, finally, that the scalar potential field
generated by a point charge $q$ located at position ${\bf r}'$ is
\begin{equation}\label{e5.4}
\phi({\bf r})= \frac{1}{4\pi\epsilon_0} \frac{q}{|{\bf r} - {\bf r}'|}.
\end{equation}

Let us build up our collection of charges one by one. It takes no work to bring the
first charge from infinity, since there is no electric field to fight against.
Let us clamp this charge in position at ${\bf r}_1$. In order to bring the
second charge into position at ${\bf r}_2$,
 we have to do work against the electric field
generated by the first charge. According to Equations~(\ref{e5.3}) and Equations~(\ref{e5.4}),
 this work is given by
\begin{equation}\label{e5.5}
W_2 = \frac{1}{4\pi\epsilon_0} \frac{q_2 \,q_1}{|{\bf r}_2 - {\bf r}_1|}.
\end{equation}
Let us now bring the third charge into position. Since electric fields
and scalar potentials  are
superposable, the work done whilst moving the third charge from infinity to ${\bf r}_3$
is simply the sum of the works done against the electric fields generated by
charges 1 and 2 taken in isolation:
\begin{equation}
W_3 = \frac{1}{4\pi\epsilon_0} \left( \frac{q_3\, q_1}{|{\bf r}_3 - {\bf r}_1|}
+ \frac{q_3\, q_2}{|{\bf r}_3 - {\bf r}_2|}\right).
\end{equation}
Thus, the total work done in assembling the three charges is given by
\begin{equation}
W = \frac{1}{4\pi\epsilon_0}\left( \frac{q_2 \,q_1}{|{\bf r}_2 - {\bf r}_1|}
+\frac{q_3\, q_1}{|{\bf r}_3 - {\bf r}_1|}
+ \frac{q_3 \,q_2}{|{\bf r}_3 - {\bf r}_2|}\right).
\end{equation}
This result can easily be generalized to $N$ charges:
\begin{equation}
W = \frac{1}{4\pi\epsilon_0} \sum_{i=1}^N \sum_{j<i}^N\frac{q_i \,q_j}{|{\bf r}_i
-{\bf r}_j|}.
\end{equation}
The restriction that $j$ must be less than $i$ makes the above summation
rather messy. If we were to sum without restriction (other than $j\neq i$) then
each pair of charges would be counted twice. It is convenient to do just
this, and then to divide the result by two. Thus, we obtain
\begin{equation}\label{e5.9}
W =\frac{1}{2} \frac{1}{4\pi\epsilon_0} \sum_{i=1}^N \sum_{
\stackrel{\scriptstyle j=1}{\scriptscriptstyle j\neq i}}^N
\frac{q_i\, q_j}{|{\bf r}_i-{\bf r}_j|}.
\end{equation}
This is the {\em potential energy}\/ ({\em i.e.}, the difference between the total energy
and the kinetic energy) of a collection of charges. We can think of this quantity as the
work required  to bring stationary charges from infinity and assemble them in the
required formation. Alternatively, it is the kinetic energy which would
be released if the collection were dissolved, and  the charges returned to infinity.
But where is this potential energy stored? Let us investigate further.

Equation~(\ref{e5.9}) can be written
\begin{equation}\label{e5.10}
W = \frac{1}{2} \sum_{i=1}^N q_i \,\phi_i,
\end{equation}
where
\begin{equation}\label{e5.11}
\phi_i = \frac{1}{4\pi\epsilon_0}\sum_{
\stackrel{\scriptstyle j=1} {\scriptscriptstyle j\neq i}}^N
\frac{q_j}{|{\bf r}_i - {\bf r}_j|}
\end{equation}
is the scalar potential experienced  by the $i$\,th charge due to the other
charges in the distribution.

Let us now consider the potential energy of a continuous charge distribution.
It is tempting to write
\begin{equation}\label{e5.12}
W = \frac{1}{2} \int \rho\,\phi\,d^3{\bf r},
\end{equation}
by analogy with Equations~(\ref{e5.10}) and (\ref{e5.11}), where 
\begin{equation}\label{e5.13}
\phi({\bf r}) = \frac{1}{4\pi\epsilon_0}\int \frac{\rho({\bf r}')}
{|{\bf r} - {\bf r}'|}\,d^3{\bf r}'
\end{equation}
is the familiar scalar potential generated by a continuous charge distribution
of charge density $\rho({\bf r})$.
Let us try this out. We know from Maxwell's equations that
\begin{equation}
\rho = \epsilon_0 \,\nabla\!\cdot \!{\bf E},
\end{equation}
so Equation~(\ref{e5.12}) can be written
\begin{equation}
W =\frac{\epsilon_0}{2}\int\phi\,  \nabla\!\cdot\! {\bf E} \,d^3{\bf r}.
\end{equation}
Now, vector field theory yields the standard result
\begin{equation}
\nabla \cdot({\bf E} \,\phi) = \phi\,\nabla\!\cdot\!{\bf E} +{\bf E}\! \cdot\!
 \nabla\phi.
\end{equation}
However, $\nabla\phi = - {\bf E}$, so we obtain
\begin{equation}
W = \frac{\epsilon_0}{2} \left[\int \nabla\!\cdot \!({\bf E}\,\phi)\,d^3{\bf r}
+
\int E^2\,d^3 {\bf r}\right]
\end{equation}
Application of Gauss' theorem gives
\begin{equation}\label{e5.18}
W = \frac{\epsilon_0}{2} \left(\oint_S \phi\,{\bf E} \cdot d{\bf S}+
\int_V E^2\,dV\right),
\end{equation}
where $V$ is some volume which encloses all of the charges, and $S$ is its bounding
surface. Let us assume that $V$ is a sphere, centred on the origin, and let
us take the limit in which the radius $r$  of this sphere goes  to infinity.
We know that, in general, the electric field at large distances from a
bounded charge
distribution looks  like the field of a point charge, and, therefore,
falls off like $1/r^2$. Likewise, the potential falls off like $1/r$---see Exercise~2.7. However,
the surface area of the sphere increases like $r^2$. Hence, it is clear that, in the
limit as $r\rightarrow \infty$, the surface integral in Equation~(\ref{e5.18}) falls off
like $1/r$, and is consequently zero. 
 Thus, Equation~(\ref{e5.18}) reduces to
\begin{equation}\label{e5.19}
W = \frac{\epsilon_0}{2} \int E^2\,d^3{\bf r},
\end{equation}
where the integral is over all space. This is a very interesting
result. It tells us that the potential energy of a continuous charge
distribution is stored in the {\em electric field}\/ generated by the distribution. Of course, we now have to assume that
an electric field possesses an {\em energy density}
\begin{equation}\label{eeu}
U  = \frac{\epsilon_0}{2} \,E^2.
\end{equation}

We can easily check that Equation~(\ref{e5.19}) is correct. Suppose that we have a
 charge $Q$ which is uniformly distributed within a sphere of
radius $a$ centered on the origin. Let us imagine building  up this charge distribution 
from  a succession of thin spherical layers of infinitesimal thickness. At each
stage, we gather a small amount of charge $dq$ from infinity, and spread it 
over  the surface of the sphere in a thin
layer extending from $r$ to $r+dr$. We continue this process until the final radius of the
sphere is $a$.  If $q(r)$ is the sphere's charge when it has attained radius
$r$, then the work done in bringing a charge $dq$ to its surface is
\begin{equation}\label{e5.21}
dW = \frac{1}{4\pi\epsilon_0} \frac{ q(r)\,dq}{r}.
\end{equation}
This follows from Equation~(\ref{e5.5}), since the electric field generated outside a spherical charge
distribution 
is the same as that of a point charge $q(r)$ located at its geometric center
($r=0$)---see Section~\ref{s33}. If the constant charge density of the sphere is
$\rho$ then 
\begin{equation}
q(r) = \frac{4\pi}{3} \,r^3\,\rho,
\end{equation}
and
\begin{equation}
dq = 4\pi\, r^2\, \rho\,dr.
\end{equation}
Thus, Equation~(\ref{e5.21}) becomes
\begin{equation}
dW = \frac{4\pi}{3\epsilon_0}\, \rho^2\, r^4\,dr.
\end{equation}
The total work needed to build up the sphere from nothing to radius $a$ is
plainly
\begin{equation}
W = \frac{4\pi}{3\epsilon_0}\, \rho^2 \int_0^a  r^4\,dr = 
\frac{4\pi}{15\epsilon_0}\, \rho^2\,  a^5.
\end{equation}
This can also be written in terms of the total charge
$Q = (4\pi/3)\,a^3\, \rho$ as
\begin{equation}\label{e5.26}
W = \frac{3}{5} \frac{Q^2}{4\pi\epsilon_0\, a}.
\end{equation}

Now that we have evaluated the potential energy of a spherical charge distribution
by the direct method, let us work it out using Equation~(\ref{e5.19}). We shall assume that the
electric field is both radial and spherically symmetric, so that ${\bf E} = E_r(r)\,
{\bf e}_r$. Application of Gauss' law,
\begin{equation}
\oint_S {\bf E} \cdot d{\bf S} = \frac{1}{\epsilon_0} \int_V \rho \,dV,
\end{equation}
where $V$ is a sphere of radius $r$, centered on the origin, gives
\begin{equation}\label{e5.28}
E_r(r) = \frac{Q}{4\pi\epsilon_0} \frac{r}{a^3}
\end{equation}
for $r<a$, and
\begin{equation}\label{e5.29}
E_r(r) = \frac{Q}{4\pi\epsilon_0\,r^2}
\end{equation}
for $r\geq a$. Equations~(\ref{e5.19}), (\ref{e5.28}), and (\ref{e5.29}) yield
\begin{equation}
W = \frac{Q^2}{8\pi\epsilon_0} \left(
\frac{1}{a^6} \int_0^a r^4\,dr + \int_a^\infty \frac{dr}{r^2} \right),
\end{equation}
which reduces to
\begin{equation}
W =  \frac{Q^2}{8\pi\epsilon_0\, a} \left( \frac{1}{5} + 1\right)=  
\frac{3}{5} \frac{Q^2}{4\pi\epsilon_0 \,a}.
\end{equation}
Thus, Equation~(\ref{e5.19}) gives the correct answer. 

The reason  that we have checked Equation~(\ref{e5.19}) so carefully is that, on close inspection,
it  is found to be 
inconsistent with Equation~(\ref{e5.10}), from which it was supposedly derived!
For instance, the energy given by Equation~(\ref{e5.19}) is manifestly positive definite, whereas
the energy given by Equation~(\ref{e5.10}) can be negative (it is certainly negative for
a collection of two point charges of opposite sign). The
inconsistency was introduced into our analysis when we replaced Equation~(\ref{e5.11}) by
Equation~(\ref{e5.13}). In Equation~(\ref{e5.11}), the  self-interaction of the $i$\,th charge with its
own electric field is specifically excluded, whereas it is included in Equation~(\ref{e5.13}). Thus,
the potential energies
(\ref{e5.10}) and (\ref{e5.19}) are different because in the former we start from 
ready-made point charges, whereas in the latter we build up the whole
charge distribution from scratch. Hence, if we were to work out the
potential energy of a point charge distribution using   Equation~(\ref{e5.19}) then 
we would obtain the energy (\ref{e5.10}) {\em plus} the energy required to assemble the
point charges. What is the energy required to assemble a point charge?
In fact, it is {\em infinite}. To see this, let us suppose, for the sake of argument, that
our point charges actually consist of charge uniformly distributed in small
spheres of radius $b$. According to Equation~(\ref{e5.26}), the energy required to assemble the
$i$\,th point charge is
\begin{equation}
W_i = \frac{3}{5} \frac{q_i^{\,2}}{4\pi\epsilon_0 \,b}.
\end{equation}
We can think of this as the self-energy of the $i$\,th charge. 
Thus, we can write
\begin{equation}
W = \frac{\epsilon_0}{2} \int E^2 \,d^3{\bf r} = \frac{1}{2} \sum_{i=1}^N
q_i \,\phi_i + \sum_{i=1}^N W_i
\end{equation}
which enables us to reconcile Equations~(\ref{e5.10}) and (\ref{e5.19}). Unfortunately, if
our point charges really are point charges then $b\rightarrow 0$, and the
self-energy of each charge becomes infinite. Thus, the potential 
energies predicted by Equations~(\ref{e5.10}) and (\ref{e5.19}) differ by an infinite amount. 
What does this all mean? We have to conclude that the idea of locating electrostatic
potential energy in the electric field is inconsistent with the existence of point charges. One way out of this difficulty would be to
say that  elementary charges, such as electrons, are not points objects, but instead have finite spatial extents. Regrettably, there is no experimental
evidence to back up this assertion.
Alternatively, we could say that our classical
theory of electromagnetism breaks down on very small length-scales due to
quantum effects. Unfortunately, the quantum mechanical version of electromagnetism
(which is called Quantum Electrodynamics) 
suffers from the same infinities in the self-energies of charged particles as the classical
version. There is a prescription, called {\em renormalization}, for steering round these 
infinities, and getting finite answers which agree with experimental data to
extraordinary accuracy. However, nobody really understands why this prescription
works. Indeed, the problem of the
infinite self-energies of elementary charged particles is still an unresolved
issue in Physics. 

\section{Ohm's Law}\label{s4.2}
A {\em conductor}\/ is a medium which contains free electric charges (usually electrons) which drift in the presence of an applied electric field, giving rise to an
electric current flowing in the same direction as the field. The well-known
relationship between the current and the voltage in a typical conductor is given by
 {\em Ohm's law}: {\em i.e.}, 
\begin{equation}
V = I\, R,
\end{equation}
where $V$ is the voltage drop across a conductor of electrical resistance $R$ through which a current $I$
flows. Incidentally, the unit of electrical resistance is the
ohm ($\Omega$), which is equivalent to a volt per ampere.

Let us generalize Ohm's law so that it is expressed in terms
of ${\bf E}$ and ${\bf j}$, rather than $V$ and  $I$. Consider a length 
$l$ of a  conductor of uniform cross-sectional area $A$ through which a current
$I$ flows. In general, we expect the electrical
resistance of the conductor to be proportional to its length, $l$, and inversely 
proportional to its cross-sectional area, $A$ ({\em i.e.}, we expect that it is harder to push an electrical
 current down a long
rather than a short wire, and  easier to push a current down a wide rather
than a narrow conducting channel.) Thus, we can write
\begin{equation}\label{e4.35x}
R = \eta\, \frac{l}{A}.
\end{equation}
Here, the constant $\eta$ is called the {\em resistivity}\/ of the conducting medium, and is measured in
units of ohm-meters. Hence, Ohm's law becomes
\begin{equation}
V = \eta\, \frac{l}{A}\, I.
\end{equation}
However, $I/A = j_z$ (supposing that the conductor is aligned along the $z$-axis)
and $V/l = E_z$, so the above equation reduces to
\begin{equation}
E_z = \eta \,j_z.
\end{equation}
There is nothing special about the $z$-axis (in an isotropic conducting medium), so the
previous formula immediately generalizes to 
\begin{equation}\label{e5.38}
{\bf E} = \eta \,{\bf j}.
\end{equation}
This is the vector form of Ohm's law. 

It is fairly easy to account for the above equation physically. Consider a metal
which has $n$ free electrons per unit volume. Of course, the metal also
has a fixed lattice of metal ions whose charge per unit volume is equal and
opposite to that of the free electrons, rendering the medium electrically neutral.
In the presence of an electric field ${\bf E}$, a given free electron accelerates
(from rest at $t=0$) such that its drift velocity is written ${\bf v} = -(e/m_e)\,t\,{\bf E}$,
where $-e$ is the electron charge, and $m_e$ the electron mass. Suppose
that, on average, a drifting electron collides with a metal ion once every $\tau$
seconds. Given that a metal ion is much more massive than an electron,
we expect a free electron to lose all of the momentum it had previously acquired from
the electric field during such a collision.
 It follows that the mean drift  velocity of the free electrons is
$\bar{\bf v} = -(e\,\tau/2\,m_e)\,{\bf E}$. Hence, the mean current density
is ${\bf j} = (n\,e^2\,\tau/2\,m_e)\,{\bf E}$. Thus, the resistivity
can be written
\begin{equation}
\eta = \frac{n\,e^2\,\tau}{2\,m_e}.
\end{equation}
We conclude that the resistivity of a typical conducting medium is determined
by the number density of free electrons, as well as the mean collision
rate of these electrons with the  fixed ions. The fact that ${\bf j}\propto {\bf E}$ leads immediately to the relation (\ref{e4.35x}) between resistivity and
resistance.

A free charge $q$ which moves through a voltage drop $V$ acquires an energy $q\,V$ from the
electric field. In a conducting medium, this energy is dissipated as {\em heat}\/ (the conversion to heat takes place each time a free charge collides with a fixed ion). This type of heating
is called  {\em ohmic heating}. Suppose that $N$ charges per unit time pass through a
conductor. The current flowing is obviously $I= N\,q$. The total energy gained by the
charges, which appears as heat inside the conductor, is
\begin{equation}\label{e5.39}
P = N\,q\,V = I\,V
\end{equation}
per unit time. Thus, the heating power is
\begin{equation}\label{e5.40}
P = I\,V = I^2 \,R = \frac{V^2}{R}.
\end{equation}
Equations (\ref{e5.39}) and (\ref{e5.40}) generalize to
\begin{equation}
P = {\bf j} \cdot {\bf E} = \eta \,j^2,
\end{equation}
where $P$ is now the power dissipated per unit volume inside the conducting medium.

\section{Conductors}\label{s5.3}
Most (but not all) electrical conductors obey Ohm's law. Such conductors are
termed  {\em ohmic}.
Suppose that we apply an electric field to an ohmic  conductor. What is going to
happen? According to Equation~(\ref{e5.38}), the electric field drives
currents. These currents redistribute the charge inside the conductor until the
original electric field is canceled out. At this point, the currents stop
flowing. It might be objected that the currents
could keep flowing in closed loops. According to Ohm's law, this would
require a non-zero  emf, $\oint {\bf E} \cdot d{\bf l}$, 
 acting around each loop (unless the conductor is a
{\em superconductor}, with $\eta = 0$). However, we know that in a steady-state
\begin{equation}
\oint_C {\bf E} \cdot d {\bf l } = 0
\end{equation}
around any closed loop $C$. This proves that a steady-state emf acting around
a closed loop inside a conductor is impossible. The only other alternative is 
\begin{equation}\label{e5.43}
{\bf j} = {\bf E} = {\bf 0}
\end{equation}
everywhere inside  the conductor. It immediately follows from the Maxwell equation
$\nabla\cdot {\bf E} = \rho/\epsilon_0$
that
\begin{equation}
\rho = 0.
\end{equation}
So, there are no electric charges in the interior of a 
conductor. But, how can a conductor cancel out an applied electric field
if it contains no charges? The answer is that all of the charges reside on the
{\em surface}\/ of the conductor. 
In reality, the charges  lie
within one or two atomic layers of the surface (see any textbook on solid-state
physics). 
The difference in scalar potential between
two points $P$ and $Q$ is simply
\begin{equation}\label{e5.45}
\phi(Q) - \phi(P) = \int_P^Q \nabla\phi \cdot d{\bf l} = -
\int_P^Q {\bf E} \cdot d{\bf l}.
\end{equation}
However, if $P$ and $Q$ both lie inside the same conductor 
then it is clear from Equations~(\ref{e5.43}) and (\ref{e5.45}) that the potential difference between $P$ and
$Q$ is zero. This is true no matter where $P$ and $Q$ are situated inside the
conductor, so we conclude that the scalar potential must be
 {\em uniform}\/ inside a conductor.
A corollary of this is that the surface of a conductor is
 an equipotential  ({\em i.e.}, $\phi = $ constant) surface.

So, the electric field inside a conductor is zero. We can demonstrate that the field within an empty cavity lying inside a conductor is
 zero as well, provided that there are no charges within the cavity.
Let us, first of all, apply Gauss' law to a surface $S$ which 
surrounds the cavity, but lies
wholly within the conducting medium---see Figure~\ref{f40}. Since the electric field is zero inside a conductor,
it follows that zero net charge is enclosed by $S$. This does not preclude the
possibility that there are equal amounts of positive and negative charges distributed
on the inner surface of the conductor. However, we can easily rule out this
possibility using the steady-state relation
\begin{equation}\label{e5.46}
\oint_C {\bf E} \cdot d{\bf l} = 0,
\end{equation}
for any closed loop $C$. If there are any electric field-lines inside the cavity
then 
they must run from the positive to the negative surface charges. Consider a
closed loop $C$ which straddles the cavity and the conductor, such as the one 
shown  in Figure~\ref{f40}. In the presence of field-lines, it is clear 
that the line integral of ${\bf E}$
along that portion of the loop 
which lies inside the cavity is non-zero. However, the line integral
of ${\bf E}$ 
along that portion of the loop which runs through the conducting medium
 is obviously
zero (since ${\bf E} = {\bf 0}$ inside a conductor).  Thus, the line integral of
the field around the closed loop $C$ is non-zero, which clearly contradicts
Equation~(\ref{e5.46}).
 In fact, this equation implies that the line integral of the electric field
along any path which runs through the cavity, from one point on the interior surface
of the conductor to another, is zero. This can only be the case if the electric
field itself is zero everywhere inside the cavity. (The above argument
is not entirely rigorous. In particular, it is not clear that it fails if there
are charges inside the cavity. We shall discuss an improved argument later on
in this chapter.)
\begin{figure}
\epsfysize=2.5in
\centerline{\epsffile{chapter5/fig5.1.eps}}
\caption{\em An empty cavity inside a conductor.}\label{f40}
\end{figure}

We have shown that if a charge-free cavity is completely enclosed by a conductor then no
stationary distribution of charges outside the conductor can ever produce any electric fields inside the cavity.
It follows that we can shield a sensitive piece of electrical equipment from stray external electric fields
by placing it inside  a metal can. In fact, a wire mesh cage will do, as long as
the mesh spacing is not too wide. Such a cage is known as a {\em Faraday
cage}.

Let us consider some  small region on the surface of a conductor. Suppose that
the local surface charge density is $\sigma$, and that the electric field just outside
the conductor is ${\bf E}$. Note that this field must be directed {\em normal}
to the surface of the conductor. Any parallel component would be shorted out
by surface currents. Another way of saying this is that the surface of a conductor
is an equipotential. We know that $\nabla\phi$ is always perpendicular to 
an equipotential, so ${\bf E} = - \nabla\phi$ must be locally perpendicular
 to a conducting surface. Let us use Gauss' law,
\begin{equation}
\oint_S {\bf E} \cdot d{\bf S} = \frac{1}{\epsilon_0}\int_V \rho\, dV,
\end{equation}
where $V$ is  a so-called  {\em Gaussian pill-box}---see Figure~\ref{f42}. This is a pill-box shaped volume whose two ends are aligned parallel to the surface of the conductor,
with the surface running between them, and whose sides are perpendicular to the
surface.
It is clear that ${\bf E}$ is parallel to the sides of the box, so the sides
make no contribution to the surface integral. The end of the box  which lies
inside  the conductor also makes no contribution,
since ${\bf E} ={\bf  0}$ 
inside a conductor. Thus, the only non-zero contribution to the
surface integral comes from the end lying in free space. This contribution
is simply $E_\perp \,A$, where $E_\perp$ denotes an outward pointing (from the
conductor)  normal
electric field, and $A$ is the cross-sectional area of the box. 
The charge enclosed
by the box is simply $\sigma\,A$, from the definition of a surface charge density.
Thus, Gauss' law yields
\begin{equation}\label{e5.58}
E_\perp = \frac{\sigma}{\epsilon_0}
\end{equation}
as  the relationship between the normal electric field immediately outside a conductor
and the surface charge density.
\begin{figure}
\epsfysize=2.5in
\centerline{\epsffile{chapter5/fig5.2.eps}}
\caption{\em The surface of a conductor.}\label{f42}
\end{figure}

Let us look at the electric field generated by a sheet charge distribution
a little more carefully. Suppose that the  charge per unit area is $\sigma$.
By symmetry, we expect the field generated below the sheet to be the mirror image
of that above the sheet (at least, locally). Thus, if we integrate Gauss' law over
a pill-box of cross-sectional area $A$, as shown in Figure~\ref{f41}, then  the
two ends both contribute $E_{\rm sheet}\,A$ to the surface integral, where $E_{\rm
sheet}$ is the normal
electric field generated above and below the sheet. The charge enclosed 
by the pill-box is just $\sigma\,A$. Thus, Gauss' law yields
a symmetric electric field 
\begin{equation}\label{e5.49}
E_{\rm sheet}= \left\{
\begin{array}{lll}
+\sigma/(2\,\epsilon_0)& \mbox{\hspace{2cm}}&\mbox{above}\\[0.5ex] 
-\sigma/(2\,\epsilon_0)&\mbox{\hspace{2cm}}&\mbox{below}
\end{array}
\right..
\end{equation}
So, how do we get the asymmetric electric field of a conducting surface,
 which is zero immediately below the surface ({\em i.e.}, inside the conductor) and
non-zero immediately above it? Clearly, we have to add in an external field
({\em i.e.}, a field which is not generated locally by the sheet charge). 
The 
requisite field is
\begin{equation}
E_{\rm ext} = \frac{\sigma}{2\,\epsilon_0}
\end{equation}
both above and below the charge sheet. The total field is the sum of the field
generated locally by the charge sheet and the external field. Thus, we obtain
\begin{equation}
E_{\rm total}= \left\{
\begin{array}{lll}
+\sigma/\epsilon_0& \mbox{\hspace{2cm}}&\mbox{above}\\[0.5ex] 
0&\mbox{\hspace{2cm}}&\mbox{below}
\end{array}
\right.,\label{e5.50}
\end{equation}
which is in agreement with Equation~(\ref{e5.58}).
\begin{figure}
\epsfysize=2.in
\centerline{\epsffile{chapter5/fig5.3.eps}}
\caption{\em The electric field of a sheet charge.}\label{f41}
\end{figure}

Now, the external field exerts a force on the charge sheet. Of course, the field generated locally
by
the sheet itself  cannot exert a local force ({\em i.e.}, the charge sheet cannot exert
a force on itself). Thus, the force per unit area acting on the surface of a
conductor always acts outward, and is given by
\begin{equation}
p = \sigma \,E_{\rm ext} = \frac{\sigma^2}{2\,\epsilon_0}.
\end{equation}
We conclude that there is an {\em electrostatic pressure}\/ acting on any charged conductor.
This effect can be observed by charging up soap bubbles: the additional
electrostatic pressure eventually causes them to burst. The electrostatic pressure
can also be written
\begin{equation}
p = \frac{\epsilon_0}{2} \,E_\perp^{\,2},
\end{equation}
where $E_\perp$ is the field-strength immediately above the surface of the conductor.
Note that, according to the above formula, the electrostatic pressure is equivalent
to the energy density of the electric field immediately outside the conductor. 
This is not a coincidence. Suppose that the conductor expands normally by an average 
distance $dx$, due to the electrostatic pressure. The electric field is excluded
from the region into which the conductor expands. The volume of this region
is $dV = A \,dx$, where $A$ is the surface area of the conductor. Thus, the energy
of the electric field decreases by an amount $dE =  U\,dV = (\epsilon_0/2) \,E_\perp^{\,2}
\,dV$,
where $U$ is the energy density of the field. This decrease in energy can be 
ascribed to the work which the field does on the conductor in order to make it expand.
This work is $dW = p\,A\,dx$, where $p$ is the force per unit area that the field exerts
on the conductor. Thus, $dE = dW$, from energy conservation, giving
\begin{equation}
p = \frac{\epsilon_0}{2} \,E_\perp^{\,2}.
\end{equation}
Incidentally, this technique for calculating a force,  given an expression
for the  energy of a
system as a function of some adjustable parameter,
 is called {\em the principle of virtual work}.

We have seen that  
an electric field is excluded from the inside of a conductor, but not
from the outside, giving  rise to a net {\em outward}\/
force. We can account for this fact by saying that the field exerts a 
{\em negative} pressure
$(\epsilon_0/2)\, E_\perp^{\,2}$ on the conductor. 
Now, we know that if we evacuate a  closed metal can then the
pressure difference between the inside and the outside  eventually causes
it to {\em implode}. Likewise, if we place the can in a strong electric field then
the pressure difference between the inside and the outside will eventually cause
it to {\em explode}.
 How big a field do we need before the electrostatic pressure difference
is the same as that obtained by 
evacuating the can? In other words, what electric field exerts a negative
pressure of one atmosphere ({\em i.e.}, $10^5$ newtons per meter squared) on the can?
The answer is a field of strength $E\sim 10^8$ volts per meter. 
Fortunately, this is a rather
large electric field, so there is no danger of your  car exploding when you turn on the
radio!

\section{Boundary Conditions on the Electric Field}
What are the  general boundary conditions satisfied by the electric field
at the interface between two different media: {\em e.g.}, the interface between a vacuum
and a conductor? Consider an interface $P$ between two media $1$ and $2$.
Let us, first of all, apply Gauss' law,
\begin{equation}\label{e4.56x}
\oint_S {\bf E} \cdot d{\bf S} = \frac{1}{\epsilon_0} \int_V \rho \,dV,
\end{equation}
to a Gaussian pill-box $S$ of cross-sectional area $A$ whose two ends are 
locally parallel to the interface---see Figure~\ref{f43}. The ends of the box can be made arbitrarily
close together. In this limit, the flux of the electric field out of the sides
of the box is obviously negligible, and the only contribution to the flux comes
from the two ends. In fact,
\begin{equation}
\oint_S {\bf E} \cdot d{\bf S} = ( E_{\perp\,1} - E_{\perp\,2})\,A,
\end{equation}
where $E_{\perp\,1}$ is the perpendicular (to the interface)
electric field in
medium $1$ at the interface, {\em etc.} The charge enclosed by the pill-box is
simply $\sigma\,A$, where $\sigma$ is the sheet charge density on the interface.
Note that any volume distribution of charge gives rise to a negligible contribution
to the right-hand side of Equation~(\ref{e4.56x}), in the limit where the two
ends of the pill-box are very closely spaced. Thus, Gauss' law
yields
\begin{equation}
E_{\perp\,1} - E_{\perp \,2} = \frac{\sigma}{\epsilon_0}
\end{equation}
at the interface: {\em i.e.}, 
the presence of a  charge sheet on an interface causes  a discontinuity in the
perpendicular component of the electric field. What about the parallel electric
field? Let us apply Faraday's law to a rectangular
 loop $C$ whose long sides, length $l$, run parallel to the interface,
\begin{equation}
\oint_C {\bf E} \cdot d{\bf l} = -\frac{\partial}{\partial t} \int_S {\bf B}\cdot
d{\bf S}
\end{equation}
---see Figure~\ref{f43}.
The length of the short sides is assumed to be arbitrarily small. Hence, the dominant
contribution to the loop integral comes from the long sides:
\begin{equation}
\oint_C {\bf E} \cdot d{\bf l} = (E_{\parallel\,1} - E_{\parallel\,2} )\, l,
\end{equation}
where
$E_{\parallel\,1}$ is the parallel (to the interface) electric field in
medium $1$ at the interface, {\em etc.} The flux of the magnetic field through the
loop is approximately $B_\perp\,A$, where $B_\perp$ is the component of the
magnetic field which is normal to the loop, and $A$ the area of the loop.
But, $A\rightarrow 0$ as the short sides of the loop are shrunk to zero. So,
unless the magnetic field becomes infinite at the interface (and we shall assume that it does not), the
flux also tends to zero. Thus,
\begin{equation}
E_{\parallel\,1} - E_{\parallel\,2}=0:
\end{equation}
{\em i.e.}, there can be no discontinuity in the parallel
component of the  electric field across an interface.
\begin{figure}
\epsfysize=1.5in
\centerline{\epsffile{chapter5/fig5.4.eps}}
\caption{\em Boundary conditions on the electric field.}\label{f43}
\end{figure}


\section{Capacitors}\label{scap}
It is clear that we can store electrical 
charge on the surface of a conductor. However,
electric fields will be generated immediately above this surface.
Now, the conductor can only successfully store charge if it is electrically insulated
from its surroundings. Of course, air is a very good insulator. Unfortunately, air
ceases to be an insulator when the electric field-strength through it exceeds some
critical value which is about $E_{\rm crit} \sim 10^6$ volts per meter. This
phenomenon, which is called {\em breakdown}, is associated with the formation
of sparks. The most well-known example of the breakdown of 
air is during a lightning
strike. Thus, a good charge storing device is one which holds a relatively large amount
of charge, but only generates relatively small external electric fields (so as to avoid breakdown). Such a device is called
a {\em capacitor}.

Consider two thin, parallel, conducting 
plates of cross-sectional area $A$ which are separated by
a {\em small}\/ distance $d$ ({\em i.e.}, $d\ll \sqrt{A}$). Suppose that each plate
carries an equal and opposite charge $\pm Q$ (where $Q>0$). We expect this charge to
spread evenly over the plates to give an effective sheet charge density 
$\pm\sigma = Q/A$ on each plate. Suppose that the upper plate carries a
positive charge and that the lower carries a negative charge. According to
Equation~(\ref{e5.49}), the field generated by the upper plate is normal to the plate and
of magnitude
\begin{equation}
E_{\rm upper} =\left\{
\begin{array}{lcl}
+\sigma /(2 \epsilon_0)&\mbox{\hspace{2cm}}&\mbox{above}
\\[0.5ex]
-\sigma/(2\epsilon_0)& \mbox{\hspace{2cm}}&\mbox{below}
\end{array}\right..
\end{equation}
Likewise, the field generated by the lower plate is
\begin{equation}
E_{\rm lower} =\left\{
\begin{array}{lcl}
-\sigma /(2 \epsilon_0)&\mbox{\hspace{2cm}}&\mbox{above}
\\[0.5ex]
+\sigma/(2\epsilon_0)& \mbox{\hspace{2cm}}&\mbox{below}
\end{array}\right..
\end{equation}
Note that we are neglecting any ``leakage'' of the field at the edges of the plates.
This is reasonable provided that the plates are relatively closely spaced. The total field is the
sum of the two fields generated by the upper and lower plates. Thus, the net field
is normal to the plates, and of magnitude
\begin{equation}
E_\perp =\left\{
\begin{array}{lcl}
\sigma /\epsilon_0&\mbox{\hspace{2cm}}&\mbox{between}
\\[0.5ex]
0& \mbox{\hspace{2cm}}&\mbox{otherwise}
\end{array}\right.
\end{equation}
---see Figure~\ref{fplate}.
Since the electric field is uniform, the potential difference between the
plates is simply
\begin{equation}
V = E_\perp\,d = \frac{\sigma \,d}{\epsilon_0}.
\end{equation}
\begin{figure}
\epsfysize=2.in
\centerline{\epsffile{chapter5/fig5.5.eps}}
\caption{\em The electric field of a parallel plate capacitor.}\label{fplate}
\end{figure}

Now, it is conventional to measure the capacity of a conductor, or set of conductors,
to store charge, but generate small external electric fields, in terms of a parameter
called the {\em capacitance}. This parameter is
usually denoted $C$. The capacitance of a charge storing
device is simply the ratio of the charge stored to the potential difference
generated by this charge: {\em i.e.}, 
\begin{equation}\label{e5.65}
C = \frac{Q}{V}.
\end{equation}
Clearly, a good charge storing device has a high capacitance. Incidentally,
capacitance is measured in farads (F), which are equivalent to coulombs per volt. This is a rather unwieldy
unit, since
capacitors in electrical circuits typically have capacitances which are only about one millionth
of a farad. For a parallel plate capacitor, we have
\begin{equation}\label{e5.66}
C = \frac{\sigma\,A}{V} = \frac{\epsilon_0\,A}{d}.
\end{equation}
Note that the capacitance only depends on {\em geometric}\/ quantities, such as the area
and spacing of the plates. This is a consequence of the superposability of
electric fields. If we double the charge on a set of conductors then we double
the electric fields generated around them, and we, therefore, double the potential
difference between the conductors. Thus, the potential difference between
the conductors is always directly proportional to the charge on the conductors:
 the constant
of proportionality (the inverse of the capacitance) can only depend on geometry.

Suppose that the charge $\pm Q$ on each plate of a parallel plate capacitor is built up gradually by transferring
small amounts of charge from one plate to another. If the 
instantaneous charge on the plates is $\pm q$, and an infinitesimal amount of
positive
charge $dq$ is transferred from the negatively charged  to the positively
charge plate, then the work done is $dW= V\,dq = q\,dq/C$, 
where $V$ is the instantaneous
voltage difference between the plates. Note that the voltage difference is such
that it opposes any increase in  the charge on either plate. 
The total work done in charging the capacitor
is
\begin{equation}\label{e5.67}
W =\frac{1}{C}  \int_0^Q q \,dq =  \frac{Q^2}{2\,C} = \frac{1}{2}\,C\,V^2,
\end{equation}
where use has been made of Equation~(\ref{e5.65}). 
The energy stored in the capacitor is  the same as the work required to
charge up the capacitor. Thus, the stored energy is
\begin{equation}
W= \frac{1}{2}\, C \,V^2.
\end{equation}
This is a general result which holds for all types of capacitor. 

The energy of
a charged parallel
plate
capacitor is actually stored in the electric field between the plates. This field
is of approximately constant magnitude $E_\perp = V/d$, and occupies a
region of volume $A\,d$. Thus, given the energy density of an electric
field, $U = (\epsilon_0/2)\,E^{2}$, the energy stored in the 
electric field is 
\begin{equation}\label{e5.69}
W = \frac{\epsilon_0}{2} \frac{V^2}{d^2}\, A \,d= \frac{1}{2}\, C\, V^2,
\end{equation}
where use has been made of Equation~(\ref{e5.66}). 
Note that Equations~(\ref{e5.67}) and (\ref{e5.69}) agree with one another. The
fact that the energy of a capacitor is stored in its electric field is also
a general result.

The idea, which we discussed earlier, that an electric field exerts a negative
pressure $(\epsilon_0/2)\,E_\perp^{\,2}$ on conductors immediately suggests that
the two plates in a parallel plate capacitor {\em attract}\/ one another with a
mutual force
\begin{equation}\label{eforcc}
F = \frac{\epsilon_0}{2}\, E_\perp^{~2}\,A= \frac{1}{2} \frac{C\, V^2}{d}.
\end{equation}

It is not actually necessary to have two oppositely charged conductors 
in order to make a capacitor.
Consider an isolated
 conducting sphere of radius $a$ which
carries an electric charge $Q$. The spherically symmetric radial electric field generated outside the sphere is
given by
\begin{equation}\label{e5.71}
E_r(r>a) = \frac{Q}{4\pi\epsilon_0 \,r^2}.
\end{equation}
It follows that the potential difference between the sphere and infinity---or, more realistically,
some large, relatively distant reservoir of charge such as the Earth---is
\begin{equation}\label{e5.72}
V = \frac{Q}{4\pi\epsilon_0 \,a}.
\end{equation}
Thus, the capacitance of the sphere is
\begin{equation}
C = \frac{Q}{V} = 4\pi\epsilon_0\, a.
\end{equation}
The energy of a spherical capacitor when it carries a charge $Q$ is again given by
$(1/2)\, C\,V^2$. It can easily be demonstrated that this is really 
the energy contained in the electric field surrounding the capacitor.


Suppose that we have two spheres of radii $a$ and $b$, respectively, which are
connected by a {\em long}\/ electric wire---see Figure~\ref{fsphere2}. The wire allows charge to move back and forth between
the spheres until they reach the same potential (with respect to infinity).
Let $Q_a$ be the charge on the first sphere, and  $Q_b$ the charge on the
second sphere.
Of course, the total charge $Q_0= Q_a +Q_b$ carried by the two spheres is a conserved
quantity. It follows from Equation~(\ref{e5.72}) that
\begin{eqnarray}
\frac{Q_a}{Q_0}& =&  \frac{a}{a+b},\\[0.5ex]
\frac{Q_b}{Q_0} &=& \frac{b}{a+b}.
\end{eqnarray}
Note that if one sphere is much smaller than the other one, {\em e.g.}, $b\ll a$, then
the large sphere grabs most of the charge: {\em i.e.}, 
\begin{equation}\label{e5.75}
\frac{Q_a}{Q_b} \simeq \frac{a}{b} \gg 1.
\end{equation}
The ratio of the electric fields generated just above the surfaces of the two
spheres follows from Equations~(\ref{e5.71}) and (\ref{e5.75}):
\begin{equation}\label{e5.82x}
\frac{E_b}{E_a}\simeq \frac{a}{b}.
\end{equation}
Note that if $b\ll a$ then the field just above the smaller sphere
is far larger  than that  above the larger sphere.
Equation~(\ref{e5.82x}) is a simple  example of a far more general rule:  {\em i.e.}, 
the electric field above  some point on the
surface of a conductor is inversely proportional to
the local radius of curvature of the surface. 
\begin{figure}
\epsfysize=1.5in
\centerline{\epsffile{chapter5/fig5.6.eps}}
\caption{\em Two conducting spheres connected by a wire.}\label{fsphere2}
\end{figure}

It is clear that if we wish to store significant amounts of charge on a conductor
then the surface of the conductor must be made as {\em smooth}\/ as possible. Any sharp
spikes on the surface will inevitably
have comparatively small radii of curvature. Intense local electric fields are thus
generated around such spikes.  These fields can easily exceed the critical field for the breakdown of air,
leading to sparking and the eventual loss of the charge on the conductor. 
Sparking can also be very destructive, because the associated
electric  currents flow through very localized
regions, giving rise to intense ohmic heating.

As a final example, consider two co-axial
 conducting cylinders of radii $a$ and
$b$, where $a<b$. Suppose that the charge per unit length carried by the
outer and inner cylinders is $+\lambda$ and $-\lambda$, respectively. We can safely
assume that ${\bf E} = E_r(r)\, {\bf e}_r$, by symmetry (adopting
standard cylindrical polar coordinates). Let us apply
Gauss' law to a cylindrical surface of radius $r$, co-axial with the conductors, and
of length $l$. For $a<r<b$, we find that
\begin{equation}
2\pi \,r\, l\,E_r(r) = \frac{\lambda\,l}{\epsilon_0},
\end{equation}
so that
\begin{equation}
E_r = \frac{\lambda}{2\pi\epsilon_0\,r}
\end{equation}
for $a<r<b$. It is fairly obvious that $E_r=0$ if $r$ is not in the range
$a$ to $b$. The potential difference between the inner and outer cylinders is
\begin{eqnarray}
V &=& - \int_{\rm outer}^{\rm inner} {\bf E} \cdot d{\bf l} = 
\int_{\rm inner}^{\rm outer}{\bf E} \cdot d{\bf l} \nonumber\\[0.5ex]
&=& \int_a^b E_r\,dr = \frac{\lambda}{2\pi \epsilon_0} \int_a^b \frac{dr}{r},
\end{eqnarray}
so
\begin{equation}
V = \frac{\lambda}{2\pi \epsilon_0} \ln \frac{b}{a}.
\end{equation}
Thus, the capacitance per unit length of the two cylinders is
\begin{equation}
C = \frac{\lambda}{V} = \frac{2\pi\epsilon_0}{\ln b/a}.
\end{equation}

\section{Poisson's Equation}
Now, we know that in a steady-state we can write
\begin{equation}
{\bf E} = -\nabla\phi,
\end{equation}
with the scalar potential satisfying Poisson's equation:
\begin{equation}
\nabla^2 \phi = - \frac{\rho}{\epsilon_0}.
\end{equation}
We even know the general solution to this equation:
\begin{equation}\label{e5.84}
\phi({\bf r}) = \frac{1}{4\pi\epsilon_0} \int \frac{\rho({\bf r}')}
{|{\bf r} - {\bf r'}|}\,d^3{\bf r}'.
\end{equation}
So, what else is there to say about Poisson's equation? Well, consider a 
positive (say) point
charge in  the vicinity of an uncharged, insulated, conducting sphere. The charge
attracts negative charges to the near side of the sphere, and repels positive
charges to the far side. The surface charge distribution induced on the
sphere is such that the surface is maintained at a constant electrical potential.
We now have a problem. We cannot use formula (\ref{e5.84}) to work out the
potential $\phi({\bf r})$ around the sphere, since we do not know beforehand how the
 charges induced on its  conducting surface  are distributed. The only things
which  we know about the surface   are that it is an equipotential,
and carries zero net charge.  Clearly, the
solution (\ref{e5.84})  to Poisson's equation is completely useless in the presence of conducting surfaces. Let us now try to
develop some techniques for solving Poisson's equation which allow us to solve
real problems (which invariably involve conductors). 
 
\section{Uniqueness Theorem}\label{suniq}
We have already seen the great value of the uniqueness theorem for Poisson's
equation (or Laplace's equation) in our discussion of the Helmholtz theorem (see
Section~\ref{s310}). Let us now examine the uniqueness theorem in  detail.


Consider a volume $V$ bounded by some surface $S$---see Figure~\ref{funq}. Suppose that we are given
the charge density $\rho$ throughout $V$, and the value of the scalar potential
$\phi_S$ on $S$. Is this sufficient information to uniquely specify the scalar
potential throughout $V$? Suppose, for the sake of argument, that the  
solution is not unique. Let there be two different potentials $\phi_1$ and $\phi_2$ which
satisfy
\begin{eqnarray}
\nabla^2 \phi_1 &=& - \frac{\rho}{\epsilon_0}, \\[0.5ex]
\nabla^2 \phi_2 &=& - \frac{\rho}{\epsilon_0}
\end{eqnarray}
throughout $V$, and
\begin{eqnarray}
\phi_1 &=& \phi_S, \\[0.5ex]
 \phi_2 &=& \phi_S
\end{eqnarray}
on $S$. We can form the difference between these two potentials:
\begin{equation}
\phi_3 = \phi_1 - \phi_2.
\end{equation}
The potential $\phi_3$ clearly satisfies
\begin{equation}
\nabla^2\phi_3 = 0
\end{equation}
throughout $V$, and
\begin{equation}
\phi_3 =0 
\end{equation}
on $S$. 
\begin{figure}
\epsfysize=2.25in
\centerline{\epsffile{chapter5/fig5.7.eps}}
\caption{\em The first uniqueness theorem.}\label{funq}
\end{figure}

Now, according to vector field theory,
\begin{equation}
\nabla\cdot(\phi_3\,\nabla\phi_3) \equiv (\nabla \phi_3)^2  +\phi_3\nabla^2\phi_3.
\end{equation}
Thus, using Gauss' theorem,
\begin{equation}
\int_V \left[ (\nabla\phi_3)^2 +\phi_3 \nabla^2\phi_3\right] dV = 
\oint_S \phi_3 \nabla \phi_3\cdot d{\bf S}. 
\end{equation}
But, $\nabla^2\phi_3 = 0$ throughout $V$, and $\phi_3=0$ on $S$, so the
above equation reduces
to
\begin{equation}
\int_V (\nabla \phi_3)^2\,dV = 0.
\end{equation}
Note that $(\nabla \phi_3)^2$ is a {\em positive definite} quantity. The only way in
which the volume integral of a positive definite quantity can be zero is if
that quantity itself is zero throughout the volume. This is not necessarily the
case for a non-positive definite quantity: we could have positive and negative
contributions from various regions inside the volume which cancel one another out.
Thus, since $(\nabla \phi_3)^2$ is positive definite, it follows that 
\begin{equation}
\phi_3 = {\rm constant}
\end{equation}
throughout $V$. However, we know that $\phi_3 =0 $ on $S$, so we get
\begin{equation}
\phi_3 = 0
\end{equation}
throughout $V$. In other words,
\begin{equation}
\phi_1 = \phi_2
\end{equation}
throughout $V$ and on $S$. Our initial assumption that $\phi_1$ and $\phi_2$
are two different solutions of Poisson's equation, satisfying the same
boundary conditions, turns out to be incorrect. Hence, the solution is unique.

The fact that the solutions to Poisson's equation are unique is very useful.
It means that if we find a solution to this equation---no matter how contrived
the derivation---then this is the only possible solution. One immediate use of the
uniqueness theorem is to prove that the electric field inside an empty cavity
situated within a conductor is zero. Recall that our previous proof of this was rather involved,
and was  also not particularly rigorous (see Section~\ref{s5.3}). 
Now, we know that the interior surface of the conductor is at some constant potential
$\phi_0$, say. So, we have $\phi=\phi_0$ on the boundary of the cavity, and 
$\nabla^2 \phi
 =0 $ inside the cavity (since it contains no charges). One rather obvious
solution to this problem is $\phi = \phi_0$ throughout the cavity. Since the
solutions to Poisson's equation are unique, this is the {\em only}\/ solution.
Thus,
\begin{equation}
{\bf E} = - \nabla\phi = -\nabla \phi_0 = {\bf 0}
\end{equation}
inside the cavity.


Suppose that some volume $V$  contains a number of conductors---see Figure~\ref{funq1}. We know that the
surface of each conductor is an equipotential, but, in general, we do not
know the potential of a given conductor  (unless we are specifically told that
the conductor is earthed, {\em etc.}). However, if the conductors are insulated then it is
plausible that we might know the charge on each conductor. Suppose that
there are $N$ conductors, each carrying a known charge $Q_i$ ($i=1$ to $N$), and suppose
that the region $V$ containing these conductors is filled by a known charge
density $\rho$, and bounded by some surface $S$ which is either infinity or
an enclosing conductor. Is this sufficient information to uniquely
specify the electric field throughout $V$?
\begin{figure}
\epsfysize=2.25in
\centerline{\epsffile{chapter5/fig5.8.eps}}
\caption{\em The second uniqueness theorem.}\label{funq1}
\end{figure}

Well, suppose that it is not  sufficient information, so that there are two different
fields ${\bf E}_1$ and ${\bf E}_2$ which satisfy 
\begin{eqnarray}
\nabla\cdot{\bf E}_1& =& \frac{\rho}{\epsilon_0},\\[0.5ex]
\nabla\cdot{\bf E}_2 &= &\frac{\rho}{\epsilon_0}
\end{eqnarray}
throughout $V$, with
\begin{eqnarray}
\oint_{S_i} {\bf E}_1\cdot d{\bf S}_i &=& \frac{Q_i}{\epsilon_0},\\[0.5ex]
\oint_{S_i} {\bf E}_2\cdot d{\bf S}_i &=& \frac{Q_i} {\epsilon_0}
\end{eqnarray}
on the surface of the $i$th conductor, and, finally,
\begin{eqnarray}
\oint_{S} {\bf E}_1\cdot d{\bf S} &=& \frac{Q_{\rm total}}{\epsilon_0},
\\[0.5ex]
\oint_{S} {\bf E}_2\cdot d{\bf S} &=& \frac{Q_{\rm total}} {\epsilon_0}
\end{eqnarray}
over the bounding surface, 
where
\begin{equation}
Q_{\rm total} = \sum_{i=1}^N Q_i + \int_V \rho\,dV
\end{equation}
is the total charge contained in volume $V$. 

Let us form the difference field
\begin{equation}
{\bf E}_3 = {\bf E}_1 - {\bf E}_2.
\end{equation}
It is clear that 
\begin{equation}
\nabla\cdot {\bf E}_3 = 0
\end{equation}
throughout $V$, and
\begin{equation}\label{e5.103}
\oint_{S_i} {\bf E}_3 \cdot d{\bf S}_i = 0
\end{equation}
for all $i$, with
\begin{equation}\label{e5.104}
\oint_S {\bf E}_3 \cdot d{\bf S} = 0.
\end{equation}

Now, we know that each conductor is at a constant potential, so if
\begin{equation}
{\bf E}_3 = -\nabla\phi_3,
\end{equation}
then $\phi_3$ is a constant on the surface of each conductor. Furthermore,
if the outer surface $S$ is infinity then $\phi_1=\phi_2 = \phi_3 =0$ on this
surface. On the other hand, if the outer surface is an enclosing conductor then $\phi_3$ is
a constant on it. Either way, $\phi_3$ is constant on $S$. 

Consider the vector identity 
\begin{equation}
\nabla\cdot(\phi_3\,{\bf E}_3) \equiv \phi_3\,\nabla \cdot{\bf E}_3 + {\bf E}_3\cdot
\nabla\phi_3.
\end{equation}
We have $\nabla\cdot {\bf E}_3 = 0$ throughout $V$, and $\nabla\phi_3 = -{\bf E}_3$,
so the above identity reduces to
\begin{equation}
\nabla\cdot (\phi_3 \,{\bf E}_3) = - E_3^{~2}
\end{equation}
throughout $V$. Integrating over $V$, and making use of Gauss' theorem, yields
\begin{equation}
\int_V E_3^{~2}\,dV =  \sum_{i=1}^N \oint_{S_i} \phi_3\,{\bf E}_3\cdot d{\bf S}_i
-\oint_S \phi_3\,{\bf E}_3\cdot d{\bf S}.
\end{equation}
However, $\phi_3$ is a constant on the surfaces $S_i$ and  $S$. So, making use of
Equations~(\ref{e5.103}) and (\ref{e5.104}), we obtain
\begin{equation}
\int_V E_3^{~2} \,dV =0.
\end{equation}
Of course, $E_3^{~2}$ is a positive definite quantity, so the above relation
implies that
\begin{equation}
{\bf E}_3 = {\bf 0} 
\end{equation}
throughout $V$: {\em i.e.}, the fields ${\bf E}_1$ and ${\bf E}_2$ are
identical throughout $V$. Hence, the solution is unique.

For a general electrostatic problem involving charges and
conductors, it is clear that  if we are given either the potential at the surface of each conductor
or the charge carried by each conductor 
(plus the charge density throughout the volume, {\em etc.})\
then we can uniquely determine the electric
field. There are many other uniqueness theorems which generalize this result
still further: {\em e.g.}, we could be given the potentials on the surfaces of some of the conductors,
and the charges on the surfaces of the others, and the solution would still be unique. 

At this point, it is worth noting  that there are also uniqueness theorems associated with
magnetostatics. For instance, if the current density, ${\bf j}$, is specified
throughout some volume $V$, and either the magnetic field, ${\bf B}$,
or the vector potential, ${\bf A}$, is specified on the bounding surface $S$, then
the magnetic field is uniquely determined throughout $V$ and on $S$.
The proof of this proposition proceeds along the usual lines. Suppose
that the magnetic field is not uniquely determined. In other words,
suppose there are two different magnetic fields, ${\bf B}_1$ and ${\bf B}_2$, 
satisfying
\begin{eqnarray}
\nabla\times{\bf B}_1 &=&\mu_0\,{\bf j},\\[0.5ex]
\nabla\times{\bf B}_2 &=&\mu_0\,{\bf j},
\end{eqnarray}
throughout $V$. Suppose, further, that either ${\bf B}_1={\bf B}_2={\bf B}_S$ or ${\bf A}_1={\bf A}_2={\bf A}_S$ on $S$. Forming the
difference field, ${\bf B}_3={\bf B}_1-{\bf B}_2$, we
have
\begin{equation}\label{ee5.124}
\nabla\times{\bf B}_3={\bf 0}
\end{equation}
throughout $V$, and either ${\bf  B}_3={\bf 0}$ or ${\bf A}_3={\bf 0}$
on $S$.
Now, according to vector field theory,
\begin{equation}
\int_V\left[(\nabla\times{\bf U})^2 - {\bf U}\cdot\nabla\times\nabla\times{\bf U}\right]dV \equiv \oint_S {\bf U}\times(\nabla\times{\bf U})\cdot {\bf dS}.
\end{equation}
Setting ${\bf U}={\bf A}_3$, and using  ${\bf B}_3=\nabla\times{\bf A}_3$ and Equation~(\ref{ee5.124}), we obtain
\begin{equation}
\int_V B_3^{\,2}\,dV = \oint_S {\bf A}_3\times{\bf B}_3\cdot{\bf dS}.
\end{equation}
However, we know that either ${\bf B}_3$ or ${\bf A}_3$ is zero on $S$.
Hence, we get
\begin{equation}
\int_V B_3^{\,2}\,dV = 0.
\end{equation}
Since $B_3^{\,2}$ is positive definite, the only way in which the above
equation can be satisfied is if $B_3$ is zero throughout $V$. Hence,
${\bf B}_1={\bf B}_2$ throughout $V$, and the solution is  unique.

\section{One-Dimensional Solutions of Poisson's Equation}
So, how do we actually solve Poisson's equation, 
\begin{equation}
\frac{\partial^2\phi}{\partial x^2} +\frac{\partial^2\phi}{\partial y^2} +
\frac{\partial^2\phi}{\partial z^2} = -\frac{\rho(x,y,z)}{\epsilon_0},
\end{equation}
in practice? In general, the answer is that we use a computer. However, there
are a few situations, possessing a high degree of symmetry, where it is possible
to find analytic solutions. Let us discuss some of these situations. 

Suppose, first of all, that there is no variation of quantities in (say) the $y$- and $z$-directions.
In this case, Poisson's equation reduces to an ordinary differential equation in $x$, 
the solution of which is relatively straightforward. Consider, for instance, a {\em vacuum diode},
in which electrons are emitted from a hot cathode and accelerated towards an
anode, which is held at a large positive potential $V$ with respect to the
cathode. We can think of this as an essentially one-dimensional problem. Suppose
that the cathode is at $x=0$ and the anode at $x=d$. Poisson's equation
takes the form
\begin{equation}
\frac{d^2\phi}{dx^2} = - \frac{\rho(x)}{\epsilon_0},
\end{equation}
where $\phi(x)$ satisfies the boundary conditions $\phi(0)=0$ and $\phi(d)=V$. 
By energy conservation, an electron emitted from rest at the cathode
has an $x$-velocity $v(x)$ which satisfies
\begin{equation}
\frac{1}{2}\, m_e\,v^2(x) - e\,\phi(x) = 0.
\end{equation}
Here, $m_e$ and $-e$ are the mass and charge of an electron, respectively.
Finally, in a steady-state, the electric current $I$ (between the anode and
cathode) is
independent of $x$ (otherwise, charge will continually build up at some points). In
fact,
\begin{equation}
I = -\rho(x)\,v(x)\,A,
\end{equation}
where $A$ is the cross-sectional area of the diode. 
The previous three equations can be combined to give
\begin{equation}
\frac{d^2\phi}{dx^2} = \frac{I}{\epsilon_0\,A}\left(\frac{m_e}{2\,e}\right)^{1/2}\phi^{-1/2}.
\end{equation}
The solution of the above equation which satisfies the
boundary conditions is
\begin{equation}
\phi (x)= V\,\left(\frac{x}{d}\right)^{4/3},
\end{equation}
with
\begin{equation}
I = \frac{4}{9}\frac{\epsilon_0\,A}{d^2}\left(\frac{2\,e}{m_e}\right)^{1/2}
V^{3/2}.
\end{equation}
This relationship between the  current and the voltage in a vacuum diode is
called the {\em Child-Langmuir law}.

Let us now consider the solution of Poisson's equation in more than one
dimension.

\section{Method of Images}\label{s5.10}
Suppose that we have a point charge $q$ held a distance $d$ from an infinite,
grounded, conducting plate---see Figure~\ref{fimage}. Let the plate lie in the $x$-$y$ plane, and suppose that
the point charge is located at coordinates (0, 0, $d$). What is the
scalar potential generated in the region above the plate? This is not a simple question, because the point
charge induces surface charges on the plate, and we do not know beforehand how these charges
are distributed. 

Well, what do we know in this problem? We know that the conducting plate is an
 equipotential surface. In fact, the potential of the plate is zero, since it is grounded. 
We also know that the potential at infinity is zero (this is our usual boundary
condition for the scalar potential). Thus, we need to solve Poisson's equation
in the region $z>0$, with a single point charge $q$ at position (0, 0, $d$),
subject to the boundary conditions
\begin{equation}
\phi(x,y,0) = 0,
\end{equation}
and
\begin{equation}
\phi(x,y,z)\rightarrow 0\mbox{\hspace{1cm}as $x^2+y^2+z^2\rightarrow\infty$}.
\end{equation}
Let us forget about the real problem, for a
moment, and concentrate on a slightly different one. We refer to this as the
{\em analog problem}---see Figure~\ref{fimage}. In the analog problem, we have a charge $q$ located at
(0, 0, $d$) and a charge $-q$ located at (0, 0, -$d$), with no conductors present.
We can easily find the scalar potential for this problem, since we know where
all the charges are located. We get
\begin{equation}\label{e5.114}
\phi_{\rm analog} (x, y, z) = \frac{1}{4\pi\epsilon_0}
\left\{ \frac{q}{\sqrt{x^2+y^2+(z-d)^2}}- \frac{q}{\sqrt{x^2+y^2+
(z+d)^2}}\right\}.
\end{equation}
Note, however, that
\begin{equation}
\phi_{\rm analog}(x,y,0) = 0,
\end{equation}
 and
\begin{equation}
\phi_{\rm analog}(x,y,z)\rightarrow 0\mbox{\hspace{1cm}as $x^2+y^2+z^2\rightarrow\infty$}.
\end{equation}
Moreover, in the region $z>0$, $\phi_{\rm analog}$
 satisfies Poisson's equation
for  a point charge $q$ located at (0, 0, $d$). Thus, in this region, $\phi_{\rm analog}$
 is a solution
to the problem posed earlier. Now, the uniqueness theorem tells
us that there is only {\em one}\/ solution to Poisson's equation
which satisfies a given well-posed set of boundary conditions. So, 
$\phi_{\rm analog}$ must be the correct potential in the region $z>0$.
Of course, $\phi_{\rm analog}$ is completely wrong in the region $z<0$.
We know this because the grounded plate shields the region $z<0$ from the
point charge, so that $\phi=0$ in this region. Note that we are leaning pretty
heavily on the uniqueness theorem here! Without this theorem,
 it would be hard to convince
a skeptical person that $\phi = \phi_{\rm analog}$ is the correct solution
in the region $z>0$. 
\begin{figure}
\epsfysize=2.5in
\centerline{\epsffile{chapter5/fig5.9.eps}}
\caption{\em The method of images for a charge and a grounded conducting plane.}\label{fimage}
\end{figure}

Now that we have found the potential in the region $z>0$, we can easily work
out the distribution of charges induced on the conducting plate. We already
know that the relation between the electric 
field immediately above a conducting surface
and the density of charge on the surface is
\begin{equation}
E_\perp = \frac{\sigma}{\epsilon_0}.
\end{equation}
In this case,
\begin{equation}
E_\perp (x,y)= E_z(x,y,0_+) = - \frac{\partial \phi(x,y,0_+)}{\partial z}
= - \frac{\partial \phi_{\rm analog}(x,y,0_+)}{\partial z},
\end{equation}
so
\begin{equation}
\sigma(x,y) = - \epsilon_0 \frac{\partial\phi_{\rm analog}(x,y,0_+)}{\partial z}.
\end{equation}
Now, it follows from Equation~(\ref{e5.114}) that
\begin{equation}
\frac{\partial\phi_{\rm analog}}{\partial z} = \frac{q}{4\pi\epsilon_0} \left\{
\frac{-(z-d)}{[x^2+y^2+(z-d)^2]^{3/2}} +
\frac{(z+d)}{[x^2+y^2+(z+d)^2]^{3/2}}\right\},
\end{equation}
so
\begin{equation}
\sigma(x,y) = - \frac{q\,d}{2\pi\, (x^2+y^2+d^2)^{3/2}}.
\end{equation}
Clearly, the charge induced on the plate has the opposite sign to the point charge.
The charge density on the plate is also symmetric about the $z$-axis, and is largest
where the plate is closest to the point charge. The total charge induced on the
plate is
\begin{equation}
Q = \int_{x-y \,\rm plane} \sigma\,dS,
\end{equation}
which yields
\begin{equation}
Q = - \frac{q\,d}{2\pi} \int_0^\infty \frac{2\pi\, r\,dr}{(r^2+d^2)^{3/2}},
\end{equation}
where $r^2 = x^2+y^2$. Thus,
\begin{equation}
Q = - \frac{q\,d}{2} \int_0^\infty \frac{dk}{(k+d^2)^{3/2}}
= q\,d\left[ \frac{1}{(k+d^2)^{1/2}}\right]_0^\infty = - q.
\end{equation}
So, the total charge induced on the plate is equal and opposite to the point charge which induces it.

As we have just seen, our point charge induces charges of the opposite sign on the conducting plate.
This, presumably, gives rise to a force of attraction between the charge and the 
plate. What is this force? Well, since the potentials, and, hence, the electric
fields, in  the vicinity of the point charge are the same in the real and analog problems,
 the forces on this charge must be the same as well. In the analog problem,
there are two charges $\pm q$ a net distance $2\,d$ apart. The force on
the charge at position (0, 0, $d$) ({\em i.e.}, the real charge) is
\begin{equation}
{\bf f} = -  \frac{q^2}{16\pi\epsilon_0\,d^2} \,{\bf e}_z.
\end{equation}
Hence, this is also the force on the charge in the real problem.

What, finally, is the potential energy of the system. For the analog problem
this is simply
\begin{equation}
W_{\rm analog} = -  \frac{q^2}{8\pi\epsilon_0\,d}.
\end{equation}
Note that in the analog problem the fields on opposite sides of the conducting plate are mirror images
of one another. So are the charges (apart from the change
in sign). This is why the technique of replacing conducting surfaces by
imaginary charges is called the {\em method of images}. We know that the potential
 energy
of a set of charges is equivalent to the energy stored in the electric field.
Thus,
\begin{equation}
W = \frac{\epsilon_0}{2} \int_{\rm all\,space} E^{2}\,dV.
\end{equation}
Moreover, as we just mentioned, in the analog problem, the fields on either side of the $x$-$y$ plane are
mirror images of one another, so that $E^2(x, y, -z) = E^2(x, y, z)$. It follows
that
\begin{equation}
W_{\rm analog} = 2 \,\frac{\epsilon_0}{2} \int_{z>0} E^2_{\rm analog} \,dV.
\end{equation}
Now, in the real problem
\begin{equation}
{\bf E} =\left\{ 
\begin{array}{lcl}
{\bf E}_{\rm analog}&\mbox{\hspace{1cm}}&\mbox{for $z>0$}\\[0.5ex]
{\bf 0} &&\mbox{for $z <0$}
\end{array}
\right..
\end{equation}
So,
\begin{equation}
W = \frac{\epsilon_0}{2} \int_{z>0} E^2\,dV=
\frac{\epsilon_0}{2} \int_{z>0} E_{\rm analog}^2 \,dV = 
\frac{1}{2} \,W_{\rm analog},
\end{equation}
giving
\begin{equation}\label{e4.151x}
W = -  \frac{q^2}{16\pi\epsilon_0\,d}.
\end{equation}

There is another method by which we can obtain the above result. Suppose that
the charge is gradually moved towards the plate along the $z$-axis, starting from infinity,
until it reaches position (0, 0, $d$). How much work is required to
achieve this? We know that the force of attraction acting on  the charge is
\begin{equation}
f_z = - \frac{q^2}{16\pi\epsilon_0 \,z^2}.
\end{equation}
Thus, the work required to move this charge by $dz$ is
\begin{equation}
d W = - f_z\,dz=\frac{q^2}{16\pi\epsilon_0\,z^2}\,dz.
\end{equation}
So, the total work needed to move the charge from $z=\infty$ to $z= d$ is
\begin{equation}
W = \frac{1}{4\pi\epsilon_0}\int_{\infty}^d \frac{q^2}{4\,z^2}\,dz
= \frac{1}{4\pi\epsilon_0} \left[ - \frac{q^2}{4 \,z} \right]_{\infty}^d
= -  \frac{q^2}{16\pi\epsilon_0\,d}.
\end{equation}
Of course, this work is equivalent to the potential energy (\ref{e4.151x}),
and is, in turn, the same as the energy contained in the electric field.


As a second example of the method of images, consider a 
{\em grounded}\/ conducting sphere
of radius $a$ centered on the  origin. Suppose that a charge $q$ is
placed outside the sphere at $(b,\,0,\, 0)$, where $b>a$---see Figure~\ref{fimage1}. What is
the force of attraction between the sphere and the charge? In this case,
we proceed by considering an analog problem in which the sphere is replaced by an image charge $-q'$ placed
somewhere on the $x$-axis at $(c,\,0,\,0)$---see Figure~\ref{fimage1}. The electric potential throughout space in the
analog problem is simply
\begin{equation}
\phi(x,y,z) = \frac{q}{4\pi\epsilon_0}\,\frac{1}{[(x-b)^2+y^2+z^2]^{1/2}}-\frac{q'}{4\pi\epsilon_0}
\frac{1}{[(x-c)^2+y^2+z^2]^{1/2}}.
\end{equation}
Now, the image charge must be chosen so as to make the surface $\phi=0$ correspond to
the surface of the sphere. Setting the above expression to zero, and performing
a little algebra, we find that the $\phi=0$ surface corresponds to
\begin{equation}
x^2 + \frac{2\,(c-\lambda\,b)}{\lambda-1} \,x + y^2 + z^2 = \frac{c^2-\lambda\,b^2}{\lambda-1},
\end{equation}
where $\lambda=q'^{\,2}/q^2$. Of course, the surface of the sphere satisfies
\begin{equation}
x^2+y^2+z^2 = a^2.
\end{equation}
The above two equations can be made identical by setting $\lambda = c/b$ and $a^2=\lambda\,b^2$,
or
\begin{equation}
q' = \frac{a}{b}\,q,
\end{equation}
and
\begin{equation}
c = \frac{a^2}{b}.
\end{equation}
According to the uniqueness theorem, the potential in the analog problem is
now  identical with that in the real problem in the region outside the sphere. (Of course, in the real
problem, the potential inside the sphere is zero.)
Hence, the
force of attraction between the sphere and the original charge in the real problem
is the same as the force of attraction between the image charge and the real charge in the analog
problem. It follows that
\begin{equation}
f = \frac{q\,q'}{4\pi\epsilon_0\,(b-c)^2} = \frac{q^2}{4\pi\epsilon_0}\,\frac{a\,b}{(b^2-a^2)^2}.
\end{equation}
\begin{figure}
\epsfysize=2.25in
\centerline{\epsffile{chapter5/fig5.10.eps}}
\caption{\em The method of images for a charge and a grounded conducting sphere.}\label{fimage1}
\end{figure}

What is the total charge induced on the grounded conducting sphere? Well, according to
Gauss' law, the flux of the electric field out of a spherical Gaussian surface lying just
outside the surface of the conducting sphere is equal to the enclosed charge divided
by $\epsilon_0$. In the real problem, the enclosed charge is the net
charge induced on the surface of the sphere. In the analog problem, the
enclosed charge is simply $-q'$. However, the electric fields outside the
conducting sphere are identical in the real and analog problems. Hence, from Gauss'
law, the charge enclosed by the Gaussian surface must also be the same in both problems.
We thus conclude that the net charge induced on the surface of the conducting sphere is
\begin{equation}
-q' =-\frac{a}{b}\,q.
\end{equation}


As another example of the method of images, consider an insulated {\em uncharged}\/
conducting sphere of radius $a$, centered on the origin, in the presence of
a charge $q$ placed outside the sphere at $(b,\,0,\, 0)$, where $b>a$---see Figure~\ref{fimage2}. What is the force of attraction between the sphere
and the charge? Clearly, this new problem is  very similar to the one which we just discussed. The only difference is that the surface of
the sphere is now at some  {\em unknown}\/ fixed potential $V$, and also carries zero net charge. Note that if we add a second image charge $q''$, located at the
origin, to the analog problem pictured in Figure~\ref{fimage1} then the surface $r=a$ remains an equipotential surface. In fact, the potential of this surface becomes $V=q''/(4\pi\epsilon_0\,a)$. Moreover, the total charge enclosed
by the surface is $-q'+q''$. This, of course, is the net charge induced on the
surface of the sphere in the real problem. Hence, we can see that if $q''=q'=(a/b)\,q$ then zero net charge is induced on the surface of the sphere. Thus, our
modified analog problem is now a solution to the  problem under discussion, in the region outside the
sphere---see Figure~\ref{fimage2}. It follows that the surface of the sphere is at potential
\begin{equation}
V = \frac{q'}{4\pi\epsilon_0\,a} = \frac{q}{4\pi\epsilon_0\,b}.
\end{equation}
Moreover, the force of attraction between the sphere and the original charge
in the real problem
is the same as the force of attraction between the image charges and the
real  charge in the analog problem. Hence, the force is given by
\begin{equation}
f = \frac{q\,q'}{4\pi\epsilon_0\,(b-c)^2} - \frac{q\,q'}{4\pi\epsilon_0\,b^2}
= \frac{q^2}{4\pi\epsilon_0}\left(\frac{a}{b}\right)^3\,\frac{(2\,b^2-a^2)}{(b^2-a^2)^2}.
\end{equation} 
\begin{figure}
\epsfysize=2.25in
\centerline{\epsffile{chapter5/fig5.11.eps}}
\caption{\em The method of images for a charge and an uncharged conducting sphere.}\label{fimage2}
\end{figure}


As a final example of the method of images, consider two identical, infinitely
long,  conducting cylinders of radius $a$ which run parallel to the
$z$-axis, and lie a distance $2\,d$ apart. Suppose that one of the conductors
is held at potential $+V$, whilst the other is held at potential $-V$---see Figure~\ref{fimage3}. What is the capacitance per unit length of the cylinders?
\begin{figure}
\epsfysize=1.75in
\centerline{\epsffile{chapter5/fig5.12.eps}}
\caption{\em The method of images for two parallel cylindrical conductors.}\label{fimage3}
\end{figure}

Consider an analog problem in which the conducting cylinders are replaced by
two infinitely long charge lines, of charge per unit length $\pm\lambda$,
which run parallel to the $z$-axis, and lie a distance $2\,p$ apart. Now,
 the potential in the $x$-$y$ plane
generated by a charge line $\lambda$ running along the $z$-axis
is
\begin{equation}
\phi(x,y) = -\frac{\lambda}{2\pi\epsilon_0}\,\ln r,
\end{equation}
where $r=\sqrt{x^2+y^2}$ is the radial cylindrical polar coordinate.
The corresponding electric field is radial, and satisfies
\begin{equation}
E_r(r) = -\frac{\partial\phi}{\partial r} = \frac{\lambda}{2\pi\epsilon_0\,r}.
\end{equation}
Incidentally, it is easily demonstrated from Gauss' law that this is the correct electric field. Hence,  the potential generated by two  charge lines $\pm\lambda$ located in the $x$-$y$ plane at $(\pm p,\,0)$, respectively,  is
\begin{equation}\label{e4.1655x}
\phi(x,y) = \frac{\lambda}{4\pi\epsilon_0}\,\ln\left[\frac{(x+p)^2+y^2}{(x-p)^2+y^2}\right].
\end{equation}

Suppose that
\begin{equation}\label{e4.166x}
\frac{(x+p)^2+y^2}{(x-p)^2+y^2} = \alpha,
\end{equation}
where $\alpha$ is a constant.
It follows that
\begin{equation}
x^2 - 2\,p\,\frac{(\alpha+1)}{(\alpha-1)}\,x + p^2 + y^2 = 0.
\end{equation}
Completing the square, we obtain
\begin{equation}\label{e4.165x}
(x-d)^2 + y^2 = a^2,
\end{equation}
where
\begin{equation}
d = \frac{(\alpha+1)}{(\alpha-1)}\,p,
\end{equation}
and
\begin{equation}
a^2 = d^2 - p^2.
\end{equation}
Of course, Equation~(\ref{e4.165x}) is the equation of a cylindrical surface of radius $a$
centered on $(d,\,0)$. Moreover, it follows from Equations~(\ref{e4.1655x}) and (\ref{e4.166x}) that this surface lies at the
constant potential
\begin{equation}
V = \frac{\lambda}{4\pi\epsilon_0}\,\ln\alpha.
\end{equation}
Finally, it is easily demonstrated that the equipotential  $\phi=-V$ corresponds
to a cylindrical surface of radius $a$ centered on $(-d,\,0)$. Hence, we can make
the analog problem match the real problem in the region outside
the cylinders by choosing
\begin{equation}
\alpha = \frac{d+p}{d-p} = \frac{d+\sqrt{d^2-a^2}}{d-\sqrt{d^2-a^2}}.
\end{equation}
Thus, we obtain
\begin{equation}
V = \frac{\lambda}{4\pi\epsilon_0} \,\ln\left(\frac{d+\sqrt{d^2-a^2}}{d-\sqrt{d^2-a^2}}\right).
\end{equation}

Now, it follows from Gauss' law, and the fact that the electric fields in the
real and analog problems are identical outside the cylinders, that the
charge per unit length stored on the surfaces of the two cylinders is $\pm\lambda$. Moreover,
the voltage difference between the cylinders is $2\,V$. Hence, the
capacitance per unit length of the cylinders is $C=\lambda/(2\,V)$, yielding
\begin{equation}
C = 2\pi\epsilon_0\left/\ln\left(\frac{d+\sqrt{d^2-a^2}}{d-\sqrt{d^2-a^2}}\right)\right..
\end{equation}
This expression simplifies to give
\begin{equation}
C = \pi\epsilon_0\left/\ln\left(\frac{d}{a}+ \sqrt{\frac{d^2}{a^2}-1}\right)\right.,
\end{equation}
which can also be written
\begin{equation}
C = \frac{\pi\epsilon_0}{\cosh^{-1}(d/a)},
\end{equation}
since $\cosh^{-1}x \equiv \ln(x+\sqrt{x^2-1})$. 

\section{Complex Analysis}
Let us now investigate another  trick for  solving Poisson's equation (actually
it only solves Laplace's equation).
Unfortunately, this method only works in {\em two dimensions}.

The complex variable is conventionally written
\begin{equation}
z = x + {\rm i}\,y,
\end{equation}
where $x$ and $y$ are both real, and are identified with the corresponding
Cartesian coordinates.
(Incidentally, $z$ should not be confused with a $z$-coordinate: this is a strictly two-dimensional discussion.) We can write functions $F(z)$ of the complex variable just like
we would write functions of a real variable. For instance,
\begin{eqnarray}
F(z) &=& z^2,\\[0.5ex]
F(z)&=& \frac{1}{z}.
\end{eqnarray}
For a given function, $F(z)$, we can substitute  $z=x +{\rm i}\,y$ and write
\begin{equation}\label{e5.137}
F(z) = U(x, y) + {\rm i}\,V(x, y),
\end{equation}
where $U$ and $V$ are two {\em real}\/ two-dimensional functions. Thus, if
\begin{equation}
F(z) = z^2,
\end{equation}
then
\begin{equation}
F(x + {\rm i}\,y) = (x+{\rm i}\,y)^2 = (x^2-y^2) + 2\,{\rm i}\, x\,y,
\end{equation}
giving
\begin{eqnarray}
U(x, y) &=& x^2 - y^2,\\[0.5ex]
V(x, y) &=& 2\, x\,y.
\end{eqnarray}

We can define the derivative of a complex function in just the same manner as
we would  define the derivative of a real function: {\em i.e.}, 
\begin{equation}\label{e5.141}
\frac{dF}{dz} = ~_{\lim |\delta z|\rightarrow\infty}
\frac{F(z+\delta z) - F(z) }{\delta z}.
\end{equation}
However, we now have a slight problem. If $F(z)$ is a ``well-defined''
function (we shall leave it to the mathematicians to specify exactly what
being well-defined entails: suffice to say that most functions we can think
of are well-defined) then it should not matter from which direction in the complex
plane we approach $z$ when taking the limit in Equation~(\ref{e5.141}).
 There are, of course, many
different directions we could approach $z$ from, but if we look at a regular complex
function, $F(z) = z^2$  (say), then
\begin{equation}
\frac{dF}{dz} = 2 \,z
\end{equation}
is perfectly well-defined, and is, therefore,  completely independent of the details of
how the limit is taken in Equation~(\ref{e5.141}).

The fact that Equation~(\ref{e5.141})
 has to give the same result, no matter from which direction we approach
$z$, means that there are some restrictions on the forms of the functions $U$ and $V$ in
 Equation~(\ref{e5.137}).
Suppose that we approach $z$ along the real axis, so that $\delta z = \delta x$.
We obtain
\begin{eqnarray}
\frac{dF}{dz}& =& ~_{\lim |\delta x|\rightarrow 0}
\frac{U(x+\delta x, y) + {\rm i}\, V(x+\delta x, y) - U(x, y) - {\rm i}\,
V(x,y)}{\delta x} 
\nonumber\\[0.5ex]
&=& \frac{\partial U}{\partial x} + {\rm i}\, \frac{\partial V}
{\partial x}.
\end{eqnarray}
Suppose that we now approach $z$ along the imaginary axis, so that $\delta z
= {\rm i}\,\delta y$. We get
\begin{eqnarray}
\frac{dF}{dz} &=& ~_{\lim |\delta y|\rightarrow 0}
\frac{U(x, y+\delta y) + {\rm i}\, V(x, y+\delta y) - U(x, y) - {\rm i}\,
V(x,y)}{{\rm i}\,\delta y}  \nonumber\\[0.5ex]
&=&
-{\rm i}\,\frac{\partial U}{\partial y} + \frac{\partial V}
{\partial y}.
\end{eqnarray}
But, if $F(z)$ is a well-defined function then its derivative must also be 
well-defined,
which implies that the above two expressions are equivalent. This 
requires that
\begin{eqnarray}
\frac{\partial U}{\partial x} &=& \frac{\partial V}{\partial y},\\[0.5ex]
\frac{\partial V}{\partial x} &=& -\frac{\partial U}{\partial y}.
\end{eqnarray}
These are called the {\em Cauchy-Riemann relations}, and are, in fact, sufficient to ensure 
that  all possible ways of taking the limit (\ref{e5.141})  give the same answer.

So far, we have found that a general complex function $F(z)$ can be written
\begin{equation}
F(z) = U(x, y)+{\rm i}\,V(x, y),
\end{equation}
where $z= x + {\rm i}\,y$.  If $F(z)$ is well-defined then $U$ and $V$ {\em
automatically} satisfy the Cauchy-Riemann relations.
But, what has all of this got to do with electrostatics? Well, we can combine the
two Cauchy-Riemann relations to give
\begin{equation}
\frac{\partial^2 U}{\partial x^2} = \frac{\partial}{\partial x}
\frac{\partial V}{\partial y}= \frac{\partial}{\partial y}
 \frac{\partial V}{\partial x}= - \frac{\partial}{\partial y} \frac{\partial U}
{\partial y},
\end{equation}
and
\begin{equation}
\frac{\partial^2 V}{\partial x^2} = -\frac{\partial}{\partial x}
\frac{\partial U}{\partial y}= -\frac{\partial}{\partial y}
 \frac{\partial U}{\partial x}= - \frac{\partial}{\partial y} \frac{\partial V}
{\partial y},
\end{equation}
which reduce to
\begin{eqnarray}
\frac{\partial^2 U}{\partial x^2} + \frac{\partial^2 U}{\partial y^2} &=& 0,
\\[0.5ex]
\frac{\partial^2 V}{\partial x^2} + \frac{\partial^2 V}{\partial y^2} &=& 0.
\end{eqnarray}
Thus, both $U$ and $V$ {\em automatically} satisfy Laplace's equation in
two dimensions: {\em i.e.}, both $U$ and $V$ are possible two-dimensional scalar potentials
in free space. 

Consider the two-dimensional gradients of $U$ and $V$:
\begin{eqnarray}
\nabla U &=& \left( \frac{\partial U}{\partial x}, \frac{\partial U}{\partial y}
\right),\\[0.5ex]
\nabla V &=& \left( \frac{\partial V}{\partial x}, \frac{\partial V}{\partial y}
\right).
\end{eqnarray}
Now
\begin{equation}
\nabla U \cdot \nabla V = \frac{\partial U}{\partial x}\frac{\partial V}{\partial x}
+\frac{\partial U}{\partial y} \frac{\partial V}{\partial y}.
\end{equation}
However, it follows from the Cauchy-Riemann relations that
\begin{equation}
\nabla U \cdot \nabla V = \frac{\partial V}{\partial y}\frac{\partial V}{\partial x}
-\frac{\partial V}{\partial x} \frac{\partial V}{\partial y}=0.
\end{equation}
Thus, the contours of $U$ are everywhere {\em perpendicular} to the contours of $V$. 
It follows that if $U$ maps out the contours of some free-space scalar potential
then the contours of $V$ indicate the directions of the associated electric field-lines,
and {\em vice versa}.


For every well-defined complex function, we get two sets
of free-space potentials, and the associated electric field-lines. For example,
consider the function  $F(z) = z^2$, for which
\begin{eqnarray}
U(x,y)&=& x^2 - y^2,\\[0.5ex]
V(x,y) &=& 2\,x\,y.
\end{eqnarray}
These are, in fact, the equations of two sets of orthogonal hyperboloids---see Figure~\ref{ff5}.
So, $U(x,y)$ (the solid lines in Figure~\ref{ff5}) 
might represent the contours of some scalar potential, and $V(x,y)$
(the dashed lines in Figure~\ref{ff5})
the associated electric field-lines, or {\em vice versa}. But, how could we
actually generate a hyperboloidal potential? This is easy. Consider the contours
of $U$ at level $\pm 1$. These could represent the surfaces of four hyperboloid
conductors maintained at potentials $\pm \cal V$, respectively. The scalar potential in the
region between these conductors is given by ${\cal V}\, U(x, y)$, and the associated
electric field-lines follow the contours of $V(x, y)$. 
Note that 
\begin{equation}
E_x = - \frac{\partial\phi}{\partial x} =- {\cal V} \,\frac{\partial U}{\partial x}
= -2 \,{\cal V}\, x
\end{equation}
Thus, the $x$-component of the electric 
field is directly proportional to the distance
from the $x$-axis. Likewise,  the  $y$-component of the field is directly proportional
to the distance from the $y$-axis. This property
can be exploited to make devices (called quadrupole electrostatic lenses) which
are useful for focusing charged particle beams.
\begin{figure}
\epsfysize=3in
\centerline{\epsffile{chapter5/fig5.13.eps}}
\caption{\em Equally spaced contours of the real (solid lines) and imaginary (dashed-lines) parts of $F(z)=z^2$ plotted in the complex plane. }\label{ff5}
\end{figure}

As a second example, consider the complex function
\begin{equation}
F(z) = z - \frac{c^2}{z},
\end{equation}
where $c$ is real and positive. Writing $F(z)=U(x,y) + {\rm i}\,V(x,y)$, we find that
\begin{eqnarray}
U(x,y) &=& x - \frac{c^2\,x}{x^2+y^2},\\[0.5ex]
V(x,y) &=& y + \frac{c^2\,y}{x^2+y^2}.
\end{eqnarray}
Far from the origin, $U\rightarrow x$, which is the potential of a
uniform electric field, of unit amplitude, pointing in the $-x$-direction. 
Moreover, the locus
of $U=0$ is  $x=0$, and
\begin{equation}
x^2+y^2 = c^2,
\end{equation}
which corresponds to a circle of radius $c$ centered on the origin. Hence,
we conclude that the potential
\begin{equation}
\phi(x,y) = - E_0\,U(x,y) = -E_0\,x + E_0\,c^2\,\frac{x}{x^2+y^2}
\end{equation}
corresponds to that outside a grounded, infinitely long,  conducting cylinder of radius $c$, co-axial with the $z$-axis, which is placed in a uniform $x$-directed electric field of
magnitude $E_0$. The corresponding electric field-lines run along
contours of $V$---see Figure~\ref{ff6}.
Of course, the potential inside the cylinder ({\em i.e.},
$x^2+y^2< c^2$) is zero. Defining standard cylindrical polar coordinates,
$r=\sqrt{x^2+y^2}$ and $\theta =\tan^{-1}(y/x)$, the
potential becomes
\begin{equation}
\phi(r,\theta) = - E_0\left(r\,\cos\theta - \frac{c^2\,\cos\theta}{r}\right).
\end{equation}
Hence, the induced charge density on the surface of
the cylinder is simply
\begin{equation}
\sigma(\theta) = \epsilon_0\,E_r(c,\theta) = - \epsilon_0\,\frac{\partial\phi(c,\theta)}{\partial r} =
2\,\epsilon_0\,E_0\,\cos\theta.
\end{equation}
 Note that zero
net charge is induced on the surface. This implies that if the cylinder were
insulated and uncharged, rather than being grounded, then the solution would not change.
\begin{figure}
\epsfysize=3in
\centerline{\epsffile{chapter5/fig5.14.eps}}
\caption{\em Equally spaced contours of the real (solid lines) and imaginary (dashed-lines) parts of $F(z)=z - 1/z$ plotted in the complex plane for $|z|>1$. }\label{ff6}
\end{figure}

As a final example, consider the complex function
\begin{equation}
F(z) = z^{1/2}.
\end{equation}
Note that we need a branch-cut in the complex plane in order to make this function single-valued. Suppose that the cut is at ${\rm arg}(z)=\pi$,
so that $-\pi \leq {\rm arg}(z) \leq \pi$. Adopting standard cylindrical polar
coordinates, it is easily seen that
\begin{eqnarray}
U(r,\theta) &=& r^{1/2}\,\cos(\theta/2),\\[0.5ex]
V(r,\theta)&=& r^{1/2}\,\sin(\theta/2),
\end{eqnarray}
where $-\pi\leq \theta \leq \pi$. Now, the locus of $U=0$ corresponds
to $\theta=\pm \pi$. Hence, $U(r,\theta)$ represents the electric potential
in the immediate vicinity of an earthed semi-infinite conducting plate occupying the negative
$x$-axis. The corresponding electric field-lines run along 
contours of $V(r,\theta)$---see Figure~\ref{fff6}. The surface charge
density on the plate is easily obtained from
\begin{eqnarray}
\sigma(r) &=& \epsilon_0\,\left[E_\theta(r,-\pi)-E_\theta(r,\pi)\right]\nonumber\\[0.5ex]
&=& -\frac{\epsilon_0}{r}\left[\frac{\partial U(r,-\pi)}{\partial\theta}-\frac{\partial U(r,\pi)}{\partial\theta}\right]\nonumber\\[0.5ex]
&=& - \epsilon_0\,r^{-1/2}.
\end{eqnarray}
\begin{figure}
\epsfysize=3in
\centerline{\epsffile{chapter5/fig5.15.eps}}
\caption{\em Equally spaced contours of the real (solid lines) and imaginary (dashed-lines) parts of $F(z)=z^{1/2}$ plotted in the complex plane for $-\pi\leq {\rm arg}(z)\leq \pi$. }\label{fff6}
\end{figure}


\section{Separation of Variables}
The method of images and complex analysis are two rather elegant techniques
for solving Poisson's equation. Unfortunately, they both  have an
extremely limited range of application. The next technique which we shall discuss---namely, the {\em separation of variables}---is somewhat  messy,
but possess a  far wider range of application. Let us start by examining a well-known
example.

Consider two semi-infinite, grounded, conducting plates lying parallel to the
$x$-$z$ plane, one at $y=0$, and the other at $y=\pi$---see Figure~\ref{f44}. Suppose that the left boundary of the region between the plates, located at
$x=0$, is closed off by an infinite strip which is insulated from the two plates,
and maintained at a specified potential $\phi_0(y)$. What is the
potential in the region between the plates?
\begin{figure}
\epsfysize=1.25in
\centerline{\epsffile{chapter5/fig5.16.eps}}
\caption{\em Two semi-infinite grounded conducting plates.}\label{f44}
\end{figure}

First of all, let us assume that the potential is $z$-independent, since everything else
in the problem possesses this symmetry. This reduces the problem to two dimensions. 
Poisson's equation is written
\begin{equation}\label{e5.156}
\frac{\partial^2\phi}{\partial x^2} + \frac{\partial^2\phi}{\partial y^2} = 0
\end{equation}
in the vacuum region between the conductors. The boundary conditions are
\begin{eqnarray}
\phi(x, 0) &=& 0,\label{e5.157a}\\[0.5ex]
\phi(x, \pi)&=& 0\label{e5.157b}
\end{eqnarray}
for $x>0$, since the two plates are earthed, plus
\begin{equation}\label{e5.159}
\phi(0, y) = \phi_0(y)
\end{equation}
for $0\leq y \leq \pi$, and
\begin{equation}\label{e5.160}
\phi(x, y)\rightarrow 0\mbox{\hspace{1cm}as $x\rightarrow\infty$.}
\end{equation}
The latter boundary condition is our usual one for the
scalar potential at infinity.

The central assumption in the separation of variables method is that a 
multi-dimensional potential can be written as the product of one-dimensional
potentials. Hence, in the present case, we would write
\begin{equation}\label{e5.160a}
\phi(x, y) = X(x)\, Y(y).
\end{equation}
The above solution is obviously a very special one, and
is, therefore,  only likely to satisfy a very small subset of possible
boundary conditions. However, it turns out that by adding together
lots of different solutions of this form we can match to general boundary
conditions. 

Substituting (\ref{e5.160a})  into (\ref{e5.156}), we obtain
\begin{equation}
Y \,\frac{d^2 X}{d x^2} + X\,\frac{d^2 Y}{d y^2} = 0.
\end{equation}
Let us now separate the variables: {\em i.e.}, let us collect all of the
$x$-dependent terms on one side of the equation, and all of the $y$-dependent
terms on the other side. Hence,
\begin{equation}
\frac{1}{X}\frac{d^2 X}{d x^2} = - \frac{1}{Y} \frac{d^2 Y}{d y^2}.
\end{equation}
This equation has the form
\begin{equation}\label{e5.164}
f(x) = g(y),
\end{equation}
where $f$ and $g$ are general functions. The only way in which the above equation
can be satisfied, for general $x$ and $y$, is if both sides are equal to the
same constant. Thus,
\begin{equation}\label{e5.164a}
\frac{1}{X}\frac{d^2 X}{d x^2} = k^2 =- \frac{1}{Y} \frac{d^2 Y}{d y^2}.
\end{equation}
The reason why we write $k^2$, rather than $-k^2$, will become apparent later on.
Equation~(\ref{e5.164a}) separates into two ordinary differential equations:
\begin{eqnarray}
\frac{d^2 X}{d x^2} =~ k^2\,X,\\[0.5ex]
\frac{d^2 Y}{d y^2} = - k^2\,Y.
\end{eqnarray}
We know the general solution to these equations:
\begin{eqnarray}
X& = &A \exp(k\,x) + B \exp(-k\,x),\\[0.5ex]
Y &=& C \sin (k\,y) + D \cos (k\,y),
\end{eqnarray}
giving
\begin{equation}
\phi(x,y) = [\,A \exp(k\,x) + B \exp(-k\,x)\,]\,[C \sin (k\,y) + D \cos (k\,y)].
\end{equation}
Here, $A$, $B$, $C$, and $D$ are arbitrary constants. The boundary condition
(\ref{e5.160}) is automatically satisfied if $A=0$ and $k>0$. 
Note that the choice $k^2$, instead of
$-k^2$, in Equation~(\ref{e5.164a}) facilitates this by making $\phi$  either grow or decay
monotonically in  the $x$-direction instead of
oscillating.  The boundary condition (\ref{e5.157a})
 is automatically
satisfied if $D=0$. The boundary condition (\ref{e5.157b}) is satisfied provided that
\begin{equation}
\sin (k \,\pi) = 0,
\end{equation}
which implies that  $k$ is a positive integer, $n$ (say). So,
our solution reduces to
\begin{equation}\label{e5.170}
\phi(x, y) = C\, \exp(-n\,x)\,\sin (n\,y),
\end{equation}
where $B$ has been absorbed into $C$. Note that this solution is only able
to satisfy the final boundary condition (\ref{e5.159}) provided that $\phi_0(y)$ is
proportional to $\sin (n\,y)$. Thus, at first sight, it would appear that the method
of separation of variables only works for a very special subset of 
boundary conditions. However, this is not the case.

Now comes the clever bit! Since Poisson's equation is {\em linear}, any
linear combination of solutions is also a solution. We can therefore form a
more general solution  than (\ref{e5.170}) by adding together lots of solutions involving
different values of $n$. Thus,
\begin{equation}
\phi(x, y) = \sum_{n=1}^\infty C_n \exp(-n\, x) \sin (n\,y),
\end{equation}
where the $C_n$ are constants. 
This solution automatically satisfies the boundary conditions (\ref{e5.157a}), (\ref{e5.157b}) and
(\ref{e5.160}).  The
final boundary condition (\ref{e5.159}) reduces to
\begin{equation}\label{e5.171}
\phi(0, y) = \sum_{n=1}^\infty C_n \sin (n\,y) = \phi_0(y).
\end{equation}

But, what choice of the $C_n$ fits an arbitrary function
$\phi_0(y)$? To answer this question, we can make use of two very useful properties
of the functions $\sin (n\,y)$. Namely, that they are mutually {\em orthogonal}, and
form a {\em complete set}. The orthogonality property of these functions manifests
itself through the relation
\begin{equation}
\int_0^\pi \sin( n\,y) \,\sin (n'\,y)\,\,dy = \frac{\pi}{2}\, \delta_{n n'},
\end{equation}
where 
$\delta_{nn'}$---which is equal to 1 if $n=n'$, and 0, otherwise---is called a {\em Kroenecker delta function}. 
The completeness property of sine functions means that any general function
$\phi_0(y)$ 
can always be adequately 
represented as a weighted sum of sine functions with various different
$n$ values. Multiplying both sides of Equation~(\ref{e5.171}) by $\sin (n'\,y)$, and integrating
over $y$, we obtain
\begin{equation}
\sum_{n=1}^\infty C_n \int_0^\pi \sin (n\,y)\,\,\sin (n'\,y)\,\,dy = 
\int_0^\pi \phi_0(y)\,\sin( n' \,y)\, \,dy.
\end{equation}
The orthogonality relation yields
\begin{equation}
\frac{\pi}{2} \sum_{n=1}^\infty C_n \,\delta_{n n'} = \frac{\pi}{2}\, C_{n'} =
\int_0^\pi \phi_0(y)\,\sin (n' \,y)\,\,dy,
\end{equation}
so
\begin{equation}
C_n = \frac{2}{\pi} \int_0^\pi \phi_0(y)\,\sin (n\,y)\,dy.
\end{equation}
Thus, we now have a general solution to the problem for any driving potential
$\phi_0(y)$. 

If the potential $\phi_0(y)$ is constant then
\begin{equation}
C_n = \frac{2\,\phi_0}{\pi} \int_0^\pi \sin (n\,y)\,\,dy
= \frac{2\,\phi_0}{n \,\pi}\, [1- \cos (n\,\pi) ],
\end{equation}
giving
\begin{equation}
C_n = 0
\end{equation}
for even $n$, and
\begin{equation}
C_n = \frac{4\,\phi_0}{n \,\pi}
\end{equation}
for odd $n$. Thus,
\begin{equation}\label{epotn}
\phi(x, y) = \frac{4\,\phi_0}{\pi}\sum_{n=1, 3, 5,\cdots}\frac{\exp(-n\,x)\sin (n\,y)}{n}.
\end{equation}
This potential is plotted in Figure~\ref{fex5}.
\begin{figure}
\epsfysize=3in
\centerline{\epsffile{chapter5/fig5.17.eps}}
\caption{\em Equally spaced contours of the potential specified in Equation~(\ref{epotn}). Only the first 50 terms in the series are retained.}\label{fex5}
\end{figure}


In the above problem, we wrote the potential as the product of one-dimensional
functions. Some of these functions grew and decayed monotonically ({\em i.e.}, the
exponential functions), and the others oscillated ({\em i.e.}, the sinusoidal functions).
The success of the separation of variables method depends crucially on the orthogonality and completeness
of the oscillatory functions. A set of functions $f_n(x)$ is {\em orthogonal}\/
if the integral of the product of two different members of the set over some
range is always zero: {\em i.e.}, 
\begin{equation}
\int_a^b f_n(x) \,f_m(x)\, dx = 0,
\end{equation}
for $n\neq m$. A set of functions is {\em complete}\/ if any other function can be
expanded as a weighted sum of them. It turns out that the scheme set out
above can be generalized to more complicated geometries. 
For instance,  in axisymmetric spherical geometry, the monotonic
functions are power law functions of the radial variable, and the oscillatory functions
are so-called Legendre polynomials involving the cosine of the polar angle $\theta$. The latter functions are both mutually orthogonal and form a
complete set. There are also cylindrical, ellipsoidal, hyperbolic, toroidal, {\em etc.}\
coordinates. In all cases, the associated oscillating functions are mutually
orthogonal
and form a complete set. This implies that the separation of variables method
is of quite general applicability. 

Finally, as a very simple example of the solution of Poisson's equation in spherical
geometry, let us consider the case of a grounded conducting sphere of radius $a$, centered on the
origin, and placed in a uniform $z$-directed electric field of magnitude $E_0$.
The scalar potential $\phi$ satisfies $\nabla^2\phi=0$ for $r\geq a$, with
the boundary conditions $\phi\rightarrow -E_0\,r\,\cos\theta$ (giving ${\bf E}\rightarrow E_0\,{\bf e}_z$) as
$r\rightarrow\infty$, and $\phi=0$ at $r=a$. Here, $r$ and $\theta$ are spherical polar coordinates. Let us, first of all, assume that $\phi$ is independent of
the azimuthal angle, since the boundary conditions possess this
symmetry. Hence, $\phi  =\phi(r, \theta)$. 
Next, let us try
the simplified separable solution
\begin{equation}
\phi(r, \theta) = r^{m}\cos\theta.
\end{equation}
It is easily demonstrated that the above solution only satisfies $\nabla^2\phi=0$
provided $m=1$ or $m=-2$. Thus, the most general solution of $\nabla^2\phi$
which satisfies the boundary condition at $r\rightarrow\infty$ is
\begin{equation}\label{e4.243x}
\phi(r,\theta) = - E_0\,r\,\cos\theta + \alpha\,r^{-2}\,\cos\theta.
\end{equation}
The boundary condition at $r=a$ is satisfied provided 
\begin{equation}\label{e4.244x}
\alpha = E_0\,a^3.
\end{equation}
Hence, the potential takes the form
\begin{equation}
\phi(r,\theta) = - E_0\left(r - \frac{a^3}{r^2}\right)\cos\theta.
\end{equation}
Of course, $\phi=0$ inside the sphere ({\em i.e.}, $r<a$). This potential
is plotted in Figure~\ref{fex6a} (for $a=1$).
The charge sheet
density induced on the surface of the sphere is given by
\begin{equation}
\sigma (\theta)= \epsilon_0\,E_r(a,\theta) = -\epsilon_0\,\frac{\partial \phi(a,\theta)}{\partial r} = 3\,\epsilon_0\,E_0\,\cos\theta.
\end{equation}
Note that zero net charge is induced on the surface of the sphere. This
implies that the solution would be unchanged were the sphere insulated and
uncharged, rather than grounded. Finally, it follows from Equations~(\ref{e4.243x}), (\ref{e4.244x}), and Exercise~2.4 that the electric field outside the
sphere consists of the original uniform field plus the field of an electric dipole
of moment
\begin{equation}
{\bf p} = 4\pi\,a^3\,\epsilon_0\,E_0\,{\bf e}_z.
\end{equation}
This is, of course, the dipole moment due to the charge
separation induced on the surface of the sphere by the external field.
\begin{figure}
\epsfysize=3in
\centerline{\epsffile{chapter5/fig5.18.eps}}
\caption{\em Equally spaced contours of the axisymmetric potential $\phi(r,\theta)=(r-1/r^2)\,\cos\theta$ plotted in the $x$-$z$ plane for  $r> 1$.}\label{fex6a}
\end{figure}

{\small
\section{Exercises}
\renewcommand{\theenumi}{5.\arabic{enumi}}
\begin{enumerate}
\item Eight identical point charges of magnitude $q$ are placed at the vertices of a cube of dimension $a$. What is the electrostatic potential
energy of this configuration of charges (excluding the self-energies of
the charges)? Suppose that four of the charges are replaced by
charges of magnitude $-q$ in such a manner that all of the nearest neighbours
of a given charge are charges of the opposite sign. What now
is the electrostatic potential energy of the configuration?
\item Find the electric field generated by a thin uniform spherical shell
of charge $Q$ and radius $a$. Calculate the electrostatic potential
energy of this charge distribution by integrating the energy density of the
electric field over all space. Verify that the electrostatic energy is
also given by
$$
W=\frac{1}{2}\int_S \sigma\,\phi\,dS,
$$
where $\phi$ is the scalar potential, $\sigma$ the surface charge density,
and the integral is taken over all surface charge distributions.
\item Suppose that a stationary charge distribution $\rho_1({\bf r})$
generates the scalar potential field $\phi_1({\bf r})$, and that an alternative charge
distribution $\rho_2({\bf r})$ generates the  potential  $\phi_2({\bf r})$. Here, both charge distributions are assumed to be sufficiently localized that the
potential fields they generate go to zero at large distances. Prove {\em Green's
reciprocity theorem}:
$$
\int\phi_1\,\rho_2\,dV = \int \phi_2\,\rho_1\,dV,
$$
where the volume integral is over all space.
Hint: Use Maxwell's equations and the divergence theorem.
\item Two  grounded infinite parallel conducting plates are separated by a
perpendicular distance $d$. A point charge $q$ is placed between the plates.
Demonstrate  that the total charge induced on one of the plates
is $(-q)$ times the fractional perpendicular distance of the point charge
from the other plate. Hint: Use Green's reciprocity theorem.
\item Two  grounded concentric thin spherical conducting shells
have radii $a$ and $b$, where $b>a$.  A point charge $q$ is placed between the shells at radius $r$ (where $a< r< b$).  Find the total
charge induced on each shell. Hint: Use Green's reciprocity theorem.
\item Consider two insulated conductors, labeled 1 and 2. Let $\phi_1$ be the
potential of the first conductor when it is uncharged and the second conductor holds a
charge $Q$. Likewise, let $\phi_2$ be the potential of the second conductor
when it is uncharged and the first conductor holds a charge $Q$. Use Green's
reciprocity theorem to demonstrate that
$$
\phi_1 = \phi_2.
$$
\item Consider two  insulated spherical conductors. Let the
first have radius $a$. Let the second be sufficiently small that it
can effectively be treated as a point charge, and let it also be located a distance $b>a$ from
the center of the first. Suppose that the first conductor is uncharged, and that the second carries a charge $q$. What is the potential of the first conductor?
Hint: Consider the result proved in the previous exercise.
\item Consider a set of $N$ conductors distributed in a vacuum. Suppose that
the $i$th conductor carries the charge $Q_i$ and is at the scalar potential
$\phi_i$. It follows from the linearity of Maxwell's equations and Ohm's law
that a linear relationship exists between the potentials and the charges:
{\em i.e.},
$$
\phi_i = \sum_{j=1,N} p_{ij}\,Q_j.
$$
Here, the $p_{ij}$ are termed the {\em coefficients of potential}. Demonstrate that $p_{ij} = p_{ji}$ for all $i,j$. Hint: Consider the result
proved in Exercise~4.6. Show that the total electrostatic potential energy of
the charged conductors is
$$
W = \frac{1}{2}\sum_{i,j=1,N}p_{ij}\,Q_i\,Q_j.
$$
\item Find the coefficients of potential for two thin concentric spherical conducting shells of radius $r_1$ and $r_2$, where $r_2> r_1$. Suppose that $Q_1$ and
$Q_2$ are the charges on the inner and outer conductors, respectively.
Calculate the electrostatic potential energy of the system.
\item Returning to the problem discussed in Exercise~4.7, suppose that the
first conductor is now earthed, rather than being insulated and uncharged. What
charge is induced on the first conductor by the charge on the second? Hint:
Consider the concept of coefficient of potential. 
\item Consider two separate conductors, the first of which is
insulated and uncharged, and the second of which is earthed. Prove that the
first conductor is also at zero ({\em i.e.}, earth) potential. Hint: Consider the
previous hint.
\item Consider a flat annular plate ({\em e.g.}, a washer) of uniform thickness $\delta$, inner
radius $a$, and outer radius $b$. Let the plate be fabricated from
metal of uniform resistivity $\eta$. Suppose that an electrical current $I$ is
fed into the plate symmetrically at its inner radius, and extracted symmetrically at its outer radius. What is the resistance of the plate?
What is the rate of ohmic heating of the plate?
\item Consider an infinite uniform network of identical resistors of resistance $R$.
Let four resistors come together at each junction of the network ({\em i.e.}, let the network have a square lattice).
Suppose that current is fed into a given junction and extracted from a
nearest neighbour junction. What is the effective resistance of the network?
\item According to the uniqueness theorem, Poisson's equation $\nabla^2\phi=-\rho/\epsilon_0$ can only have one solution if $\rho$ is given
in some volume $V$, and $\phi$ is specified on the bounding surface $S$.
Demonstrate that two solutions can differ by, at most, a constant if
the normal derivative of the potential, rather than the potential itself,
is specified on the bounding surface.
\item Consider  a point charge $q$ which  is placed inside a thin grounded spherical conducting shell 
of radius $a$ a distance $r$ from its center. Use the method of images
to find the surface charge density induced on the inside of the shell. What is
the net charge induced on the inside of the shell? What is the magnitude and direction of
the force of attraction between the the charge and the shell? What electric field is induced outside the shell. How would
these results be modified if the sphere were (a) uncharged and insulated, or
(b) maintained at the constant potential $V$?
\item Using the method of images, show that the force of attraction, or repulsion, between a
point charge $q$ and an insulated conducting sphere of radius $a$ carrying a charge $Q$ is
$$
f = \frac{q}{4\pi\epsilon_0}\left[\frac{Q + (a/d)\,q}{d^2} - \frac{a\,q}{d\,(d-a^2/d^2)}\right],
$$
where $d>a$ is the distance of the charge from the center of the sphere.
Demonstrate that when $Q$ and $q$ are of the same sign then the force
is attractive provided that
$$
\frac{Q}{d} < \frac{a\,d^2}{(d^2-a^2)^2}-\frac{a}{d}.
$$
\item An infinitely long conducting cylinder carries a charge per unit length
$\lambda$ and runs parallel to an infinite grounded conducting plane. Let the
radius of the cylinder be $a$, and let the perpendicular distance between
the cylinder's axis and the plane be $d$ (where $d>a$). What is the
force of attraction per unit length between the cylinder and the plane? 
What is the charge per unit length induced on the plane? Use the method of images.
\item A point charge $q$ is located between two parallel
grounded infinite conducting planes separated by a perpendicular distance $d$. Suppose that the perpendicular distance of the charge from one of
the planes is $x$. Find the locations of the infinite number of image charges, and, hence, express the force exerted on the charge as an infinite series.
Plot the magnitude of this force as a function of $x/d$. Find expressions
for the net charges induced on the two planes. Plot these expressions as
functions of $x/d$. 
\item Two semi-infinite grounded conducting planes meet at right-angles.
A charge $q$ is located a perpendicular distance $a$ from one, and
$b$ from another. Use the method of images to find the magnitude and direction of the
force of attraction between the planes and the charge. What is the
net charge induced on the planes?
\item Two semi-infinite grounded conducting planes meet at sixty degrees.
A charge $q$ is located the same perpendicular distance $a$ from both. Use the method of images to find the magnitude and direction of the
force of attraction between the planes and the charge. What is the
net charge induced on the planes?
\item Find the function $F(z)$, where $z$ is the complex variable,
whose real part can be interpreted as the scalar potential
associated with (a) a uniform electric field of magnitude $E_0$ directed
along the $x$-axis, (b) a uniform electric field of magnitude $E_0$
directed along the $y$-axis, and (c) a  line charge of charge per unit length
$\lambda$ located at the origin.
\item Two semi-infinite conducting plates  meet at $90^\circ$, and are both held
at the constant potential $V$.  Use complex analysis to find the  variation
of the surface charge density with  perpendicular distance
from the vertex on both sides of the plates.
\item Two semi-infinite conducting plates  meet at $60^\circ$, and are both held
at the constant potential $V$.  Use complex analysis to find the variation
of the surface charge density with perpendicular distance
from the vertex on both sides of the plates.

\item Consider the complex function $F(z)$ defined implicitly by the
equation
$$
{\rm i}\,z = {\rm i}\,F(z) + {\rm e}^{\,{\rm i}\,F(z)}.
$$
Suppose that the real part of this function is interpreted as an electric
potential. Plot the contours of this potential. What problem in electrostatics does this potential best describe?
\item Consider an empty cubic box of dimension $a$ with conducting walls.
Two opposite walls are held at the constant potential $V$, whilst the
other walls are earthed. Find an expression for the electric potential
inside the box. (Assume that the box is centered on the origin, that the
walls are all normal to one of the Cartesian axes, and that the non-grounded
walls are normal to the $x$-axis.) Suppose that the two walls normal
to the $x$-axis are held at potentials $\pm V$. What now is the potential
inside the box? Use separation of variables.
\end{enumerate}
\renewcommand{\theenumi}{arabic{enumi}}
}
