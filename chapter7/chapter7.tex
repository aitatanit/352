\chapter{Magnetic Induction}\label{ind}
\section{Introduction}
In this chapter, we shall use Maxwell's equations to investigate
magnetic induction and related phenomena.

\section{Inductance}
We have already learned about the concepts of voltage, resistance, and capacitance. Let us now investigate the concept of {\em inductance}.
Electrical engineers like to reduce all pieces of electrical circuitary to an
{\em equivalent circuit}\/ consisting  of pure voltage sources,
pure inductors, pure capacitors, and pure resistors. Hence, once we understand inductors, we shall
be ready to apply the laws of electromagnetism to general electrical circuits. 

Consider two stationary loops of wire, labeled 1 and 2---see Figure~\ref{floop}. Let us run a steady current
$I_1$ around the first loop to produce a magnetic field ${\bf B}_1$. Some of the field-lines
of ${\bf B}_1$ will pass through the second loop. Let ${\mit\Phi}_2$ be the flux
of ${\bf B}_1$ through loop 2,
\begin{equation}
{\mit\Phi}_2 = \int_{\rm loop\, 2} {\bf B}_1\cdot d{\bf S}_2,
\end{equation}
where $d{\bf S}_2$ is a surface element of loop 2. 
This flux is generally quite difficult to calculate exactly (unless the two loops
have a particularly simple geometry). However, we can infer from the Biot-Savart law,
\begin{equation}
{\bf B}_1({\bf r}) = \frac{\mu_0 \,I_1}{4\pi} \oint_{\rm loop \,1} 
\frac{ d{\bf l}_1 \times ({\bf r} - {\bf r}_1)}{|{\bf r} - {\bf r}_1|^3},
\end{equation}
that the magnitude of ${\bf B}_1$ is proportional to the current $I_1$.
This is
ultimately a consequence of the linearity of Maxwell's equations. 
Here, $d{\bf l}_1$ is a line element of loop 1 located at position
vector ${\bf r}_1$.  
It follows that
the flux ${\mit\Phi}_2$ must also be proportional to $I_1$. Thus, we can write
\begin{equation}
{\mit\Phi}_2 = M_{21}\,I_1,
\end{equation}
where $M_{21}$ is a constant of proportionality. This constant is  called
the {\em mutual inductance}\/ of the two loops. 
\begin{figure}
\epsfysize=2.in
\centerline{\epsffile{chapter7/fig7.1.eps}}
\caption{\em Two current carrying loops.}\label{floop}
\end{figure}

Let us write the magnetic field ${\bf B}_1$ in terms of a vector potential ${\bf A}_1$, so that
\begin{equation}
{\bf B}_1 = \nabla\times {\bf A}_1.
\end{equation}
It follows from Stokes' theorem that
\begin{equation}\label{e7.5}
{\mit\Phi}_2 = \int_{\rm loop \,2} {\bf B}_1\cdot d{\bf S}_2 = \int_{\rm loop \,2}
\nabla\times {\bf A}_1 \cdot d{\bf S}_2 = \oint_{\rm loop\, 2} {\bf A}_1 \cdot
d{\bf l}_2,
\end{equation}
where $d{\bf l}_2 $ is a line element of loop 2. 
However, we know that
\begin{equation}
{\bf A}_1 ({\bf r}) = \frac{\mu_0 \,I_1}{4\pi}
\oint_{\rm loop \,1} \frac{d {\bf l}_1}{|{\bf r} - {\bf r}_1|}.
\end{equation}
 The above equation is just a special case of the more general law,
\begin{equation}
{\bf A}_1({\bf r}) = \frac{\mu_0}{4\pi} \int_{\rm all \,space} 
\frac{{\bf j}({\bf r}')}
{|{\bf r} - {\bf r}'|} \,d^3{\bf r}',
\end{equation}
for ${\bf j}({\bf r}_1) = d{\bf l}_1 \,I_1/ dl_1\, dA$ and $d^3{\bf r}' = dl_1 \,dA$, where
$dA$ is the cross-sectional area of loop 1. Thus,
\begin{equation}
{\mit\Phi}_2 = \frac{\mu_0 \,I_1}{4\pi} 
\oint_{\rm loop \,1}\oint_{\rm loop \,2} \frac{d{\bf l}_1\cdot d{\bf l}_2}{|{\bf r}_2-
{\bf r}_1|},
\end{equation}
where ${\bf r}_2$ is the position vector of the line element $d{\bf l}_2$
of loop 2, which implies that
\begin{equation}
M_{21}   = \frac{\mu_0}{4\pi} 
\oint_{\rm loop\, 1}\oint_{\rm loop \,2} \frac{d{\bf l}_1\cdot d{\bf l}_2}{|{\bf r}_2-
{\bf r}_1|}.
\end{equation}
In fact, mutual
inductances are rarely worked out using the above formula, because it is usually
much too difficult. However, this expression---which is known as the {\em Neumann formula}---tells us two important things. 
Firstly, the mutual inductance of two current loops is a purely {\em geometric}\/ quantity,
having to do with the sizes, shapes, and relative orientations of the loops.
Secondly, the integral is unchanged if we switch the roles of loops 1 and 2.
In other words, 
\begin{equation}
M_{21} = M_{12}.
\end{equation}
Hence, we can drop the subscripts, and just call both  these quantities $M$.
This is a rather surprising result. It implies that no matter what the shapes and
relative positions of the two loops, the magnetic flux through loop 2 when we run a
current $I$ around loop 1 is {\em exactly} the same as the flux through loop 1
when we run the same current around loop 2.
 
We have seen that a current $I$ flowing around some wire loop, 1, generates a magnetic
flux linking some other loop, 2. However,  flux is also generated through the
first loop. As before, the magnetic field, and, therefore, the flux, ${\mit\Phi}$, 
is proportional to the current, so we can write
\begin{equation}
{\mit\Phi} = L\, I.
\end{equation}
The constant of proportionality $L$ is called the {\em self-inductance}. Like
$M$ it only depends on the geometry of the loop. 

Inductance is measured in SI units called henries (H): 1 henry is 1 volt-second
per ampere. The henry, like the farad, is a rather unwieldy unit, since
inductors in electrical circuits typically have a inductances of order a micro-henry. 

\section{Self-Inductance}
Consider a long, uniformly wound, cylindrical solenoid of length $l$, and radius $r$,
which has $N$ turns per unit length,
and carries a current $I$. The longitudinal ({\em i.e.}, directed along the
axis of the solenoid) magnetic field within the solenoid is approximately uniform,
and is given by
\begin{equation}
B= \mu_0\, N\, I.
\end{equation}
(This result is easily obtained by integrating Amp\`{e}re's law over a rectangular
loop whose long sides run parallel to the axis of the solenoid, one inside the
solenoid, and the other outside, and whose short sides run perpendicular to the
axis.) The magnetic flux though each turn of the solenoid wire is $B\,\pi \,r^2=
\mu_0\, N \,I\,\pi \,r^2$. The total flux through
the solenoid wire, which has $N\,l$ turns, is
\begin{equation}
{\mit\Phi} = N\,l\, \mu_0 \,N\,I\, \pi \,r^2.
\end{equation}
Thus, the self-inductance of the solenoid is
\begin{equation}
L = \frac{{\mit\Phi}}{I} = \mu_0\, N^2\, \pi\, r^2\, l.
\end{equation}
Note that the self-inductance only depends on geometric quantities, such as the number
of turns per unit length of the solenoid, and the cross-sectional area of the turns. 

Suppose that the current $I$ flowing through the solenoid changes. Let us
assume that the change is sufficiently slow that we can neglect the displacement
current, and retardation effects, in our calculations. This implies that the typical
time-scale of the change must be much longer than the time for a light-ray to traverse the
circuit. If this is the case then the above formulae remain valid. 

A change in the current implies a change in the magnetic flux linking the solenoid
wire, since ${\mit\Phi} = L \,I$. According to Faraday's
law, this change
generates an emf in the wire. By Lenz's law, the emf is such
as to oppose the change in the current---{\em i.e.}, it is a {\em back-emf}. Thus, we can write
\begin{equation}
V = - \frac{d {\mit\Phi}}{d t} = - L\, \frac{d I}{dt},
\end{equation}
where $V$ is the generated back-emf. 


Suppose that our solenoid has an electrical resistance $R$. Let us
connect the ends of the solenoid across the terminals of a battery of
constant voltage $V$. What is going to happen? The equivalent circuit is shown in Figure~\ref{f45}.
\begin{figure}
\epsfysize=2.5in
\centerline{\epsffile{chapter7/fig7.2.eps}}
\caption{\em The equivalent circuit of a solenoid connected to a battery.}\label{f45}
\end{figure}
The inductance and resistance of the solenoid are represented by a perfect
inductor, $L$, and a perfect resistor, $R$, connected in series. The voltage drop
across the inductor and resistor is equal to the voltage of the battery,
$V$. The voltage drop across the resistor is simply $I\,R$, whereas the
voltage drop across the inductor ({\em i.e.},  the back-emf) is
$L \,dI/dt$. Here, $I$ is the current flowing through the solenoid. 
It follows that 
\begin{equation}
V = I\,R + L \,\frac{dI}{dt}.
\end{equation}
This is a differential equation for the current $I$. We can rearrange it to
give
\begin{equation}
\frac{dI}{dt}+ \frac{R}{L}\,I= \frac{V}{L}.
\end{equation}
The general solution is
\begin{equation}
I(t) = \frac{V}{R} + k \exp(-R\,t/L).
\end{equation}
The constant $k$ is fixed by the boundary conditions. Suppose that the
battery is connected at time $t=0$, when $I=0$. It follows that $k=-V/R$, so
that
\begin{equation}
I(t) = \frac{V}{R} \left[1-\exp(-R\,t/L)\,\right].
\end{equation}
This curve is shown in Figure~\ref{flr}. It can be seen that, after the battery is connected, the current
ramps up, and attains its steady-state value $V/R$ (which comes from Ohm's
law), on the characteristic time-scale
\begin{equation}
\tau = \frac{L}{R}.
\end{equation}
This time-scale is sometimes called the  {\em time constant}\/ of the circuit, or
(somewhat unimaginatively) the  {\em L over R time}\/ of the circuit. 
\begin{figure}
\epsfysize=2.5in
\centerline{\epsffile{chapter7/fig7.3.eps}}
\caption{\em Typical current rise profile in a circuit of the type shown in Figure~\ref{f45}. Here, $I_0 = V/R$ and $\tau=L/R$.}\label{flr}
\end{figure}

We can now appreciate the significance of self-inductance. The back-emf 
generated in an inductor, as the current flowing through it tries to change, prevents the
current from rising (or falling) much faster than the $L/R$ time. This effect is
sometimes advantageous, but  is often a great nuisance.
All circuit elements possess some self-inductance, as well as some resistance, and thus have a finite $L/R$ time. This means that when we power up a DC circuit, the current
does not jump up instantaneously to its steady-state value. Instead, the
rise is spread out over the L/R time of the circuit. This is a good thing.
If the current were to rise instantaneously then extremely large electric
fields would be generated by the sudden jump in the induced magnetic field, leading,
inevitably, to breakdown and electric arcing. So, if there were no such thing
as self-inductance then every time we switched an electric circuit on or off
there would be a blue flash due to arcing between conductors. Self-inductance
can also be a bad thing. Suppose that we possess an expensive power supply which can generate a wide variety of complicated voltage waveforms, and we wish
to use it to send such a waveform down a wire (or transmission line).
Of course, the wire or transmission line will possess both resistance and inductance,
and will, therefore, have some characteristic $L/R$ time. Suppose that we
try to send a square-wave signal down the wire. Since the current in the wire
cannot rise or fall faster than the $L/R$ time,  the leading and trailing edges of
the signal will get smoothed out over an $L/R$ time. The typical difference between
the signal fed into the wire, and that which comes out of the
other end, is illustrated in Figure~\ref{fsquare}. Clearly, there is little
point in having an expensive power supply unless we also possess a low inductance
wire, or transmission line, so that the signal from the power supply can be
transmitted to some load device without serious distortion. 
\begin{figure}
\epsfysize=2.in
\centerline{\epsffile{chapter7/fig7.4.eps}}
\caption{\em Difference between the input waveform (left panel) and
the output waveform (right panel) when a square-wave is sent down a
wire whose $L/R$ time is 1/20th of the square-wave period, $T$.}\label{fsquare}
\end{figure}

\section{Mutual Inductance}
Consider, now, two long thin cylindrical solenoids, one wound on top of the other. The length
of each solenoid is $l$, and the common radius is $r$. Suppose that the bottom
coil has $N_1$ turns per unit length, and carries a current $I_1$. The magnetic
flux passing through each turn of the top coil is $\mu_0\, N_1\, I_1\,\pi \,r^2$, and
the total flux linking the top coil is therefore ${\mit\Phi}_2 = N_2 \,l\, \mu_0 N_1\,
I_1\,\pi\, r^2$, where $N_2$ is the number of turns per unit length in the top
coil. It follows that the mutual inductance of the two coils, defined ${\mit\Phi}_2
=M\, I_1$, is given by 
\begin{equation}
M = \mu_0\, N_1\, N_2 \,\pi\, r^2\, l.
\end{equation}
Recall that the self-inductance of the bottom coil is
\begin{equation}
L_1 = \mu_0\, N_1^{\,2} \,\pi \,r^2 \,l,
\end{equation}
and that  of the top coil is
\begin{equation}
L_2 = \mu_0 \,N_2^{\,2} \,\pi \,r^2 \,l.
\end{equation}
Hence, the mutual inductance can be written
\begin{equation}
M = \sqrt{L_1\, L_2}.
\end{equation}
Note that this result depends on the assumption that {\em all}\/ of the magnetic flux produced
by one coil passes through the other coil. In reality, some of the flux
leaks out, so that the mutual inductance is somewhat less than that given in the
above formula. We can write
\begin{equation}\label{e7.25}
M= k \,\sqrt{L_1 \,L_2},
\end{equation}
where the dimensionless constant $k$ is called the {\em coefficient of coupling},
and lies in the range $0\leq k \leq 1$. 

Suppose that the two coils have resistances $R_1$ and $R_2$. If the bottom coil
has an instantaneous  current $I_1$ flowing through it, and a total voltage drop
$V_1$,  then the voltage drop due to its resistance is $I_1 \,R_1$. The voltage drop
due to the back-emf generated by the self-inductance of the coil is
$L_1\, d I_1/dt$. There is also a back-emf due to inductive coupling with
the top coil. We know that the flux through the bottom coil due to the instantaneous
current $I_2$ flowing in the top coil is
\begin{equation}
{\mit\Phi}_1 = M \,I_2.
\end{equation}
Thus, by Faraday's law and Lenz's law, the  back-emf induced in the bottom
coil is
\begin{equation}
V = - M \,\frac{dI_2}{dt}.
\end{equation}
The voltage drop across the bottom coil due to its mutual inductance with the
top coil is minus this expression. Thus, the circuit equation for the bottom coil is
\begin{equation}
V_1 = R_1\, I_1 + L_1 \frac{dI_1}{dt} + M\,\frac{dI_2}{dt}.
\end{equation}
Likewise, the circuit equation for the top coil is
\begin{equation}
V_2 = R_2 \,I_2 + L_2 \frac{d I_2}{dt} + M\,\frac{d I_1}{dt}.
\end{equation}
Here, $V_2$ is the total voltage drop across the top coil. 

Suppose that we suddenly connect a battery of constant voltage
$V_1$ to the bottom coil, at  time $t=0$. The top coil is assumed to be
open-circuited, or connected to a voltmeter of very high internal resistance,
so that $I_2=0$. What is the voltage generated in the top coil?
Since $I_2=0$, the circuit equation for the bottom coil is
\begin{equation}
V_1 = R_1 \,I_1 + L_1 \frac{d I_1}{dt},
\end{equation}
where $V_1$ is constant, and $I_1(t=0)=0$. We have already seen the solution to
this equation:
\begin{equation}
I_1 = \frac{V_1}{R_1} \left[ 1 - \exp(-R_1\, t/L_1)\right].
\end{equation}
The circuit equation for the top coil is
\begin{equation}
V_2 = M \,\frac{d I_1}{dt},
\end{equation}
giving 
\begin{equation}
V_2 = V_1\, \frac{M}{L_1} \exp(- R_1\, t/L_1).
\end{equation}
It follows from Equation~(\ref{e7.25}) that
\begin{equation}
V_2 = V_1 \,k \,\sqrt{\frac{L_2}{L_1}} \,\exp(- R_1\, t/L_1).
\end{equation}
Since $L_1/L_2 =N_1^{\,2}/N_2^{\,2}$, we obtain
\begin{equation}
V_2 = V_1\, k \,\frac{N_2}{N_1} \,\exp(-R_1 \,t/L_1).
\end{equation}
Note that $V_2(t)$ is discontinuous at $t=0$. This is not a problem, since the
resistance of the top circuit is infinite, so there is no discontinuity in the
current (and, hence, in the magnetic field). But, what about the displacement current,
which is proportional to $\partial {\bf E}/\partial t$? Surely, this is discontinuous
at $t=0$ (which is clearly unphysical)? The crucial point, here,
 is that we have specifically
neglected the displacement current in all of our previous analysis, so it does not
make much sense to start worrying about it now. If we had retained the displacement
current in our calculations then
 we would have found that the voltage in the top circuit jumps up,
at $t=0$, on a time-scale similar to the light traverse time across the circuit
({\em i.e.}, the jump is instantaneous, to all intents and purposes, but the
displacement current remains finite). 

Now,
\begin{equation}
\frac{V_2(t=0)}{V_1} = k\, \frac{N_2}{N_1},
\end{equation}
so if $N_2 \gg N_1$  then  the voltage in the bottom circuit is considerably amplified
in the top circuit. This effect is  the basis for  old-fashioned car ignition
systems. A large voltage spike is induced in a secondary circuit (connected to
a coil with very many turns) whenever the current in a primary circuit 
(connected to a coil with not so many turns) is either switched on or off.
The primary circuit is connected to the car battery (whose voltage is
typically 12 volts). 
The switching is done by a set of points, which are mechanically opened and
closed as the engine turns. The large voltage spike induced in the secondary circuit,
as the points are either opened or closed, causes a spark to jump across a gap
in this circuit. This spark ignites a petrol/air mixture in one of the cylinders. 
We might think that the optimum configuration is to have only one turn in the primary
circuit, and lots of turns in the secondary circuit, so that the ratio
$N_2/N_1$ is made as large as possible. However, this is not the case. Most of
the magnetic  flux generated by a single turn primary coil is likely to
miss the secondary coil altogether. This means that the coefficient of coupling $k$
is small, which reduces the voltage induced in the secondary circuit. Thus, we
need a reasonable number of turns in the primary coil in order to localize the
induced magnetic flux, so that it links effectively with the secondary coil.

\section{Magnetic Energy}
Suppose that, at $t=0$, a coil of inductance, $L$, and resistance $R$, is connected
across the terminals of a battery of voltage $V$. The circuit equation is
\begin{equation}\label{e7.37}
V = L\,\frac{d I}{dt} + R\,I.
\end{equation}
Now, the power output of the battery is $V\,I$. [Every charge $q$ that goes around the circuit
falls through a potential difference $q \,V$. In order to raise it back to
the starting potential, so that it can perform another circuit, the battery must do
work $q\,V$. The work done per unit time ({\em i.e.}, the power) is $n\,q\,V$, where $n$ is
the number of charges per unit time passing a given point on the circuit. 
But, $I=n\,q$, so the power output is $V\,I$.]\@ Thus, the net work done by the battery in
raising the current in the circuit from zero at time $t=0$ to $I_T$ at
time $t=T$ is 
\begin{equation}
W = \int_0^T V\,I \,dt.
\end{equation}
Using the circuit equation (\ref{e7.37}), we obtain
\begin{equation}
W = L \int_0^T I\,\frac{dI}{dt} \,dt + R \int_0^T I^2\,dt,
\end{equation}
giving
\begin{equation}
W = \frac{1}{2}\, L\, I_T^{\,2} + R \int_0^T I^2\,dt. 
\end{equation}
The second term on the right-hand side of the above equation represents the irreversible conversion of
electrical energy into heat energy by the resistor. The first term is the amount of
energy stored in the inductor at time $T$. This energy can be recovered after the
inductor is disconnected from the battery. Suppose that the battery is disconnected
at time $T$. The circuit equation is now
\begin{equation}
0 = L\,\frac{dI}{dt} + RI,
\end{equation}
giving 
\begin{equation}\label{e7.42}
I= I_T \exp \left[ - \frac{R}{L} \,(t-T)\right],
\end{equation}
where we have made use of the boundary condition $I(T) = I_T$. 
Thus, the current
decays away exponentially. The energy stored in the inductor is dissipated as
heat in the resistor. The total heat energy appearing in the resistor after the
battery is disconnected is
\begin{equation}
\int_T^\infty I^2 \,R \,dt = \frac{1}{2}\, L\, I_T^{\,2},
\end{equation}
where use has been made of Equation~(\ref{e7.42}). 
Thus, the heat energy appearing in the resistor is equal to the
energy stored in the inductor. This energy is actually stored in the magnetic
field generated around the inductor. 

Consider, again, our circuit with two coils wound on top of one another. Suppose that 
each coil is connected to its own battery. The circuit equations are thus
\begin{eqnarray}
V_1 &=& R_1\, I_1 + L_1\,\frac{d I_1}{dt} +M\,\frac{d I_2}{dt},\nonumber\\[0.5ex]
V_2 &=& R_2 \,I_2 + L_2\, \frac{d I_2}{d t} + M\,\frac{d I_1}{dt},
\end{eqnarray}
where $V_1$ is the voltage of the battery in the first circuit, {\em etc.} 
The net work done by the two batteries in increasing the currents in the two circuits,
from zero at time 0, to $I_1$ and $I_2$ at time $T$, respectively, is
\begin{eqnarray}
W &=& \int_0^T (V_1\, I_1 + V_2 \,I_2 )\,dt\nonumber\\[0.5ex]
&=& \int_0^T (R_1 \,I_1^{\,2} + R_2 \,I_2^{\,2})\,dt +\frac{1}{2}\, L_1 \,I_1^{\,2}
+ \frac{1}{2}\, L_2 \,I_2^{\,2}\nonumber\\[0.5ex]
&& + M \int_0^T \left(I_1\, \frac{dI_2}{dt}  + I_2\,\frac{d I_1}{dt } \right)dt.
\end{eqnarray}
Thus, 
\begin{eqnarray}
W &=& \int_0^T (R_1 \,I_1^{\,2} + R_2\, I_2^{\,2} )\,dt\nonumber\\[0.5ex]
&& + \frac{1}{2} \,L_1 \,I_1^{\,2} + \frac{1}{2}\, L_2 \,I_2^{\,2} + M\, I_1\, I_2.
\end{eqnarray}
Clearly, the total magnetic energy stored in the two coils is
\begin{equation} 
W_B =  \frac{1}{2}\, L_1\, I_1^{\,2} + \frac{1}{2} \,L_2 \,I_2^{\,2} + M \,I_1 \,I_2.
\end{equation}
Note that the mutual inductance term increases the stored magnetic energy if $I_1$ and
$I_2$ are of the same sign---{\em i.e.}, if the currents in the  two coils flow
in the same direction, so that they generate magnetic fields which reinforce
one another. Conversely, the mutual inductance term decreases the stored 
magnetic energy if $I_1$ and $I_2$ are of the opposite sign. However, the total 
stored energy can never be negative, otherwise the coils
would constitute a power source (a negative stored energy is equivalent to
a positive generated energy). Thus,
\begin{equation} 
\frac{1}{2}\, L_1 \,I_1^{\,2} + \frac{1}{2}\, L_2 \,I_2^{\,2} + M\, I_1 \,I_2\geq 0,
\end{equation}
which can be written
\begin{equation}
\frac{1}{2}\left( \sqrt{L_1} \,I_1 + \sqrt{L_2}\,I_2\right)^2 - I_1 \,I_2 (\sqrt{L_1 \,L_2}
-M) \geq 0,
\end{equation}
assuming that $I_1\, I_2 <0$. It follows that
\begin{equation}
M \leq \sqrt{L_1 \,L_2}.
\end{equation}
The equality sign corresponds to the situation in which
all of the magnetic flux generated by one coil passes through the other. If some of
the flux misses then the inequality sign is appropriate. 
In fact, the above formula is valid 
for any two inductively coupled circuits, and effectively sets an upper limit
on their mutual inductance.

We intimated previously that the energy stored in an inductor is actually
stored in the surrounding  magnetic field. Let us now obtain an
explicit formula for the energy stored in a magnetic field. Consider an ideal
cylindrical solenoid. The energy stored in the solenoid when a current $I$ flows through it
is
\begin{equation}\label{e7.51}
W = \frac{1}{2} \,L \,I^2,
\end{equation}
where $L$ is the self-inductance. We know that
\begin{equation}
L = \mu_0\, N^2\,\pi\, r^2\, l,
\end{equation}
where $N$ is the number of turns per unit length of the
solenoid, $r$ the radius, and $l$ the length. The magnetic field inside the solenoid is approximately
uniform, with magnitude
\begin{equation}
B = \mu_0 \,N\,I,
\end{equation}
and is approximately zero outside the solenoid. Equation~(\ref{e7.51}) can be rewritten 
\begin{equation}
W = \frac{B^2}{2\mu_0}\,V,
\end{equation}
where $V = \pi\, r^2 \,l$ is the volume of the solenoid. The above formula strongly
suggests that a magnetic field possesses an energy density
\begin{equation}\label{eeb}
U = \frac{B^2}{2\mu_0}.
\end{equation}

Let us now examine  a more general proof of the above formula. Consider a system
of $N$ circuits   (labeled $i=1$ to $N$), each carrying a current $I_i$. 
The magnetic flux through the $i$th circuit is written [{\em cf}., Equation~(\ref{e7.5})]
\begin{equation}\label{e7.56}
{\mit\Phi}_i = \int {\bf B} \cdot d{\bf S}_i = \oint {\bf A} \cdot d{\bf l}_i,
\end{equation}
where ${\bf B} = \nabla\times{\bf A}$, and $d{\bf S}_i$ and $d{\bf l}_i$ denote a
surface element and a line element of this  circuit, respectively. The 
back-emf induced in the $i$th circuit follows from Faraday's law:
\begin{equation}
V_i = - \frac{d {\mit\Phi}_i}{dt}.
\end{equation}
The rate of work of the battery which maintains  the current $I_i$ 
in the $i$th circuit
against this back-emf is 
\begin{equation}
P_i = I_i\, \frac{d {\mit\Phi}_i}{dt}.
\end{equation}
Thus, the total work required
 to raise the currents in the $N$ circuits from zero at time
0, to $I_{0\,i}$ at time $T$, is
\begin{equation}
W = \sum_{i=1}^N \int_0^T I_i \,\frac{d{\mit\Phi}_i}{dt}\,dt.
\end{equation}
The above expression for the work done is, of course, equivalent to the total
energy stored in the magnetic field surrounding the various circuits. 
This energy is independent of the manner in which the currents
 are set up. 
Suppose, for the sake of simplicity,  that the currents are ramped up linearly,
so that
\begin{equation}
I_i = I_{0\,i}\, \frac{t}{T}.
\end{equation}
The fluxes are proportional to the currents, so they must also ramp up linearly: {\em i.e.}, 
\begin{equation}
{\mit\Phi}_i = {\mit\Phi}_{0\,i} \,\frac{t}{T}.
\end{equation}
It follows that
\begin{equation}
W = \sum_{i=1}^N \int_0^T I_{0\,i} \,{\mit\Phi}_{0\,i}\, \frac{t}{T^2}\,dt,
\end{equation}
giving
\begin{equation}
W = \frac{1}{2} \sum_{i=1}^N I_{0\,i} \,{\mit\Phi}_{0\,i}.
\end{equation}
So, if instantaneous currents $I_i$ flow in the the $N$ circuits, which link 
instantaneous fluxes ${\mit\Phi}_i$, then the instantaneous stored energy is
\begin{equation}\label{e7.64}
W= \frac{1}{2} \sum_{i=1}^N I_i \,{\mit\Phi}_i.
\end{equation}

Equations (\ref{e7.56}) and (\ref{e7.64}) imply that
\begin{equation}\label{e7.65}
W = \frac{1}{2} \sum_{i=1}^N I_i \oint {\bf A} \cdot d{\bf l}_i.
\end{equation}
It is convenient, at this stage, to replace our $N$ line currents by
$N$  current
distributions  of
small, but finite, cross-sectional area.
Equation (\ref{e7.65})
transforms to
\begin{equation}
W = \frac{1}{2} \int_V {\bf A} \cdot {\bf j} \, dV,
\end{equation}
where $V$ is a volume which contains all of the circuits.
Note that for an element of the $i$th circuit, ${\bf j} =I_i\, d{\bf l}_i
/dl_i \,A_i$ and  $dV = dl_i \,A_i$, where $A_i$ is the cross-sectional area of the
 circuit.
Now, $\mu_0\, {\bf j} = \nabla\times {\bf B}$ (we are neglecting the displacement
current in this calculation), so
\begin{equation}
W = \frac{1}{2\mu_0} \int_V {\bf A} \cdot \nabla\times{\bf B} \, dV.
\end{equation}
According to vector field theory,
\begin{equation}
\nabla\cdot({\bf A} \times{\bf B}) \equiv {\bf B} \cdot \nabla\times{\bf A} -
{\bf A} \cdot \nabla\times{\bf B},
\end{equation}
which implies that
\begin{equation}
W = \frac{1}{2\mu_0} \int_V \left[- \nabla\cdot ({\bf A} \times{\bf B})
+{\bf B}\cdot \nabla \times {\bf A} \right] \,dV.
\end{equation}
Using Gauss' theorem, and ${\bf B} = \nabla\times{\bf A}$, we obtain
\begin{equation}
W = -\frac{1}{2\mu_0} \oint_S {\bf A}\times{\bf B} \cdot d{\bf S} +
\frac{1}{2\mu_0} \int_V B^2\,dV,
\end{equation}
where $S$ is the bounding surface of  some volume $V$. Let us take this surface
to infinity. It is easily demonstrated that the magnetic field generated by a current
loop falls of like $r^{-3}$ at large distances. The vector potential
falls off like $r^{-2}$. However, the  area of surface $S$ only increases like $r^2$. 
It follows that the surface integral is negligible in the limit $r\rightarrow\infty$.
Thus, the above expression reduces to
\begin{equation}
W = \int \frac{B^2}{2\mu_0} \, dV,
\end{equation}
where the integral is over all space.
Since this expression is valid for any magnetic field whatsoever, we can safely conclude 
that the energy density of a general magnetic field generated by a system of electrical circuits is given by
\begin{equation}
U = \frac{B^2}{2\mu_0}.
\end{equation}
Note, that the above expression is consistent with the expression
(\ref{e6en}) which we previously obtained during our investigation of magnetic
media.

\section{Alternating Current Circuits}
Alternating current (AC) circuits are made up of voltage sources and {\em three}\/
different types of passive element: {\em i.e.}, resistors, inductors,
and capacitors.  Resistors satisfy Ohm's law,
\begin{equation}
V = I \,R,
\end{equation}
where $R$ is the resistance, $I$ the current flowing through the resistor, and
$V$ the voltage drop across the resistor (in the direction in which the current
flows). Inductors satisfy
\begin{equation}
V = L\, \frac{dI}{dt},
\end{equation}
where $L$ is the inductance. Finally, capacitors obey
\begin{equation}
V = \frac{q}{C} = \left.\int_0^t I\,dt\right/C,
\end{equation}
where $C$ is the capacitance, $q$ is the charge stored on the plate with the most
positive potential, and $I=0$ for $t<0$. Note that any 
passive component of a real electrical
circuit can always be represented as a combination of ideal resistors, inductors, and
capacitors. 


Let us consider the classic LCR circuit, which consists of an inductor, $L$, a
capacitor, $C$, and a resistor, $R$, all connected in series with an voltage source,
$V$---see Figure~\ref{flcr}. The circuit equation is obtained by setting the input voltage $V$  equal to
the sum of the voltage drops across the three passive elements in the circuit. 
Thus,
\begin{equation}\label{e776}
V = I\,R+ L\,\frac{dI}{dt} + \left.\int_0^t I\,dt\right/C.
\end{equation}
This is an integro-differential equation which, in general, is quite difficult to
solve. Suppose, however, that both the voltage and the current 
oscillate at some fixed angular frequency $\omega$, so that
\begin{eqnarray}
V(t) &=& V_0\, \exp({\rm i}\,\omega\, t),\\[0.5ex]
I(t) &=& I_0 \,\exp({\rm i}\,\omega \,t),\label{e777}
\end{eqnarray}
where the physical solution is understood to be the {\em real part} of
the above expressions. The assumed behaviour of the voltage and current is
clearly relevant to electrical
 circuits powered by the mains voltage (which oscillates at 60 hertz). 
 \begin{figure}
\epsfysize=2.5in
\centerline{\epsffile{chapter7/fig7.5.eps}}
\caption{\em An LCR circuit.}\label{flcr}
\end{figure}


Equations~(\ref{e776})--(\ref{e777}) yield
\begin{equation}
V_0\,  \exp({\rm i}\,\omega \,t) =I_0 \, \exp({\rm i}\,\omega \,t)\,
R+ L\, {\rm i}\,\omega \,I_0\exp({\rm i}\,\omega \,t)
+\frac{I_0 \exp({\rm i}\,\omega \,t)}{{\rm i}\,\omega\, C},
\end{equation}
giving
\begin{equation}
V_0 = I_0 \left({\rm i}\,\omega\, L + \frac{1}{{\rm i}\,\omega\, C} + R\right).
\end{equation}
It is helpful to define the {\em impedance} of the circuit:
\begin{equation}\label{e7.80}
Z = \frac{V}{I} = {\rm i}\,\omega\, L + \frac{1}{{\rm i }\, \omega \,C} + R.
\end{equation}
Impedance is a generalization of the concept of resistance. In general, the impedance
of an AC circuit is a {\em complex} quantity. 

The average power output of  the voltage source is
\begin{equation}
P = \langle  V(t) \,I(t) \rangle,
\end{equation}
where the average is taken over one period of the oscillation. Let us, first of all,
calculate the power using real (rather than complex)  voltages and currents.
We can write
\begin{eqnarray}
V(t) &=& |V_0|\, \cos(\omega\, t),\\[0.5ex] 
I(t) &=& |I_0| \,\cos(\omega\, t - \theta),
\end{eqnarray}
where $\theta$ is the phase-lag of the current with respect to the voltage.
It follows that
\begin{eqnarray}
P& = &|V_0|\, |I_0| \int_{\omega t = 0}^{\omega t = 2\pi}
\cos(\omega\, t)\, \cos(\omega\, t - \theta)\,\,\frac{d(\omega \,t)}{2\pi}\\[0.5ex]
&=& |V_0|\, |I_0| \int_{\omega t =0}^{\omega t = 2\pi} 
\cos(\omega\, t) \left[\cos(\omega\, t)\, \cos\theta + \sin(\omega \,t) \,\sin \theta\right]\,\,
\frac{d(\omega\, t)}{2\pi},\nonumber
\end{eqnarray}
giving
\begin{equation}
P = \frac{1}{2}\, |V_0|\, |I_0| \cos\theta,
\end{equation}
since $\langle \cos(\omega\, t)\,\sin(\omega\, t)\rangle = 0$ and $\langle \cos(\omega\, t) \,\cos(\omega \,t)\rangle = 1/2$. 
In complex representation, the voltage and the current are written
\begin{eqnarray}
V(t) &=& |V_0| \,\exp({\rm i}\,\omega\,t),\\[0.5ex] 
I(t) &=& |I_0|\, \exp[{\rm i}\,(\omega \,t - \theta)].
\end{eqnarray}
Now,
\begin{equation}
\frac{1}{2} ( V\, I^\ast +V^\ast\, I)=
|V_0|\, |I_0|\, \cos\theta.
\end{equation}
It follows that
\begin{equation}
P = \frac{1}{4} ( V\, I^\ast + V^\ast \,I) = \frac{1}{2}\, {\rm Re}(V \,I^\ast).
\end{equation}
Making use of Equation~(\ref{e7.80}), we find that
\begin{equation}
P = \frac{1}{2} \,{\rm Re}(Z)\, |I|^2 = \frac{1}{2} \frac{{\rm Re}(Z)\,|V|^2}{|Z|^2}.
\end{equation}
Note that power dissipation is associated with the {\em real part} of the impedance.
For the special case of an LCR circuit, 
\begin{equation}
P = \frac{1}{2} \,R \,|I_0|^{\,2}.
\end{equation}
We conclude that only the resistor dissipates energy in this circuit. The inductor and
the capacitor both store energy, but they eventually return it to the circuit
without dissipation. 

According to Equation~(\ref{e7.80}), the amplitude of the current which flows in an LCR circuit
for a given amplitude of the input voltage  is
given by
\begin{equation}
|I_0| = \frac{|V_0|}{|Z|}= \frac{|V_0|}{\sqrt{(\omega \,L-1/\omega \,C)^2 + R^2}}.
\end{equation}
As can be seen from Figure~\ref{flcr1}, the response of the circuit is
  {\em resonant}, peaking at $\omega = 1/\sqrt{L\,C}$, and reaching
$1/\sqrt{2}$ of the peak value at $\omega = 1/\sqrt{L\,C} \pm R/(2\,L)$ (assuming that
$R \ll \sqrt{L/C}$). For this reason, LCR circuits are used in analog radio tuners to filter out
signals whose frequencies fall outside a given band. 
\begin{figure}
\epsfysize=2.25in
\centerline{\epsffile{chapter7/fig7.6.eps}}
\caption{\em The characteristics of an LCR circuit. The left-hand and right-hand panes show the amplitude and phase-lag of the current versus frequency, respectively. Here, $\omega_c=1/\sqrt{L\,C}$ and $Z_0 = \sqrt{L/C}$. The solid, short-dashed, long-dashed,
and dot-dashed curves correspond to $R/Z_0 = 1$, 1/2, 1/4, and 1/8, respectively.}\label{flcr1}
\end{figure}

The phase-lag of the current with respect to the voltage is given by
\begin{equation}
\theta = {\rm arg}(Z) = \tan^{-1}\left( \frac{\omega \,L - 1/\omega\, C }
{R} \right).
\end{equation}
As can be seen from Figure~\ref{flcr1}, the phase-lag varies from $-\pi/2$ for frequencies significantly  below the resonant
frequency, to zero at the resonant frequency ($\omega = 1/\sqrt{L\,C}$), to
$\pi/2$ for frequencies significantly above the resonant frequency.

It is clear that in conventional AC circuits the circuit equation reduces to a
simple algebraic equation, and that the behaviour of the circuit is summed up
by the complex impedance, $Z$. The real part of  $Z$ tells us the power dissipated in
the circuit, the magnitude of $Z$ gives the ratio of the peak current to the
peak voltage, and the argument of $Z$ gives the phase-lag of the current
with respect to the voltage. 

\section{Transmission Lines}\label{strans}
The central assumption made in the analysis of conventional AC circuits is that
the voltage (and, hence,  the current) has the {\em same phase}\/ throughout the circuit. 
Unfortunately, if the circuit is sufficiently large, or the frequency of
oscillation, $\omega$, is sufficiently high, then this assumption becomes invalid. 
The assumption of a constant phase throughout the circuit is reasonable if the 
wavelength of the oscillation, $\lambda = 2\pi \,c/\omega$, is 
much larger than the dimensions of
the circuit. (Here, we assume that signals propagate around electrical circuits
at about the velocity of light. This assumption will be justified later on.)
This is generally not the case in electrical circuits which are
associated with {\em communication}. The frequencies in such
circuits tend to be very high, and the dimensions  are, almost
by definition,  large. For instance,
leased telephone lines (the type to which computers are connected) run at
$56$ kHz. The corresponding wavelength is about 5 km, so the constant-phase
approximation clearly breaks down for long-distance calls. Computer networks
generally run at about 100 MHz, corresponding to $\lambda \sim 3$ m. Thus,
 the constant-phase approximation also breaks down for most computer networks,
since such networks are generally significantly larger than 3\,m.
It turns out that we need a
special sort of wire, called a {\em transmission line}, to propagate signals around circuits
whose dimensions greatly exceed the wavelength, $\lambda$. Let us investigate
transmission lines. 

An idealized transmission line consists of two parallel conductors of uniform
cross-sectional area. 
 The conductors possess a capacitance per unit length,
$C$, and an inductance per unit length, $L$. Suppose that $x$ measures the position
along the line. 

Consider the voltage difference between two neighbouring points
on the line, located at positions $x$ and $x+\delta x$, respectively---see Figure~\ref{ftrans}.
The self-inductance of the portion of the line lying between these two
points is $L\,\delta x$. This small section of the line can be thought of as
a conventional inductor, and, therefore, obeys the well-known equation
\begin{equation}
V(x, t) -V(x+\delta x, t) =  L\,\delta x\,\frac{\partial I(x, t)}{\partial t},
\end{equation}
where $V(x, t)$ is the voltage difference between the two conductors at
position $x$ and time $t$, and 
$I(x, t)$ is the current flowing in one of the conductors at position
$x$ and time $t$ [the current flowing
in the other conductor is $-I(x, t)$]. In the limit $\delta x\rightarrow 0$,
the above equation reduces to
\begin{equation}\label{e7.93}
\frac{\partial V}{\partial x} = - L\,\frac{\partial I}{\partial t}.
\end{equation}
\begin{figure}
\epsfysize=2.in
\centerline{\epsffile{chapter7/fig7.7.eps}}
\caption{\em A segment of a transmission line.}\label{ftrans}
\end{figure}

Consider the difference in current between two neighbouring points on the
line, located at positions $x$ and $x+\delta x$, respectively---see Figure~\ref{ftrans}. The capacitance
of the portion of the line lying between these two points is
$C\,\delta x$. This small section of the line can be thought of
as a conventional capacitor, and, therefore, obeys the well-known equation
\begin{equation}
\int_0^t I(x, t)\,dt - \int_0^t I(x+\delta x, t)\,dt =  C\,\delta x\,V(x,t),
\end{equation}
where $t=0$ denotes a time at which the charge stored in either  of
the conductors in the region $x$ to
$x+\delta x$ is zero. In the limit $\delta x \rightarrow 0$, the above equation
yields
\begin{equation}\label{e7.95}
\frac{\partial I}{\partial x} = - C\,\frac{\partial V}{\partial t}.
\end{equation}
Equations~(\ref{e7.93}) and (\ref{e7.95}) are generally known as the {\em Telegrapher's equations,}
since an old-fashioned telegraph line can be thought of as a primitive
transmission line (telegraph lines consist of a single wire---the other conductor
is the Earth.) 

Differentiating Equation~(\ref{e7.93}) with respect to $x$, we obtain
\begin{equation}
\frac{\partial^2 V}{\partial x^2} = - L\, \frac{\partial^2 I}{\partial x\, \partial t}.
\end{equation}
Differentiating Equation~(\ref{e7.95}) with respect to $t$ yields
\begin{equation}
\frac{\partial^2 I}{\partial x\, \partial t} = 
-C\,\frac{\partial^2 V}{\partial t^2}.
\end{equation}
The above two equations can be combined to give
\begin{equation}\label{e7.98}
L\,C\,\frac{\partial^2 V}{\partial t^2} = \frac{\partial^2 V}{\partial x^2}.
\end{equation}
This is clearly a wave equation, with wave velocity  $v= 1/\sqrt{L\,C}$. An
analogous equation can be written for the current, $I$.

Consider a transmission line which is connected to a generator at one end 
($x=0$), and
a resistor, $R$, at the other ($x=l$). Suppose that the generator outputs
a voltage $V_0 \,\cos(\omega\, t)$. If follows that
\begin{equation}
V(0, t) = V_0\, \cos(\omega\, t).
\end{equation}
The solution to the wave equation (\ref{e7.98}), subject to the above boundary condition, is
\begin{equation}\label{e7.100}
V(x, t) = V_0 \cos(\omega\, t - k\,x),
\end{equation}
where $k= \omega / v$. This clearly corresponds to a wave which propagates
from the generator towards the resistor. Equations (\ref{e7.93}) and (\ref{e7.100}) yield
\begin{equation}
I(x, t)  = \frac{V_0}{\sqrt{L/C}} \cos(\omega \,t -k\,x).
\end{equation}
For self-consistency, the resistor at the end of the line must have a particular 
value:
\begin{equation}
R = \frac{V(l, t)}{I(l, t)} = \sqrt{\frac{L}{C}}.
\end{equation}
The  so-called  {\em input impedance} of the line is defined
\begin{equation}
Z_{\rm in} = \frac{V(0, t)}{I(0, t)} = \sqrt{\frac{L}{C}}.
\end{equation}
Thus, a transmission line terminated by a resistor $R= \sqrt{L/C}$ acts very
much like a conventional resistor $R=  Z_{\rm in}$ in the circuit containing
the generator. In fact, the transmission line could be replaced by an effective
resistor $R=  Z_{\rm in}$ in the circuit diagram for the generator circuit.
The power loss due to this effective resistor corresponds to power which
is extracted from the circuit, transmitted down the line, and absorbed by the
terminating resistor. 

The most commonly occurring type of
transmission line is a co-axial cable, which consists of
two co-axial cylindrical conductors of radii $a$ and $b$ (with $b>a$). We
have already shown that the capacitance per unit length of such a cable is
(see Section~\ref{scap})
\begin{equation}
C = \frac{ 2\pi\, \epsilon_0}{\ln (b/a)}.
\end{equation}
Let us now calculate the inductance per unit length. Suppose that the inner conductor
carries a current $I$. According to Amp\`{e}re's law, the magnetic field in the
region between the conductors is given by
\begin{equation}
B_\theta = \frac{\mu_0 \,I}{2\pi\, r}.
\end{equation}
The flux linking unit length of the cable is
\begin{equation}
{\mit\Phi} = \int_a^b B_\theta \,dr = \frac{\mu_0\, I}{2\pi} \ln(b/a).
\end{equation}
Thus, the self-inductance  per unit length is
\begin{equation}
L = \frac{{\mit\Phi}}{I} = \frac{\mu_0}{2\pi} \ln(b/a). 
\end{equation}
So, the speed of propagation of a wave down a co-axial cable is
\begin{equation}
v = \frac{1}{\sqrt{L\,C}} = \frac{1}{\sqrt{\epsilon_0 \,\mu_0}} = c.
\end{equation}
Not surprisingly, the wave (which is a type of electromagnetic wave) propagates at
the speed of light. The impedance of the cable is given by
\begin{equation}
Z_0 = \sqrt{\frac{L}{C} } = \left(\frac{\mu_0}{4\pi^2\, \epsilon_0}\right)^{1/2}
\ln\, (b/a) = 60 \,\ln\, (b/a) \,\,{\rm ohms}.
\end{equation}

If we fill the region between the two cylindrical conductors with a
dielectric of dielectric constant $\epsilon$, then, according to the
discussion in Section~\ref{spolz}, the capacitance per unit length
of the transmission line goes up by a factor $\epsilon$. However,
the dielectric has no effect on magnetic fields, so the inductance
per unit length of the line remains unchanged. It follows that the
propagation speed of signals down a dielectric filled co-axial cable
is
\begin{equation}
v = \frac{1}{\sqrt{L\,C}}= \frac{c}{\sqrt{\epsilon}}.
\end{equation}
As we shall discover later, this is simply the propagation velocity of electromagnetic waves
through a dielectric medium of dielectric constant $\epsilon$. The impedance of the cable
becomes
\begin{equation}\label{ez0}
Z_ 0 = 60\,\frac{\ln(b/a)}{\sqrt{\epsilon}}\,\,{\rm ohms}.
\end{equation}

We have seen that if a transmission line is terminated by a resistor whose
resistance $R$ matches the impedance $Z_0$ 
of the line then all of the power sent down the
line is absorbed by the resistor. What happens if $R\neq Z_0$? The answer is
that
some of the power is reflected back down the line. Suppose that the
beginning of the line lies at $x=-l$, and the end of the line is at $x=0$.
Let us consider a solution
\begin{equation}
V(x, t) = V_0 \,\exp[{\rm i}\,(\omega \,t - k\,x)] + K\, V_0 \, \exp[{\rm i}\,(\omega\, t + k\,x)].
\end{equation}
This corresponds to a voltage wave of amplitude $V_0$ which travels down the line,
and is reflected at the end of the line, with reflection coefficient $K$. 
It is easily demonstrated from the Telegrapher's equations that the corresponding
current waveform is
\begin{equation}
I(x, t) = \frac{V_0}{Z_0} \exp[{\rm i}\,(\omega\, t - k\,x)] - \frac{K \,V_0}{Z_0}
 \exp[{\rm i}\,(\omega\, t + k\,x)].
\end{equation}
Since the line is terminated by a resistance $R$ at $x=0$, we have, from
Ohm's law, 
\begin{equation}
\frac{V(0,t)}{I(0,t)} = R.
\end{equation}
This yields an expression for the coefficient of reflection,
\begin{equation}
K = \frac{ R - Z_0}{R+Z_0}.
\end{equation}
The input impedance of the line is given by
\begin{equation}
Z_{\rm in} = \frac{V(-l, t)}{I(-l, t)} = Z_0\, \frac{R \,\cos (k\,l) +{\rm i}\,Z_0 \,\sin( k\,l)}
{Z_0 \,\cos (k\,l) + {\rm i}\, R \,\sin (k\,l)}.
\end{equation}

Clearly, if the resistor at the end of the line is properly matched, so that
$R=Z_0$, then there is no reflection ({\em i.e.}, $K= 0$), and the input impedance of
the line is $Z_0$. If the line is short-circuited, so that $R=0$, then there is total
reflection at the end of the line ({\em i.e.}, $K= -1$), and the input impedance becomes
\begin{equation}
Z_{\rm in} = {\rm i} \,Z_0 \tan (k\,l).
\end{equation}
This impedance is purely imaginary, implying that the transmission line absorbs no
net power from the generator circuit. In fact, the line acts rather like a pure
inductor or capacitor in the generator circuit 
({\em i.e.}, it can store, but cannot absorb, energy). If the line
is open-circuited, so that $R\rightarrow \infty$, then there is again
total reflection at the end of the line ({\em i.e.}, $K = 1$), and the input 
impedance becomes
\begin{equation}
Z_{\rm in} = {\rm i}\,Z_0\tan(k\,l-\pi/2).
\end{equation}
Thus, the open-circuited line acts like a closed-circuited line which is shorter by
one quarter of a wavelength. For the special case where the length of the line
is exactly one quarter of a wavelength ({\em i.e.}, $k\,l=\pi/2$), we find that
\begin{equation}\label{e7q}
Z_{\rm in} = \frac{Z_0^{\,2}}{R}.
\end{equation}
Thus, a quarter-wave line looks like a pure resistor in the generator circuit.
Finally, if the length of the line is much less than the wavelength ({\em i.e.}, $k\,l\ll 1$)
then we enter the constant-phase regime, and $Z_{\rm in}\simeq R$ ({\em i.e.}, we can
forget about the transmission line connecting the
terminating  resistor to the generator circuit).

Suppose that we wish to build a radio transmitter. We can use a standard half-wave antenna ({\em i.e.}, an antenna whose length is half the wavelength
of the transmitted radio-waves)
to emit the radiation. In electrical circuits, such an antenna acts like a resistor of resistance
73 ohms (it is more usual to  say that the antenna has an impedance of 73 ohms---see Section~\ref{s9.1}). 
Suppose that we buy a 500\,kW
generator to supply the power to the antenna. How do we transmit
the power from the generator to the antenna? We use a transmission line, of course. 
(It is clear that if the distance between  the generator and the antenna is of
order the dimensions of the antenna ({\em i.e.}, $\lambda/2$) then the constant-phase 
approximation breaks down, and so we have to use a transmission line.)
Since the impedance of the antenna is fixed at 73 ohms, we need to use a
73 ohm transmission line ({\em i.e.}, $Z_0 = 73$ ohms) to connect the generator to
the antenna, otherwise some of the power we send down the line is reflected
({\em i.e.}, not all of the power output of the generator is converted into
radio waves). If we wish to use a co-axial cable to connect the generator to
the antenna then it is clear from Equation~(\ref{ez0}) that the  radii of the
inner and outer conductors need to be such that $b/a = 3.38\,\exp(\sqrt{\epsilon})$.

Suppose, finally, that we upgrade our transmitter  to use a full-wave antenna
({\em i.e.}, an antenna whose length equals the wavelength of the emitted radiation).
A full-wave antenna has a different impedance than a half-wave antenna. Does
this mean that we have to rip out our original co-axial cable, and replace it
by one whose impedance matches that of the new antenna? Not necessarily.
Let $Z_0$ be the impedance of the co-axial cable, and $Z_1$ the impedance of
the antenna. Suppose that we place a quarter-wave transmission line ({\em i.e.}, one whose
length is one quarter of a wavelength) of characteristic
impedance $Z_{1/4} = \sqrt{Z_0 \,Z_1}$ between the end of the cable and the
antenna. According to Equation~(\ref{e7q}) (with $Z_0 \rightarrow  \sqrt{Z_0 \,Z_1}$
and $R\rightarrow Z_1$), the input impedance of the quarter-wave line
is $Z_{\rm in} = Z_0$, which matches that of the cable. The output impedance
matches that of the antenna. Consequently, there is no reflection of the power
sent down the cable to the antenna. A quarter-wave line of the appropriate impedance
can easily
 be fabricated from a short length of co-axial cable of the appropriate $b/a$. 

{\small
\section{Exercises}
\renewcommand{\theenumi}{7.\arabic{enumi}}
\begin{enumerate}
\item A planar wire loop of resistance $R$ and cross-sectional area $A$ is placed in a uniform magnetic
field of strength $B$. Let the normal to the loop subtend an angle $\theta$ with the direction of the magnetic field. Suppose that the loop is made to
rotate
steadily, such that $\theta = \omega\,t$. Use Faraday's law to find the
emf induced around the loop. What is the current circulating around the loop. 
Find 
the torque exerted on the loop by the magnetic field. Demonstrate that the
mean rate of work required to maintain the rotation of the loop against
this torque is equal to the mean ohmic power loss in the loop.
Hint: It may be helpful
to treat the loop as a magnetic dipole. Neglect the self-inductance of the loop.
\item Consider a long, uniformly wound, cylindrical solenoid of length $l$,
radius $r$, and turns per unit length $N$. Suppose that the solenoid is
wound around a ferromagnetic core of permeability $\mu$. What is the
self-inductance of the solenoid?
\item A cable consists of a long cylindrical conductor  of radius $a$ which carries current uniformly
distributed over its cross-section. The current returns in a thin insulated sheath  on the surface of the cable. Find the self-inductance per unit length
of the cable.
\item Consider two co-planar and concentric circular wire loops of radii
$a$ and $b$, where $a\ll b$. What is the mutual inductance of the loops?
Suppose that the smaller loop is shifted a distance $z$ out of the
plane of the larger loop (whilst remaining co-axial with the larger loop).
What now is the mutual inductance of the two loops?
\item Two small current loops are sufficiently far apart that they interact
like two magnetic dipoles. Suppose that the loops have position vectors ${\bf r}_1$ and ${\bf r}_2$, cross-sectional
areas $A_1$ and $A_2$, and unit normals ${\bf n}_1$ and ${\bf n}_2$, respectively. What is the mutual inductance of the loops?
\item A circular loop of wire of radius $a$ lies in the plane of a long
straight wire, with its center a perpendicular distance $b > a$ from the wire.
Find the mutual inductance of the two wires.
\item An electric circuit consists of a resistor, $R$, a capacitor, $C$, and
an inductor, $L$, connected in series with a switch, and a battery of constant voltage
$V$. Suppose that the switch is turned on at $t=0$. What current subsequently flows in the circuit? Consider the three cases $\omega_0> \nu$,
$\omega_0=\nu$, and $\omega_0< \nu$ separately, where $\omega_0=1/\sqrt{L\,C}$, and $\nu = R/(2\,L)$.
\item A coil of self-inductance $L$ and resistance $R$ is connected
in series with a switch, and  a battery of constant voltage $V$. The
switch is closed, and the steady current $I=V/R$
is established in the circuit. The switch is then opened at $t=0$. Find 
the current as a function of time, for $t >0$. 
\item A steady voltage is suddenly applied to a coil of self-inductance $L_1$ in the presence of a nearby closed second coil of self-inductance $L_2$. Suppose that the mutual inductance of the two coils is $M$. Demonstrate
that the presence of the second coil effectively decreases the initial self-inducatance of the first coil from $L_1$ to $L_1-M^2/L_2$. 
\item An alternating circuit consists of a resistor, $R_1$, and an inductor, $L_1$, in series with an alternating voltage source of peak voltage $V_1$
and angular frequency $\omega$. This circuit is inductively coupled
to a closed wire loop of self-inductance $L_2$ and resistance $R_2$. Let
$M$ be the mutual inductance of the two circuits. Find the impedance of
the first circuit. Demonstrate that the presence of the second circuit
causes the effective resistance of the first circuit to increase to
$$
R_1 + \frac{\omega^2\,M^2\,R_2}{(R_2^{\,2} + \omega^2\,L^2)},
$$
and its effective inductance to decrease to
$$
L_1 - \frac{\omega^2\,M^2\,L_2}{(R_2^{\,2} + \omega^2\,L^2)}.
$$
\item An alternating circuit consists of a coil and a capacitor connected
in parallel across an alternating voltage source of angular frequency $\omega$. Suppose that the coil has self-inductance $L$, and resistance $R$. Find
the impedance of the circuit. Demonstrate that if $L\gg C\,R^2$ then the amplitude
of the current drawn from the voltage source goes through a minimum at
$\omega=1/\sqrt{L\,C}$. 
\item Repeat the calculation of Exercise~7.1, taking into account the self-inductance, $L$, of the loop.
\item Find the characteristic impedance of a transmission line consisting of
two identical parallel cylindrical wires of radius $a$ and spacing $d$. 
\item Suppose that a transmission line has an inductance per unit length, $L$,
a capacitance per unit length, $C$, and a resistance per unit length, $R$. 
Demonstrate that a signal sent down the line decays exponentially
on the characteristic length-scale $l = 2\,L/(R\,v)$, where $v$ is the
propagation velocity. You may assume that $l$ is much longer
than the wavelength of the signal.
\item Three co-axial cables of impedance $Z_0$ have their central
conductors connected via three identical resistors of resistance $R$, as shown in the diagram. The outer conductors are all earthed.
What must the value $R$ be in order to ensure that there is no reflection of signals coming into the
junction from any cable?
\epsfysize=2in
\centerline{\epsffile{chapter7/fig7.8.eps}}
\end{enumerate}
\renewcommand{\theenumi}{arabic{enumi}}
}
